% !TEX root = ../Guia SGSI.tex

\subsubsection{Procedimiento de seguridad de la información para las relaciones con proveedores}

Con respecto a las normas referentes a la seguridad de la información accesible por los proveedores, se identifican a los prestadores de servicio, terceros y colaboradores que realicen alguna actividad en la Organización, como puedan ser:

\begin{enumerate}
    \item Proveedores de servicios: consultores, auditores, desarrolladores, personal de soporte, servicios gestionados, outsourcing, etc.
    \item Clientes.
    \item Estudiantes en prácticas o personal Temporal.
\end{enumerate}

\Beneficiario{} cuenta con un sistema para identificar los riesgos en función del tipo de actividades a realizar, y propone una serie de salvaguardas y medidas compensatorias destinadas a minimizar los riesgos detectados.

Para dichas actividades, se definen los Términos y Condiciones de los Contratos teniendo en cuenta, al menos, los siguientes puntos:

\begin{enumerate}
    \item El Catálogo de Patrimonio.
    \item Requisitos específicos de la Organización (de seguridad, horarios, ubicación, etc.).
    \item Aceptación de la Política de Seguridad de la Información y Normativas vigentes asociadas.
    \item Términos de uso de la información y los sistemas que la soportan.
    \item Descripción de los servicios o actividades, tipo de accesos y permisos necesarios.
    \item Derechos de auditoría y revisión por parte de \Beneficiario{}.
\end{enumerate}

\Beneficiario{} no mantiene reuniones de seguimiento con los proveedores de cada uno de los proyectos para evaluar si las características del proyecto cumplen con los niveles contratados y con los requisitos exigidos en lo relativo a la seguridad de la información.

\Beneficiario{} debe implantar una política que establezca sesiones de seguimiento para verificar que se cumplen los estándares que garantizan la seguridad de su información en las relaciones con proveedores y clientes.

\Beneficiario{} no define condiciones de finalización de los contratos o acuerdos con sus clientes y proveedores en relación con la destrucción o devolución de activos o información propiedad de la Organización. \Beneficiario{} debe establecer un protocolo de destrucción de información interna tras la finalización de sus contratos con clientes y proveedores para evitar posibles exfiltraciones y accesos no autorizados.

Para tener acceso a la red de \Beneficiario{}, todo tercero se asegura que sus propios sistemas están conectados de manera consistente con los requisitos exigidos, inclusive, sin límites, del derecho a auditar sin previo aviso las medidas de seguridad de la información de aquellos sistemas conectados y el derecho a terminar de inmediato las conexiones de todos los sistemas de terceros.

Todo contrato estipula que toda la información recopilada para la realización de un proyecto o trabajo será entregada al momento que ésta lo requiera, sin coste adicional.

Todo proveedor de servicios de aplicaciones que maneje información de producción de \Beneficiario{} deposita periódicamente la versión más reciente del código y suministrar documentación actualizada y detallada relacionada.

Las decisiones referentes a quiénes tendrán acceso a la información y sistemas de información que dan servicio a \Beneficiario{}, gestionados o supervisados, son tomados únicamente por los propietarios de estos.

Cuando \Beneficiario{} cuente con algún servicio subcontratado a un tercero que implique un tratamiento de datos de carácter personal, el encargado del tratamiento de los datos suele ser una entidad diferente al responsable. Por tanto, existirá un acceso a los datos de carácter personal por cuenta de terceros. Como consecuencia de dicho acceso se aplican las normas de tratamiento de datos de carácter personal establecidas en el Procedimiento sobre tratamiento y acceso a datos de carácter personal de proveedores.

\textbf{Seguimiento y Revisión de los Servicios de los Proveedores}

\begin{enumerate}
    \item La responsabilidad de la gestión de las relaciones con los proveedores se asigna a una persona individual o al equipo de gestión del servicio.
    \item Se asegura de que los proveedores asignen responsabilidades internas para revisar el cumplimiento y la aplicación de los requisitos de los acuerdos.
    \item Están disponibles los suficientes recursos técnicos para supervisar, en particular, que se estén cumpliendo los requisitos de seguridad de la información del acuerdo.
    \item Se toman las medidas apropiadas cuando se observan deficiencias en la prestación de servicios.
    \item Se mantiene suficiente control y visibilidad general sobre todos los aspectos de seguridad de la información o de las instalaciones de tratamiento de la información, sensibles o críticas, que se accedan, procesen o gestionen por un proveedor.
    \item Se mantiene la visibilidad de las actividades de seguridad, tales como la gestión del cambio, la identificación de las vulnerabilidades y la notificación de incidentes de seguridad de información y la respuesta a través de un proceso definido de reporte.
\end{enumerate}