% !TEX root = ../Guia SGSI.tex

\subsubsection{Normativa de uso de los sistemas de información}

\begin{enumerate}[label=\alph*)]

    \item \textbf{General}

    \textbf{Regulación} Las siguientes instrucciones serán de aplicación en la observancia del cumplimiento del Reglamento (UE) 2016/679 del Parlamento Europeo y del Consejo, de 27 de abril de 2016, relativo a la protección de las personas físicas en lo que respecta al tratamiento de datos personales y a la libre circulación de estos datos, y de la Ley Orgánica de Protección de Datos y Garantía de los Derechos Digitales.

    \textbf{Obligaciones} Dado que esta normativa trata de salvaguardar un derecho fundamental mediante la adopción de diferentes medidas de seguridad, técnicas y organizativas, el Usuario deberá atender a las siguientes obligaciones:

    \begin{table}[H]
        \centering
        \small
        \tablacentrada{3.5cm}{
            \hline
            \textbf{Deber de secreto} & Guardar el necesario secreto respecto a cualquier tipo de información de carácter personal conocida en función del trabajo desarrollado, incluso una vez concluida la relación con \Beneficiario{}. \\ \hline
            \textbf{Contraseñas} &
            \begin{itemize}[label={}, leftmargin=0pt, topsep=0pt, itemsep=0pt]
                \item Las contraseñas de acceso al sistema informático son personales e intransferibles. El Usuario será el único responsable de las consecuencias derivadas de su mal uso, divulgación o pérdida.
                \item Está prohibido emplear identificadores y contraseñas de otros Usuarios para acceder al sistema informático.
                \item Los usuarios deberán utilizar contraseñas seguras.
            \end{itemize} \\ \hline
            \textbf{Bloqueo del puesto} & El Usuario deberá bloquear la sesión en caso de ausentarse temporalmente de su puesto de trabajo para evitar accesos no autorizados. \\ \hline
            \textbf{Almacenamiento de archivos} & Los ficheros de carácter personal deberán ser guardados en el espacio habilitado por la Organización para facilitar la realización de copias de seguridad y proteger el acceso frente a personas no autorizadas. \\ \hline
            \textbf{Manipulación de los archivos} & Solo las personas autorizadas podrán introducir, modificar o anular los datos personales contenidos en los ficheros. \\ \hline
            \textbf{Ficheros temporales} &
            \begin{itemize}[label={}, leftmargin=0pt, topsep=0pt, itemsep=0pt]
                \item Los ficheros temporales se borrarán una vez hayan dejado de ser necesarios para los fines que motivaron su creación.
                \item Mientras estén vigentes, deberán ser almacenados en la carpeta habilitada en la red informática o de forma que puedan ser fácilmente localizados.
            \end{itemize} \\ \hline
            \textbf{Correo electrónico} & No se utilizará el correo electrónico para el envío de información de carácter personal especialmente sensible, salvo que se adopten mecanismos necesarios para evitar que la información sea inteligible o manipulada por terceros. \\ \hline
            \textbf{Violaciones de datos personales} & Se deberán comunicar las violaciones de seguridad de datos personales de las que se tenga conocimiento, de acuerdo con el procedimiento establecido. \\ \hline
        }
        \caption{Obligaciones del Usuario en relación con los sistemas de información}
        \label{tab:guia-sgsi-obligaciones-usuario-sistemas}
        \normalsize
    \end{table}

    \item \textbf{Ficheros en papel}

    En relación con los ficheros en soporte o documento papel, el Usuario deberá observar las diligencias indicadas anteriormente con respecto a la confidencialidad de la información, acceso autorizado a la información en atención a las necesidades de su trabajo, gestión de soportes y documentos, trabajo fuera de las instalaciones de \Beneficiario{}.

    Asimismo, con carácter especial y únicamente de aplicación a los ficheros en papel, el Usuario deberá cumplir además con las siguientes diligencias:

    \begin{table}[H]
        \centering
        \small
        \tablacentrada{3.5cm}{
            \hline
            \celdaTitulo{Elemento} & \celdaTitulo{Diligencia} \\ \hline
            \textbf{Archivadores o dependencias} &
            \begin{itemize}[label={}, leftmargin=0pt, topsep=0pt, itemsep=0pt]
                \item Mantener debidamente custodiadas las llaves de acceso a locales, despachos, armarios o archivadores que contengan documentos con datos personales.
                \item Cerrar con llave la puerta del despacho al término de la jornada o en caso de ausentarse temporalmente.
            \end{itemize} \\ \hline
            \textbf{Almacenamiento de documentos} & Los documentos deberán ser archivados siguiendo los criterios establecidos por la Organización para garantizar su correcta conservación, localización y consulta. \\ \hline
            \textbf{Custodia de documentos} & Los documentos en soporte papel deberán ser custodiados para impedir accesos no autorizados, especialmente fuera de la jornada laboral. \\ \hline
            \textbf{Traslado} & Durante el traslado de documentos, se deberán adoptar medidas para impedir el acceso o manipulación por terceros. \\ \hline
            \textbf{Destrucción} & Los documentos en papel que contengan datos personales deberán ser destruidos mediante destructora de papel u otro medio que garantice la eliminación segura. \\ \hline
            \textbf{Violaciones de seguridad} & Se deberán comunicar las violaciones de seguridad relacionadas con los ficheros en papel al Responsable de Seguridad. \\ \hline
        }
        \caption{Obligaciones del Usuario en relación con los ficheros en papel}
        \label{tab:guia-sgsi-obligaciones-usuario-papel}
        \normalsize
    \end{table}

    \item \textbf{Plazos de retención}

    La información será mantenida durante los plazos que legal o contractualmente sean aplicables, así como los plazos necesarios para responder ante auditorías.

    Cuando por cualquier motivo el tratamiento por \Beneficiario{} de datos de carácter personal debiera ser cancelado, los datos en cuestión serán bloqueados durante los plazos de retención que legal o contractualmente les fueran aplicables. Transcurridos estos plazos, no siendo posible la destrucción de los datos que hayan de ser cancelados, bien sea por imposibilidad técnica o por razón del procedimiento o soporte empleados, el acceso a los datos de carácter personal deberá ser bloqueado permanentemente, hasta que su destrucción fuera posible.

    Ejemplos de plazos de retención:

    \begin{table}[H]
        \centering
        \small
        \tablacentrada{3.5cm}{
            \hline
            \celdaTitulo{Datos} & \celdaTitulo{Plazo de retención} \\ \hline
            Información de empleados & 4 años \\ \hline
            Cámaras de seguridad & 30 días \\ \hline
            Control de accesos de personas físicas & 30 días \\ \hline
            Datos de clientes & Según lo definido en el contrato con cada cliente \\ \hline
        }
        \caption{Ejemplos de plazos de retención}
        \label{tab:guia-sgsi-obligaciones-usuario-plazos-retencion}
        \normalsize
    \end{table}

    Se puede ver con más detalle en el Registro de Actividades de Tratamiento (RAT).

    \begin{itemize}
        \item Las copias de seguridad se realizarán por el Departamento de Sistemas. Se deberá evitar la realización de copias de la información por parte de personas no autorizadas.
        \item Si se almacena localmente información en los dispositivos móviles, se comunicará al Departamento de Sistemas, y se seguirán sus instrucciones para la realización de copias de seguridad.
        \item Si se requiere realizar copias por el usuario se adoptarán las medidas adecuadas para la protección de dichas copias.
        \item No se realizarán copias en dispositivos o sistemas de información para uso privado.
    \end{itemize}

\end{enumerate}
