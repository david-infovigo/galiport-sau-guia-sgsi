% !TEX root = ../Guia SGSI.tex

\subsubsection{Política de seguridad de la información}

\begin{enumerate}[label=\alph*)]
\item Objetivos de la Política de Seguridad de la Información

El objetivo principal de la creación de esta Política de Seguridad de la Información, por parte del Responsable de Seguridad del Sistema de Gestión de la Seguridad de la Información (SGSI) y de la Dirección de \textbf{\Beneficiario}, será el de garantizar, a los clientes y usuarios de los servicios, el acceso a la información con la calidad y el nivel de servicio que se requieren para el desempeño acordado, así como el de evitar serias pérdidas o alteración de la información y accesos no autorizados a la misma.

Se establecerá un marco para la consecución de los objetivos de seguridad de la información para la Organización. Dichos objetivos se alcanzarán a través de una serie de medidas organizativas y de normas concretas y claramente definidas.

Esta Política de Seguridad será mantenida, actualizada y adecuada a los fines de la Organización. Además, se compartirá con todos los trabajadores de \textbf{\Beneficiario}.

Los principios que tendrán que respetarse, en base a las dimensiones básicas de la seguridad, serán los siguientes:

\begin{itemize}
  \item \textbf{Confidencialidad:} propiedad por la cual únicamente puede acceder a la información gestionada por la Organización quién esté autorizado para ello, previa identificación, en el momento y por los medios habilitados.

  \item \textbf{Integridad:} propiedad que garantiza la validez, exactitud y completitud de la información gestionada por la Organización, siendo su contenido el facilitado por los afectados sin ningún tipo de manipulación y permitiendo que sea modificada únicamente por quién esté autorizado para ello.

  \item \textbf{Disponibilidad:} propiedad de ser accesible y utilizable en los intervalos acordados. La información gestionada por la Organización es accesible y utilizable por los clientes y usuarios autorizados e identificados en todo momento, quedando garantizada su propia persistencia ante cualquier eventualidad prevista.
\end{itemize}

Adicionalmente, dado que cualquier Sistema de Gestión de la Seguridad de la Información debe cumplir con la legislación vigente, se atenderá al siguiente principio:

\begin{itemize}
  \item \textbf{Legalidad:} referido al cumplimiento de las leyes, normas, reglamentaciones o disposiciones a las que está sujeta la Organización, especialmente en materia de protección de datos personales.
\end{itemize}

\item \textbf{Alcance}

Esta Política se aplicará a los sistemas de información de \Beneficiario, relacionados con el ejercicio de sus competencias y a todos los usuarios con acceso autorizado a los mismos. Todos ellos tienen la obligación de conocer y cumplir esta Política de Seguridad de la Información y su Normativa de Seguridad derivada, siendo responsabilidad del Comité de Seguridad disponer los medios necesarios para que la información llegue al personal afectado.

\item \textbf{Gestión del riesgo}

La gestión de la Seguridad de la Información en \Beneficiario será basada en el riesgo, de conformidad con la Norma internacional ISO/IEC 27001.

Se articulará mediante un proceso general de apreciación y tratamiento del riesgo, que potencialmente puede afectar a la seguridad de la información de los servicios prestados, consistente en:

\begin{itemize}
    \item Identificar las amenazas, que aprovecharán vulnerabilidades de los Sistemas de Información que soportan, o de los que depende, la seguridad de la información.
    \item Analizar el riesgo, en base a la consecuencia de materializarse la amenaza y de la probabilidad de ocurrencia.
    \item Evaluar el riesgo, según un nivel previamente establecido y aprobado de riesgo ampliamente aceptable, tolerable e inaceptable.
    \item Tratar el riesgo inaceptable, mediante los controles o salvaguardas adecuadas.
\end{itemize}

Dicho proceso será cíclico y se llevará a cabo de forma periódica, como mínimo una vez al año. Para cada riesgo identificado se asignará un propietario, pudiendo recaer múltiples responsabilidades en una misma persona o comité.

\item \textbf{Marco normativo}

El marco legal en materia de seguridad de la información viene establecido por la siguiente legislación:

\begin{itemize}
    \item El Real Decreto 311/2022, de 3 de mayo, por el que se regula el Esquema Nacional de Seguridad.
    \item La Ley Orgánica 3/2018, de 5 de diciembre, de Protección de Datos Personales y garantía de los derechos digitales, que tiene por objeto adaptar el ordenamiento jurídico español al Reglamento (UE) 2016/679 del Parlamento Europeo y el Consejo, de 27 de abril de 2016, relativo a la protección de las personas físicas en lo que respecta al tratamiento de sus datos personales y a la libre circulación de estos datos, y completar sus disposiciones.
    \item Reglamento (UE) 2016/679 del Parlamento Europeo y del Consejo, de 27 de abril de 2016, relativo a la protección de las personas físicas en lo que respecta al tratamiento de datos personales y a la libre circulación de estos datos y por el que se deroga la Directiva 95/46/CE (RGPD).
    \item Ley 6/2020, de 11 de noviembre, reguladora de determinados aspectos de los servicios electrónicos de confianza, cuyo objeto es la regulación de determinados aspectos de los servicios electrónicos de confianza, como complemento del Reglamento (UE) n.º 910/2014 del Parlamento Europeo y del Consejo, de 23 de julio de 2014, relativo a la identificación electrónica y los servicios de confianza para las transacciones electrónicas en el mercado interior (Reglamento eIDAS).
\end{itemize}

\item \textbf{Objetivos del SGSI}

El SGSI de \Beneficiario deberá garantizar:

\begin{itemize}
    \item Que se desarrollen políticas, normativas, procedimientos y guías operativas para apoyar la política de seguridad de la información.
    \item Que se identifique la información que deba ser protegida.
    \item Que se establezca y mantenga la gestión del riesgo alineada con los requerimientos de la política del SGSI y la estrategia de la Organización.
    \item Que se establezca una metodología para la apreciación y tratamiento del riesgo.
    \item Que se establezcan criterios con los que medir el nivel de cumplimiento del SGSI.
    \item Que se revise el nivel de cumplimiento del SGSI.
    \item Que se corrijan las no conformidades mediante la implementación de acciones correctivas.
    \item Que el personal reciba formación y concienciación sobre la seguridad de la información.
    \item Que todo el personal sea informado sobre la obligación de cumplimiento de la política de seguridad de la información.
    \item La asignación de los recursos necesarios para gestionar el SGSI.
    \item La identificación y cumplimiento de todos los requerimientos legales, regulatorios y contractuales.
    \item Que se identifiquen y analicen las implicaciones de seguridad de la información respecto a los requerimientos de negocio.
    \item Que se mida el grado de madurez del propio sistema de gestión de la seguridad de la información.
    \item Que se realice la mejora continua sobre el SGSI.
\end{itemize}

\item \textbf{Criterios de la seguridad de la informaci\'on}

\Beneficiario{} establece las siguientes acciones para organizar la Seguridad de la Informaci\'on:

\begin{itemize}
  \item Designar\'a roles de seguridad: responsables unificados de Servicios y de la Informaci\'on, Responsable de la Seguridad, Responsable del Sistema y Delegado de Protecci\'on de Datos.
  \item Constituir\'a un \'organo consultivo y estrat\'egico para la toma de decisiones en materia de Seguridad de la Informaci\'on. Este \'{o}rgano se denominar\'a Comit\'e de Seguridad de la Informaci\'on.
\end{itemize}

\item \textbf{Definici\'on de Roles y Responsabilidades}

La estructura organizativa de seguridad, y jerarqu\'ia en el proceso de decisiones la componen:

\begin{tabular}{|p{4cm}|p{11cm}|}
\hline
\textbf{Rol} & \textbf{Funciones} \\
\hline
Direcci\'on & \'{O}rganos colegiados o unipersonales que deciden la misi\'on y los objetivos de la Organizaci\'on. \\
\hline
Comit\'e de Seguridad y Privacidad & \'{O}rganos colegiados o unipersonales que toman decisiones que concretan c\'omo alcanzar los objetivos de seguridad y protecci\'on de la privacidad marcados por los \'{o}rganos de gobierno. \\
\hline
Responsable de la Informaci\'on & A nivel de gobierno. Tiene la responsabilidad \'{u}ltima sobre qu\'e seguridad requiere una cierta informaci\'on manejada por la Organizaci\'on. \\
\hline
Responsable de Servicio & A nivel de gobierno o, en ocasiones baja a nivel ejecutivo. Tiene la responsabilidad \'{u}ltima de determinar los niveles de servicio aceptables por la Organizaci\'on. \\
\hline
Responsable de Seguridad & A nivel ejecutivo. Funciona como supervisor de la operaci\'on del sistema y veh\'iculo de reporte al Comit\'e de Seguridad de la Informaci\'on. \\
\hline
Responsable del Sistema & A nivel operacional. Toma decisiones operativas: arquitectura del sistema, adquisiciones, instalaciones y operaci\'on del d\'ia a d\'ia. \\
\hline
Administradores de seguridad & Son las personas encargadas de ejecutar las acciones diarias de operaci\'on del sistema seg\'un las indicaciones recibidas de sus superiores jer\'arquicos. \\
\hline
Delegado de Protecci\'on de Datos & Figura obligatoria para administraciones p\'ublicas, es el encargado de asesorar y supervisar todos los aspectos relacionados con el tratamiento de datos de car\'acter personal, incluidos los aspectos de seguridad (integridad, confidencialidad y disponibilidad) y violaci\'on de datos personales. Su nombramiento se produce por otra v\'ia ya que sus cometidos no se ci\~nen \'{u}nicamente a aspectos de seguridad. \\
\hline
\end{tabular}

\vspace{1em}

En el caso de PYMES, donde los equipos especializados como los Responsables de Informaci\'on, Seguridad de la Informaci\'on, Sistema y el Delegado de Protecci\'on de Datos (DPD) pueden no estar definidos de manera individual, estas responsabilidades se agrupan de forma m\'as compacta. La direcci\'on de la empresa, el responsable de TI y el DPD (en caso de que exista), se encargan de asumir estas funciones de manera coordinada.

\textbf{Responsables de Informaci\'on y de los Servicios}

Ser\'an funciones de los Responsables de la Informaci\'on y de los Servicios:

\begin{itemize}
  \item Establecer los requisitos de seguridad aplicables a la Informaci\'on (niveles de seguridad de la Informaci\'on) y a los Servicios (niveles de seguridad de los servicios) pudiendo recabar una propuesta al Responsable de Seguridad y teniendo en cuenta la opini\'on del Responsable del Sistema.
  \item Dictaminar respecto a los derechos de acceso a la informaci\'on y los servicios.
  \item Aceptar los niveles de riesgo residual que afectan a la informaci\'on y los servicios.
  \item Poner en comunicaci\'on del Responsable de Seguridad cualquier variaci\'on respecto a la Informaci\'on y los Servicios de los que es responsable, especialmente la incorporaci\'on de nuevos Servicios o Informaci\'on a su cargo. El cual dar\'a traslado de dichos cambios, al Comit\'e de Seguridad de la Informaci\'on, en su pr\'oxima reuni\'on.
  \item Tiene la responsabilidad \'{u}ltima del uso que se haga de determinados servicios e informaci\'on y, por tanto, de su protecci\'on.
\end{itemize}

\textbf{Responsable de la Seguridad de la informaci\'on}

Ser\'an funciones del Responsable de Seguridad:

\begin{itemize}
  \item Mantener y verificar el nivel adecuado de seguridad de la Informaci\'on manejada y de los Servicios electr\'onicos prestados por los sistemas de informaci\'on.
  \item Promover la formaci\'on y concienciaci\'on en materia de seguridad de la informaci\'on.
  \item Designar responsables de la ejecuci\'on del an\'alisis de riesgos, de la Declaraci\'on de Aplicabilidad, identificar medidas de seguridad, determinar configuraciones necesarias y elaborar documentaci\'on del sistema.
  \item Aprobar la Declaraci\'on de Aplicabilidad a partir de las medidas de seguridad requeridas conforme al Anexo II del ENS y la ISO 27001.
  \item Proporcionar asesoramiento para la determinaci\'on de la Categor\'ia del Sistema, en colaboraci\'on con el Responsable del Sistema y/o Comit\'e de Seguridad.
  \item Participar en la elaboraci\'on e implantaci\'on de los planes de mejora de la seguridad y, llegado el caso, en los planes de continuidad, procediendo a su validaci\'on.
  \item Gestionar las revisiones externas o internas del sistema, as\'i como los procesos de certificaci\'on.
  \item Elevar al Comit\'e de Seguridad la aprobaci\'on de cambios y otros requisitos del sistema.
  \item Aprobar los procedimientos de seguridad que forman parte del Mapa Normativo (y no son competencia del Comit\'e) y poner en conocimiento al Comit\'e de las modificaciones que se hayan realizado a lo largo del periodo en curso.
  \item Participar\'a en la elaboraci\'on, en el marco del Comit\'e de Seguridad de la Informaci\'on, de la Pol\'itica de Seguridad de la Informaci\'on, para su aprobaci\'on por Direcci\'on.
  \item Actuar\'a como secretario del Comit\'e de Seguridad de la Informaci\'on, realizando las siguientes funciones:
  \begin{itemize}
    \item Convocar las reuniones del Comit\'e de Seguridad de la Informaci\'on.
    \item Preparar los temas a tratar en las reuniones del Comit\'e, aportando informaci\'on puntual para la toma de decisiones.
    \item Elaborar el acta de las reuniones.
    \item Es responsable de la ejecuci\'on directa o delegada de las decisiones del Comit\'e.
  \end{itemize}
\end{itemize}

\textbf{Responsable del Sistema}

Ser\'an funciones del Responsable del Sistema:

\begin{itemize}
  \item Desarrollar, operar y mantener el sistema de informaci\'on durante todo su ciclo de vida, elaborando los procedimientos operativos necesarios.
  \item Definir la topolog\'ia y la gesti\'on del Sistema de Informaci\'on estableciendo los criterios de uso y los servicios disponibles en el mismo.
  \item Detener el acceso a informaci\'on o prestaci\'on de servicio si tiene el conocimiento de que estos presentan deficiencias graves de seguridad.
  \item Cerciorarse de que las medidas espec\'ificas de seguridad se integren adecuadamente dentro del marco general de seguridad.
  \item Proporcionar asesoramiento para la determinaci\'on de la Categor\'ia del Sistema, en colaboraci\'on con el Responsable de Seguridad y/o Comit\'e de Seguridad de la Informaci\'on.
  \item Participar en la elaboraci\'on e implantaci\'on de los planes de mejora de la seguridad y llegado el caso en los planes de continuidad.
  \item Coordinar las funciones del administrador de la seguridad del sistema:
  \begin{itemize}
    \item La gesti\'on, configuraci\'on y actualizaci\'on, en su caso, del hardware y software.
    \item La gesti\'on de las autorizaciones concedidas a los usuarios del sistema, en particular los privilegios concedidos.
    \item Aprobar los cambios en la configuraci\'on vigente del Sistema de Informaci\'on.
    \item Asegurar que los controles de seguridad establecidos son cumplidos estrictamente.
    \item Asegurar que son aplicados los procedimientos aprobados para manejar el Sistema de Informaci\'on.
    \item Supervisar las instalaciones de hardware y software, sus modificaciones y mejoras.
    \item Monitorizar el estado de seguridad.
    \item Informar al Responsable de Seguridad de cualquier anomal\'ia, compromiso o vulnerabilidad relacionada con la seguridad.
    \item Colaborar en la investigaci\'on y resoluci\'on de incidentes de seguridad, desde su detecci\'on hasta su resoluci\'on.
  \end{itemize}
\end{itemize}

\textbf{Delegado de Protecci\'on de Datos (DPD)}

Ser\'an funciones del Delegado de Protecci\'on de Datos:

\begin{itemize}
    \item Informar y asesorar a la organizaci\'on, y a los usuarios que se ocupen del tratamiento, de las obligaciones que les incumben en virtud de la normativa vigente en materia de Protecci\'on de Datos.
    \item Supervisar el cumplimiento de lo dispuesto en normativa de seguridad y de las pol\'iticas internas de la organizaci\'on, en materia de protecci\'on de datos, incluida la asignaci\'on de responsabilidades, la concienciaci\'on y formaci\'on del personal que participa en las operaciones de tratamiento, y las auditor\'ias correspondientes.
    \item Ofrecer el asesoramiento que se le solicite acerca de la evaluaci\'on de impacto relativa a la protecci\'on de datos y supervisar\'a su aplicaci\'on.
    \item Cooperar con la Agencia Espa\~nola de Protecci\'on de Datos cuando \'esta lo requiera, actuando como punto de contacto con \'esta para cuestiones relativas al tratamiento de datos.
    \item El Delegado de Protecci\'on de Datos desempe\~nar\'a sus funciones prestando atenci\'on a los riesgos asociados a las operaciones de tratamiento. Para ello, debe ser capaz de:
    \item Recabar informaci\'on para determinar las actividades de tratamiento.
    \item Analizar y comprobar la conformidad de las actividades de tratamiento.
    \item Informar, asesorar y emitir recomendaciones al responsable o el encargado del tratamiento.
    \item Recabar informaci\'on para supervisar el registro de las operaciones de tratamiento.
    \item Asesorar en el principio de la protecci\'on de datos por dise\~no y por defecto.
    \item Asesorar sobre si se lleva a cabo o no las evaluaciones de impacto, metodolog\'ia, salvaguardas a aplicar, etc.
    \item Priorizar actividades en base a los riesgos.
    \item Asesorar al Responsable de Tratamiento sobre \'{a}reas a cometer a auditor\'ias, actividades de formaci\'on a realizar y operaciones de tratamiento a dedicar m\'as tiempo y recursos.
\end{itemize}

\item \textbf{Comit\'e de Seguridad de la Informaci\'on}

\Beneficiario{} debe crear un Comit\'e de Seguridad para coordinar la protecci\'on de datos, gestionar riesgos y garantizar el cumplimiento normativo, fortaleciendo as\'i la cultura de seguridad en toda la organizaci\'on.

Este comit\'e deber\'a estar compuesto por los siguientes miembros que se clasificar\'an en permanentes o no permanentes atendiendo a la obligatoriedad de la participaci\'on del Comit\'e de Seguridad de la Informaci\'on:

\textbf{Miembros permanentes:}
\begin{itemize}
  \item Direcci\'on
  \item Responsable de Sistema (RSIS)
  \item Responsable de Seguridad de la Informaci\'on (RSEG)
  \item Asesores que se consideren oportunos para los temas en cuesti\'on con voz, pero sin voto.
\end{itemize}

\textbf{Miembros no permanentes:}
\begin{itemize}
  \item Responsables del Servicio y de la Informaci\'on.
  \item El Delegado de Protecci\'on de Datos.
  \item Representantes de la organizaci\'on, especialistas externos del sector p\'ublico o privado cuya presencia, por raz\'on de su experiencia o vinculaci\'on con los asuntos tratados, sea necesaria o aconsejable.
  \item Los Responsables de la Informaci\'on y los Servicios ser\'an convocados por la presidencia en funci\'on de los asuntos a tratar, en representaci\'on de los distintos \'{a}mbitos o \'{a}reas de seguridad TIC. Cada \'{a}rea estar\'a representada por un vocal con voto, sin perjuicio de que acudan varios representantes de la misma.
  \item El Delegado de Protecci\'on de Datos participar\'a con voz, pero sin voto, en las reuniones del Comit\'e de Seguridad de la Informaci\'on cuando se aborden cuestiones relacionadas con el tratamiento de datos de car\'acter personal, o cuando se requiera su participaci\'on. En todo caso, si un asunto se sometiese a votaci\'on, se har\'a constar en acta la opini\'on del Delegado de Protecci\'on de Datos.
  \item El Secretario/a del Comit\'e realizar\'a las convocatorias y levantar\'a actas de las reuniones del Comit\'e de Seguridad. A las sesiones podr\'an asistir asesores invitados por el Presidente.
\end{itemize}

En una PYME donde no es posible disponer de todos los cargos necesarios para el Comit\'e, \Beneficiario{} podr\'a designar miembros con m\'ultiples roles o incorporar asesores externos. Aunque reducido, este comit\'e sigue siendo esencial para coordinar la seguridad, gestionar riesgos y fomentar una cultura de protecci\'on.

\textbf{Funciones del Comit\'e de Seguridad:}
\begin{itemize}
  \item Atender las inquietudes de la Alta Direcci\'on y de los diferentes departamentos.
  \item Informar regularmente del estado de la seguridad de la informaci\'on a la Alta Direcci\'on.
  \item Promover la mejora continua del SGSI.
  \item Elaborar la estrategia de evoluci\'on de la seguridad de la informaci\'on.
  \item Promover la realizaci\'on de auditor\'ias peri\'odicas.
  \item Aprobar documentaci\'on de seguridad.
  \item Estar informado sobre la normativa y entidades relacionadas con la certificaci\'on del ENS.
  \item Proponer directrices y recomendaciones, que se reflejar\'an en actas.
  \item Coordinar esfuerzos de diferentes \'{a}reas en seguridad.
  \item Resolver conflictos de responsabilidad o escalarlos si fuera necesario.
  \item Asesorar en materia de seguridad cuando se le requiera.
  \item Revisar la Pol\'itica de Seguridad antes de su aprobaci\'on.
  \item Aprobar el Plan de Adecuaci\'on para la implantaci\'on del ENS.
\end{itemize}

\textbf{Periodicidad de reuniones y adopci\'on de acuerdos:}
\begin{itemize}
  \item El Comit\'e se reunir\'a al menos una vez al a\~no, pudiendo convocarse m\'as a menudo si es necesario.
  \item Las reuniones se convocar\'an por su Presidencia, por iniciativa propia o a petici\'on de la mayor\'ia de sus miembros permanentes.
  \item Las decisiones se tomar\'an por consenso de los miembros permanentes.
\end{itemize}

\textbf{Designaci\'on y resoluci\'on de conflictos:}
\begin{itemize}
  \item La creaci\'on y constituci\'on del Comit\'e, as\'i como la designaci\'on de sus miembros, se formalizar\'a mediante acta inicial.
  \item Los roles nombrados se renovar\'an autom\'aticamente cada a\~no. Cualquier cambio se comunicar\'a siguiendo el procedimiento establecido.
  \item Seg\'un el art\'iculo 13.3 del RD del ENS, no puede existir dependencia jer\'arquica entre el RSEG y el RSIS, salvo excepciones justificadas. En tal caso, se establecer\'an medidas compensatorias.
\end{itemize}

\item \textbf{Datos personales y riesgos derivados del tratamiento}

El legislador, en el art\'iculo 12.1.f) del RD del ENS, hace referencia al cumplimiento adecuado en materia de protecci\'on de datos personales. Para dar cumplimiento a la LOPD-GDD y al RGPD, \Beneficiario{}:

\begin{itemize}
  \item Publicar\'a el registro de actividades de tratamiento.
  \item Realizar\'a an\'alisis de riesgos y evaluaciones de impacto (EIPD) cuando sea necesario.
  \item Aplicar\'a las fases indicadas en los Anexos I y II del RD del ENS para los sistemas afectados.
\end{itemize}

\item \textbf{Obligaciones del personal}

Todo el personal, tanto interno como externo, que interact\'ue con el sistema de informaci\'on, deber\'a cumplir con la presente Pol\'itica de Seguridad de la Informaci\'on.

\item \textbf{Formaci\'on y concienciaci\'on}

El Responsable de Seguridad del SGSI deber\'a garantizar que todo el personal involucrado:

\begin{itemize}
  \item Conoce la pol\'itica, objetivos y procesos del SGSI.
  \item Recibe acciones formativas y de concienciaci\'on.
  \item Tiene acceso a la documentaci\'on correspondiente seg\'un su rol.
\end{itemize}

\item \textbf{Terceras partes}

Cuando \Beneficiario{} preste servicios o maneje informaci\'on de terceros:

\begin{itemize}
  \item Se les har\'a part\'icipes de esta Pol\'itica de Seguridad.
  \item Se establecer\'an canales de reporte y coordinaci\'on con comit\'es correspondientes.
  \item Deber\'an aplicar esta normativa, pudiendo desarrollar procedimientos propios.
  \item Su personal recibir\'a formaci\'on equivalente a la de esta pol\'itica.
  \item Si no pueden cumplir alg\'un punto, el Responsable de Seguridad emitir\'a informe de riesgos que deber\'a ser aprobado por los responsables implicados.
  \item Se establecer\'a un Punto Operacional de Comunicaci\'on (POC) para el enlace.
\end{itemize}

\item \textbf{Auditor\'ia}

La Direcci\'on General de \Beneficiario{} garantizar\'a, mediante auditor\'ias internas y/o externas, el cumplimiento de esta pol\'itica, aplicando medidas correctoras necesarias para la mejora continua.

\item \textbf{Validez y actualizaci\'on}

Esta pol\'itica ser\'a efectiva desde su publicaci\'on y se revisar\'a anualmente. En la revisi\'on se considerar\'an:

\begin{itemize}
  \item Cambios en el contexto interno o externo de la organizaci\'on.
  \item Incidentes y No Conformidades.
  \item Resultados de los procesos de apreciaci\'on de riesgo.
  \item Actualizaci\'on de normas y documentos relacionados.
\end{itemize}

Se elaborar\'a una lista de objetivos y acciones a emprender durante el a\~no siguiente.

\item \textbf{Sanciones}

El incumplimiento de esta pol\'itica y sus procedimientos podr\'a dar lugar a sanciones conforme a la legislaci\'on laboral vigente, seg\'un la magnitud del incumplimiento.

\item \textbf{Ratificaci\'on}

Todos los abajo firmantes asumen y aceptan plenamente el contenido de esta Pol\'itica y se comprometen a aplicarla en sus respectivas \'{a}reas para lograr el correcto funcionamiento del SGSI.

\end{enumerate}