% !TEX root = ../Guia SGSI.tex

\subsubsection{Normativa de acceso a internet}

Con carácter general, los usuarios de \Beneficiario{} disponen de acceso a Internet como herramienta de productividad y conocimiento, así como de mejora de los sistemas de trabajo y búsqueda de información. Esta herramienta es propiedad de la Organización, la cual se reserva el derecho de conceder o anular dichos accesos conforme a los criterios que crea convenientes.

Será necesario garantizar un uso adecuado de los recursos informáticos de acceso a Internet, por los siguientes motivos:

\begin{itemize}
    \item Seguridad: debido al riesgo de infección por software maligno.
    \item Volumen del tráfico externo de datos: garantizando que el acceso a contenidos necesarios para la actividad profesional no se vea perjudicado por el tráfico generado por contenidos no vinculados con las competencias de \Beneficiario{}.
    \item Volumen del tráfico interno de datos: como consecuencia de contenidos descargados de la Web y su posterior almacenamiento. Esta situación aconseja también regular el tipo de ficheros cuya descarga y almacenamiento está permitido.
    \item Ética: es ineludible el compromiso que la Organización debe mantener con la sociedad, a la hora de vetar el acceso a contenidos que pudieran ser poco éticos, ofensivos o delictivos.
\end{itemize}

\begin{enumerate}[label=\alph*)]

\item \textbf{Limitaciones}

\tablacentrada{3.5cm}{
    \hline
    \textbf{Responsabilidad} &
    \begin{itemize}
    \item Internet es un servicio que \Beneficiario{} pone a disposición de su personal para uso estrictamente profesional.
    \item Los usuarios serán los únicos responsables de las sesiones iniciadas en Internet desde sus terminales de trabajo, y se comprometen a acatar las reglas y normas de funcionamiento establecidas en la presente Normativa.
    \item El acceso a Internet por personal externo requerirá la previa autorización por escrito de la Dirección.
    \end{itemize} \\ \hline
    \textbf{Monitorización} &
    \begin{itemize}
    \item La Organización se reservará el derecho a filtrar el contenido al que el usuario puede acceder a través de Internet desde los recursos y servicios propiedad de la Organización, así como a monitorizar y registrar los accesos realizados desde los mismos.
    \item En caso de que un usuario considere necesario acceder a alguna dirección incluida en una de las categorías filtradas, se pondrá en contacto con su responsable directo para que éste gestione el acceso correspondiente.
    \end{itemize} \\ \hline
}

\tablacentrada{3.5cm}{
    \hline
    \textbf{Usos no permitidos que implican un riesgo para la seguridad} &
    \begin{itemize}
    \item En ningún caso se modificarán las configuraciones de los navegadores (opciones de Internet) de los equipos ni la activación de servidores o puertos sin la autorización expresa. Todos los equipos que así lo estima la empresa, ya estarán configurados para su acceso a Internet.
    \item Se prohibirá expresamente el acceso, la descarga y/o el almacenamiento en cualquier soporte, de páginas con contenidos ilegales, dañinos, inadecuados o que atenten contra la moral y las buenas costumbres y, en general, de todo tipo de contenidos que incumplan las normas éticas y de cortesía de la Organización.
    \item No se permitirá el almacenamiento en los equipos de archivos y contenidos personales descargados vía Internet, especialmente aquellos que violen la legislación vigente relativa a Propiedad Intelectual. Los usuarios deberán respetar y dar cumplimiento a las disposiciones legales de derechos de autor, marcas registradas y derechos de propiedad intelectual de cualquier información visualizada u obtenida mediante Internet haciendo uso de los recursos informáticos o de red de la Organización.
    \item Estará prohibido el uso de Internet mediante los recursos informáticos o de red de la empresa con fines recreativos, así como para obtener o distribuir material violento o pornográfico, o para obtener o distribuir material incompatible con los valores de la Organización.
    \item El uso de chats o programas de conversación en tiempo real estarán restringidos a aquellos que la Organización haya autorizado expresamente.
    \item No se permitirá la descarga de software ejecutable desde internet.
    \end{itemize}
    \\ \hline
    \textbf{Incidencias} &
    \begin{itemize}
    \item Cualquier incidente de seguridad relacionado con la navegación por Internet, deberá ser comunicado sin demora al Departamento de Sistemas.
    \end{itemize} \\ \hline
}

\item \textbf{Normativa para trabajar fuera de las instalaciones}

El trabajo fuera de las instalaciones de \Beneficiario{} comprende tanto el teletrabajo habitual y permanente de los usuarios desplazados, como el trabajo ocasional, usando en ambos casos, dispositivos de computación y comunicación (usualmente: ordenador portátil, Tablet, teléfono móvil, etc.). Este modo de trabajo comprenderá también las conexiones remotas realizadas desde congresos o sesiones de formación, alojamientos o situaciones similares.

Esta modalidad conlleva el riesgo de trabajar en lugares desprotegidos, esto es, sin las barreras de seguridad físicas y lógicas implementadas en las instalaciones de la entidad. Fuera de este perímetro de seguridad aumentan las vulnerabilidades y la probabilidad de materialización de las amenazas, lo que hace necesario adoptar medidas de seguridad adicionales.

Se incluyen, seguidamente, un conjunto de normas de obligado cumplimiento, que tienen como objetivo el reducir el riesgo, que complementan lo ya recogido en la Normativa de uso de los sistemas de información.

\begin{enumerate}[label=\arabic*)]
    \item Uso personal y profesional: Los dispositivos sólo pueden utilizarse para fines profesionales.
    \item Copias de seguridad: Sigue las instrucciones del departamento de informática.
    \item Cierre de sesión: También en casa debes cerrar las sesiones abiertas con tu organización cuando finalices el trabajo.
    \item Doble Factor de Autenticación: Habilítalo para no depender únicamente de la seguridad de tu contraseña.
    \item Normativa interna: Sigue la normativa de seguridad existente.
    \item Prevención Anti-malware: No instales software desde direcciones no seguras, activa el firewall y mantén el antivirus actualizado.
    \item Uso de canales de comunicación establecidos: Utiliza sólo los canales establecidos y autorizados.
    \item Correo electrónico: Extrema tu cautela cuando recibas correos electrónicos no solicitados.
    \item Mantenimiento de los equipos: Cuida de tu equipo y mantenlo actualizado.
    \item Gestión de incidencias: Comunica incidencias según el procedimiento habitual.
\end{enumerate}

\end{enumerate}