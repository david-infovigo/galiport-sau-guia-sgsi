% !TEX root = ../Guia SGSI.tex

\subsubsection{Normativa de protección de correo electrónico, servidores y endpoints}

\begin{enumerate}[label=\alph*)]

\item \textbf{Normativa de uso del correo electrónico}

\textbf{Concepto}. El correo electrónico (e-mail) es un servicio de red para permitir a los usuarios de la Organización enviar y recibir mensajes. Junto con los mensajes también pueden ser enviados ficheros adjuntos.

\textbf{Caracteres}. Las características peculiares de este medio de comunicación (universalidad, bajo coste, anonimato, etc.) han propiciado la aparición de amenazas que utilizan el correo electrónico para propagarse o que aprovechan sus vulnerabilidades.

\textbf{Especificaciones}. \Beneficiario{}, consciente de los problemas de seguridad y responsabilidad legal que ocasiona el uso del correo electrónico, dispondrá de las siguientes especificaciones:

\begin{itemize}
    \item Responsabilidad
    \begin{itemize}
        \item Los usuarios serán responsables de todas las actividades realizadas con las cuentas de acceso y su respectivo buzón de correos provistos por la Organización.
        \item Los usuarios deberán ser conscientes de los riesgos que acarrea el uso indebido de las direcciones de correo electrónico suministradas por la Organización.
        \item Las cuentas de correo son personales e intransferibles. Salvo en casos puntuales para los que deberá solicitarse y obtenerse la correspondiente autorización, no se debe ceder el uso de la cuenta de correo a terceras personas, lo que podría provocar una suplantación de identidad y el acceso a información confidencial.
        \item Los mensajes de correo transmiten información en sus cabeceras (en principio ocultas) que indican datos adicionales del emisor, por lo que deben tenerse en cuenta posibles repercusiones (como daños a la imagen institucional) que podría acarrear una mala utilización de este recurso.
    \end{itemize}
    \item Uso aceptable
    \begin{itemize}
        \item Como norma general no se utilizará la herramienta de correo electrónico con fines ajenos al propio desarrollo de las actividades que cada usuario tiene encomendadas en la Organización.
        \item La utilización del correo electrónico por personal externo requerirá la previa autorización por escrito de la Dirección.
        \item La forma y contenidos de los correos enviados por el usuario estarán alineados con las normas éticas y de cortesía marcadas por la Organización, y en ningún caso se enviarán correos ofensivos, amenazantes o de mal gusto.
        \item El usuario deberá mantener ordenados y clasificados todos sus buzones y carpetas. Los correos inservibles deberán ser eliminados, y todos los archivos adjuntos almacenados en el equipo o unidad de disco habilitada.
    \end{itemize}
    \item Usos que no serán permitidos y que implican un riesgo para la seguridad
    \begin{itemize}
        \item La instalación y uso de cualquier otra aplicación de correo electrónico, así como de una versión diferente de la aplicación homologada que no haya sido autorizada e instalada por el personal técnico autorizado.
        \item La difusión de contenido ilegal; como por ejemplo amenazas, código malicioso, apología del terrorismo, pornografía infantil, software pirata, o de cualquier otra naturaleza delictiva.
        \item El uso no autorizado de servidores propiedad de la Organización para el envío de correo personal.
        \item El envío masivo de correos publicitarios o de cualquier otro tipo que no guarde relación alguna con las necesidades de negocio de la Organización. Este hecho, además, puede llegar a ser interpretado como “spamming”.
        \item La divulgación, independientemente del formato en que se encuentren, de correos que revelen datos del directorio o de usuarios pertenecientes a la Organización, fuera de los límites laborales establecidos por la misma.
        \item En el caso de se requiera enviar un mensaje de correo electrónico a varios destinatarios, se utilizará preferentemente el campo CCO (copia oculta) para introducir las direcciones de correo de los destinatarios, con excepción de aquellos mensajes en los que necesariamente se requiera la identificación de todos los destinatarios para confirmar que han sido informados.
    \end{itemize}
    \item Diligencia
    \begin{itemize}
        \item Los archivos adjuntos recibidos serán analizados por las herramientas antivirus antes de ser abiertos o ejecutados. Los correos sospechosos o de dudosa procedencia no serán abiertos, y menos aún los archivos adjuntos que contengan. Su eliminación debe ser inmediata. Gran parte del código malicioso suele insertarse en ficheros adjuntos, ya sea en forma de ejecutables (.exe, por ejemplo) o en forma de macros de aplicaciones (Word, Excel, etc.).
        \item No se empleará el correo electrónico como medio de comunicación para enviar o recibir información confidencial o que contenga datos de carácter personal. Únicamente, y en aquellos casos en los que sea estrictamente necesario, se utilizará este medio, en cuyo caso, se enviará con las medidas de seguridad apropiadas para cada tipo concreto de información mediante la utilización de un software de cifrado, previa autorización expresa del Responsable de Seguridad.
        \item En la medida de lo posible, se desactivará la vista previa. Utilizar la vista previa para los correos de la bandeja de entrada comporta los mismos riesgos que abrirlos. Del mismo modo, limitar el uso de HTML. El código malicioso puede encontrarse fusionado con el código HTML del mensaje. Desactivar la visualización HTML de los mensajes ayuda a evitar que el código malicioso se ejecute.
        \item Los navegadores utilizados para acceder al correo vía web deberán estar permanentemente actualizados a su última versión, al menos en cuanto a parches de seguridad, así como correctamente configurados.
        \item Una vez finalizada la sesión web, será obligatoria la desconexión con el servidor mediante un proceso que elimine la posibilidad de reutilización de la sesión cerrada.
        \item Se desactivarán las características de recordar contraseñas para el navegador.
        \item Se activará la opción de borrado automático al cierre del navegador, de la información sensible registrada por el mismo: histórico de navegación, descargas, formularios, caché, cookies, contraseñas, sesiones autenticadas, etc.
    \end{itemize}
    \item Incidencias
    Los usuarios deberán comunicar a sus responsables directos sobre cualquier anomalía que detecten en su correo, así como de la apertura de un correo sospechoso o cualquier alerta generada por el antivirus.
    \item Monitorización
    La Organización se reserva el derecho a revisar los ficheros LOG de los servidores, con el fin de comprobar el cumplimiento de estas normas y prevenir actividades que puedan afectar a la Organización como responsable civil subsidiario.
    \item Phishing
    \begin{itemize}
        \item Si se recibe un correo electrónico sospechoso se deberá comprobar la veracidad del remitente por otras vías (teléfono, mensajería instantánea), por ejemplo, comprobar el dominio del correo del remitente y que su nombre coincide con su cuenta de correo electrónico (nombre y dominio).
        \item Se deberá revisar el contenido del mensaje y desconfiar de él si está mal redactado o contiene faltas de ortografía.
        \item Si el correo incluye un enlace sospechoso, no se deberá abrir. Habrá que fijarse en la URL. En los casos de phishing, la URL no coincide con la de la organización que están intentando suplantar, aunque la apariencia de la web sea similar.
        \item Si el correo incluye ficheros adjuntos, se deben verificar antes de abrir.
        \item No se responderá nunca a un email sospechoso. Además, no se reenviará el correo a personas de nuestro entorno, viniendo de nosotros podrían confiar en él y caer en la trampa.
    \end{itemize}
\end{itemize}

\item \textbf{Seguridad en los dispositivos}

\Beneficiario{} facilita a sus usuarios el equipamiento informático necesario para la realización de las tareas relacionadas con su puesto de trabajo.

\textbf{Propiedad de los recursos}. Este equipamiento será propiedad de la Organización y por tanto no estará destinado a un uso personal. Como consecuencia de esto, \Beneficiario{} se reservará el derecho de revisar, sin previo aviso, los equipos, el uso de Internet y el teléfono corporativo que esté haciendo cada usuario y, en caso de que existieran indicios de que se está llevando a cabo una utilización indebida. De esta forma el usuario quedará informado de que el resultado de los controles efectuados puede ser utilizado para llevar a cabo, en su caso, las actuaciones disciplinarias previstas por la normativa vigente.

\textbf{Obligaciones de los usuarios}. Los Usuarios deberán cumplir las siguientes medidas de seguridad para el uso de los ordenadores personales:

\begin{itemize}
    \item Conexión de otros dispositivos
    \begin{itemize}
        \item No estará permitido conectar dispositivos que no estén autorizados a la red de la Organización.
        \item Tampoco se podrán conectar a los dispositivos autorizados, otros dispositivos que no estén autorizados expresamente.
    \end{itemize}
    \item Ubicación del dispositivo
    No estará permitido variar la ubicación física de los dispositivos asignados a una ubicación.
    \item Configuración del dispositivo
    No estará permitido alterar la configuración física, configuración de seguridad ni el software de los dispositivos.
    \item Uso de dispositivos y de la red
    Los dispositivos, así como la red de información que \Beneficiario{} ponga a disposición de los usuarios estarán destinados a permitir el desempeño de las funciones y tareas profesionales que estos tienen encomendadas, estando prohibido el uso para finalidades de carácter personal, o bien para realizar actos desleales o que pudieran ser considerados ilícitos.
    \item Antivirus
    El Usuario deberá comprobar que su antivirus se actualiza con regularidad. En caso contrario deberá notificarlo como una incidencia de seguridad.
    \item Uso de la información
    \begin{itemize}
        \item Estará prohibido utilizar, copiar o transmitir información contenida en los sistemas informáticos para uso privado.
        \item El Usuario se abstendrá de copiar la información contenida en los ficheros en los que se almacenen datos de carácter personal u otro tipo de información de este Organismo en ordenador propio, pendrives o a cualquier otro soporte informático, salvo que solicite autorización al Responsable de Seguridad, y se adopten las medidas de seguridad correspondiente. Asimismo, los datos contenidos en este tipo de soportes deberán ser suprimidos una vez hayan dejado de ser útiles y pertinentes para la satisfacción de los fines que motivaron su creación. Durante el periodo de tiempo que los ficheros o archivos permanezcan en el equipo o soporte informático externo, deberá restringir el acceso y uso de la información que obra en los mismos.
        \item El Usuario deberá restringir a terceros (familiares, amistades o cualesquiera otros) el acceso a los archivos o ficheros titularidad de la Organización y dispuesto a razón única de las funciones o tareas desempeñadas en la misma. Se establecerán medidas de protección adicionales que aseguren la confidencialidad de la información almacenada en el equipo se almacenen datos de carácter personal.
        \item Se prohibirá todo uso de programas de compartición de contenidos, habitualmente utilizados para la descarga de archivos de música, vídeo, etc. que no estén permitidos por la empresa.
    \end{itemize}
    \item Identificación y autenticación
    Las contraseñas de acceso al equipo, sistema y/o a la red, concedidos por \Beneficiario{} serán personales e intransferibles, y será el Usuario el único responsable de las consecuencias que pudieran derivarse de su mal uso, divulgación o pérdida.
    Por cuestiones de seguridad no estarán permitidas prácticas como:
    \begin{itemize}
        \item Emplear identificadores y contraseñas de otros Usuarios para acceder al sistema y a la red de la Organización.
        \item Intentar modificar o acceder al registro de accesos.
        \item Burlar las medidas de seguridad establecidas en el sistema informático, intentando acceder a ficheros.
    \end{itemize}
    \item Incidencias con los dispositivos o accesos
    Cuando se considere que el acceso se ha visto comprometido se deberá comunicar al responsable correspondiente.
    \item Bloqueo del puesto de trabajo
    Al abandonar el puesto de trabajo deberán cerrarse las sesiones con las aplicaciones establecidas, se habilitará el protector de pantalla con bloqueo con contraseña, y se apagarán los equipos al finalizar la jornada laboral, excepto en los casos en que el equipo deba permanecer encendido.
\end{itemize}

\item \textbf{Mantenimiento de los endpoints}

\begin{itemize}
\item Se utilizarán los equipos siempre de acuerdo con las especificaciones indicadas por el área de informática o, en su caso, del fabricante.
\item Se deberán mantener los equipos en buen estado de conservación.
\item Se evitará el uso en condiciones de temperatura o humedad inadecuadas, o en entornos que lo desaconsejen (mesas con alimentos y líquidos, entornos sucios, etc.).
\item Se transportarán de manera segura los equipos, evitando proporcionar información sobre el contenido de estos.
\item Se deberán realizar las actualizaciones en los equipos siguiendo las instrucciones que se reciban.
\end{itemize}

\item \textbf{Prevención antimalware}

\begin{itemize}
\item En caso de disponer de autorización del Responsable de Sistemas para instalar software, se realizará desde direcciones seguras.
\item Se deberá mantener el antivirus actualizado y disponer de una correcta configuración del firewall.
\item No se descargarán ficheros adjuntos o imágenes cuya fuente no haya sido comprobada o no sea fiable, ya que pueden contener malware.
\item Si se sospecha que un dispositivo ha sido infectado por un virus u otro software malicioso, se dejará de procesar la información, y se comunicará con urgencia a Departamento de Sistemas.
\item No se activarán opciones de trabajo de sesión abierta en un sitio web.
\item Se eliminará la posibilidad de reutilización de la sesión cerrada.
\item No se eliminará la configuración definida de bloqueo automático de sesión, ni de bloqueo de pantalla.
\end{itemize}

\end{enumerate}