% !TEX root = ../Guia SGSI.tex

\subsubsection{Política de seguridad de la información}

\begin{enumerate}[label=\alph*)]
\item Objetivos de la Política de Seguridad de la Información

El objetivo principal de la creación de esta Política de Seguridad de la Información, por parte del Responsable de Seguridad del Sistema de Gestión de la Seguridad de la Información (SGSI) y de la Dirección de \textbf{\Beneficiario}, será el de garantizar, a los clientes y usuarios de los servicios, el acceso a la información con la calidad y el nivel de servicio que se requieren para el desempeño acordado, así como el de evitar serias pérdidas o alteración de la información y accesos no autorizados a la misma.

Se establecerá un marco para la consecución de los objetivos de seguridad de la información para la Organización. Dichos objetivos se alcanzarán a través de una serie de medidas organizativas y de normas concretas y claramente definidas.

Esta Política de Seguridad será mantenida, actualizada y adecuada a los fines de la Organización. Además, se compartirá con todos los trabajadores de \textbf{\Beneficiario}.

Los principios que tendrán que respetarse, en base a las dimensiones básicas de la seguridad, serán los siguientes:

\begin{itemize}
  \item \textbf{Confidencialidad:} propiedad por la cual únicamente puede acceder a la información gestionada por la Organización quién esté autorizado para ello, previa identificación, en el momento y por los medios habilitados.

  \item \textbf{Integridad:} propiedad que garantiza la validez, exactitud y completitud de la información gestionada por la Organización, siendo su contenido el facilitado por los afectados sin ningún tipo de manipulación y permitiendo que sea modificada únicamente por quién esté autorizado para ello.

  \item \textbf{Disponibilidad:} propiedad de ser accesible y utilizable en los intervalos acordados. La información gestionada por la Organización es accesible y utilizable por los clientes y usuarios autorizados e identificados en todo momento, quedando garantizada su propia persistencia ante cualquier eventualidad prevista.
\end{itemize}

Adicionalmente, dado que cualquier Sistema de Gestión de la Seguridad de la Información debe cumplir con la legislación vigente, se atenderá al siguiente principio:

\begin{itemize}
  \item \textbf{Legalidad:} referido al cumplimiento de las leyes, normas, reglamentaciones o disposiciones a las que está sujeta la Organización, especialmente en materia de protección de datos personales.
\end{itemize}

\item \textbf{Alcance}

Esta Política se aplicará a los sistemas de información de \Beneficiario, relacionados con el ejercicio de sus competencias y a todos los usuarios con acceso autorizado a los mismos. Todos ellos tienen la obligación de conocer y cumplir esta Política de Seguridad de la Información y su Normativa de Seguridad derivada, siendo responsabilidad del Comité de Seguridad disponer los medios necesarios para que la información llegue al personal afectado.

\item \textbf{Gestión del riesgo}

La gestión de la Seguridad de la Información en \Beneficiario será basada en el riesgo, de conformidad con la Norma internacional ISO/IEC 27001.

Se articulará mediante un proceso general de apreciación y tratamiento del riesgo, que potencialmente puede afectar a la seguridad de la información de los servicios prestados, consistente en:

\begin{itemize}
    \item Identificar las amenazas, que aprovecharán vulnerabilidades de los Sistemas de Información que soportan, o de los que depende, la seguridad de la información.
    \item Analizar el riesgo, en base a la consecuencia de materializarse la amenaza y de la probabilidad de ocurrencia.
    \item Evaluar el riesgo, según un nivel previamente establecido y aprobado de riesgo ampliamente aceptable, tolerable e inaceptable.
    \item Tratar el riesgo inaceptable, mediante los controles o salvaguardas adecuadas.
\end{itemize}

Dicho proceso será cíclico y se llevará a cabo de forma periódica, como mínimo una vez al año. Para cada riesgo identificado se asignará un propietario, pudiendo recaer múltiples responsabilidades en una misma persona o comité.

\item \textbf{Marco normativo}

El marco legal en materia de seguridad de la información viene establecido por la siguiente legislación:

\begin{itemize}
    \item El Real Decreto 311/2022, de 3 de mayo, por el que se regula el Esquema Nacional de Seguridad.
    \item La Ley Orgánica 3/2018, de 5 de diciembre, de Protección de Datos Personales y garantía de los derechos digitales, que tiene por objeto adaptar el ordenamiento jurídico español al Reglamento (UE) 2016/679 del Parlamento Europeo y el Consejo, de 27 de abril de 2016, relativo a la protección de las personas físicas en lo que respecta al tratamiento de sus datos personales y a la libre circulación de estos datos, y completar sus disposiciones.
    \item Reglamento (UE) 2016/679 del Parlamento Europeo y del Consejo, de 27 de abril de 2016, relativo a la protección de las personas físicas en lo que respecta al tratamiento de datos personales y a la libre circulación de estos datos y por el que se deroga la Directiva 95/46/CE (RGPD).
    \item Ley 6/2020, de 11 de noviembre, reguladora de determinados aspectos de los servicios electrónicos de confianza, cuyo objeto es la regulación de determinados aspectos de los servicios electrónicos de confianza, como complemento del Reglamento (UE) n.º 910/2014 del Parlamento Europeo y del Consejo, de 23 de julio de 2014, relativo a la identificación electrónica y los servicios de confianza para las transacciones electrónicas en el mercado interior (Reglamento eIDAS).
\end{itemize}

\item \textbf{Objetivos del SGSI}

El SGSI de \Beneficiario deberá garantizar:

\begin{itemize}
    \item Que se desarrollen políticas, normativas, procedimientos y guías operativas para apoyar la política de seguridad de la información.
    \item Que se identifique la información que deba ser protegida.
    \item Que se establezca y mantenga la gestión del riesgo alineada con los requerimientos de la política del SGSI y la estrategia de la Organización.
    \item Que se establezca una metodología para la apreciación y tratamiento del riesgo.
    \item Que se establezcan criterios con los que medir el nivel de cumplimiento del SGSI.
    \item Que se revise el nivel de cumplimiento del SGSI.
    \item Que se corrijan las no conformidades mediante la implementación de acciones correctivas.
    \item Que el personal reciba formación y concienciación sobre la seguridad de la información.
    \item Que todo el personal sea informado sobre la obligación de cumplimiento de la política de seguridad de la información.
    \item La asignación de los recursos necesarios para gestionar el SGSI.
    \item La identificación y cumplimiento de todos los requerimientos legales, regulatorios y contractuales.
    \item Que se identifiquen y analicen las implicaciones de seguridad de la información respecto a los requerimientos de negocio.
    \item Que se mida el grado de madurez del propio sistema de gestión de la seguridad de la información.
    \item Que se realice la mejora continua sobre el SGSI.
\end{itemize}

\item \textbf{Criterios de la seguridad de la información}

\Beneficiario{} establece las siguientes acciones para organizar la Seguridad de la Información:

\begin{itemize}
  \item Designará roles de seguridad: responsables unificados de Servicios y de la Información, Responsable de la Seguridad, Responsable del Sistema y Delegado de Protección de Datos.
  \item Constituirá un órgano consultivo y estratégico para la toma de decisiones en materia de Seguridad de la Información. Este órgano se denominará Comité de Seguridad de la Información.
\end{itemize}

\item \textbf{Definición de Roles y Responsabilidades}

La estructura organizativa de seguridad, y jerarquía en el proceso de decisiones la componen:

\begin{table}[H]
  \centering
  \small
  \begin{tabular}{|p{4cm}|p{11cm}|}
  \hline
  \celdaTitulo{Rol} & \celdaTitulo{Funciones} \\
  \hline
  Dirección & Órganos colegiados o unipersonales que deciden la misión y los objetivos de la Organización. \\
  \hline
  Comité de Seguridad y Privacidad & Órganos colegiados o unipersonales que toman decisiones que concretan cómo alcanzar los objetivos de seguridad y protección de la privacidad marcados por los órganos de gobierno. \\
  \hline
  Responsable de la Información & A nivel de gobierno. Tiene la responsabilidad última sobre qué seguridad requiere una cierta información manejada por la Organización. \\
  \hline
  Responsable de Servicio & A nivel de gobierno o, en ocasiones baja a nivel ejecutivo. Tiene la responsabilidad última de determinar los niveles de servicio aceptables por la Organización. \\
  \hline
  Responsable de Seguridad & A nivel ejecutivo. Funciona como supervisor de la operación del sistema y vehículo de reporte al Comité de Seguridad de la Información. \\
  \hline
  Responsable del Sistema & A nivel operacional. Toma decisiones operativas: arquitectura del sistema, adquisiciones, instalaciones y operación del día a día. \\
  \hline
  Administradores de seguridad & Son las personas encargadas de ejecutar las acciones diarias de operación del sistema según las indicaciones recibidas de sus superiores jerárquicos. \\
  \hline
  Delegado de Protección de Datos & Figura obligatoria para administraciones públicas, es el encargado de asesorar y supervisar todos los aspectos relacionados con el tratamiento de datos de carácter personal, incluidos los aspectos de seguridad (integridad, confidencialidad y disponibilidad) y violación de datos personales. Su nombramiento se produce por otra vía ya que sus cometidos no se ciñen únicamente a aspectos de seguridad. \\
  \hline
  \end{tabular}
  \caption{Estructura organizativa de seguridad}
  \label{tab:guia-sgsi-estructura-organizativa}
  \normalsize
\end{table}

\vspace{1em}

En el caso de PYMES, donde los equipos especializados como los Responsables de Información, Seguridad de la Información, Sistema y el Delegado de Protección de Datos (DPD) pueden no estar definidos de manera individual, estas responsabilidades se agrupan de forma más compacta. La dirección de la empresa, el responsable de TI y el DPD (en caso de que exista), se encargan de asumir estas funciones de manera coordinada.

\textbf{Responsables de Información y de los Servicios}

Serán funciones de los Responsables de la Información y de los Servicios:

\begin{itemize}
  \item Establecer los requisitos de seguridad aplicables a la Información (niveles de seguridad de la Información) y a los Servicios (niveles de seguridad de los servicios) pudiendo recabar una propuesta al Responsable de Seguridad y teniendo en cuenta la opinión del Responsable del Sistema.
  \item Dictaminar respecto a los derechos de acceso a la información y los servicios.
  \item Aceptar los niveles de riesgo residual que afectan a la información y los servicios.
  \item Poner en comunicación del Responsable de Seguridad cualquier variación respecto a la Información y los Servicios de los que es responsable, especialmente la incorporación de nuevos Servicios o Información a su cargo. El cual dará traslado de dichos cambios, al Comité de Seguridad de la Información, en su próxima reunión.
  \item Tiene la responsabilidad última del uso que se haga de determinados servicios e información y, por tanto, de su protección.
\end{itemize}

\textbf{Responsable de la Seguridad de la información}

Serán funciones del Responsable de Seguridad:

\begin{itemize}
  \item Mantener y verificar el nivel adecuado de seguridad de la Información manejada y de los Servicios electrónicos prestados por los sistemas de información.
  \item Promover la formación y concienciación en materia de seguridad de la información.
  \item Designar responsables de la ejecución del análisis de riesgos, de la Declaración de Aplicabilidad, identificar medidas de seguridad, determinar configuraciones necesarias y elaborar documentación del sistema.
  \item Aprobar la Declaración de Aplicabilidad a partir de las medidas de seguridad requeridas conforme al Anexo II del ENS y la ISO 27001.
  \item Proporcionar asesoramiento para la determinación de la Categoría del Sistema, en colaboración con el Responsable del Sistema y/o Comité de Seguridad.
  \item Participar en la elaboración e implantación de los planes de mejora de la seguridad y, llegado el caso, en los planes de continuidad, procediendo a su validación.
  \item Gestionar las revisiones externas o internas del sistema, así como los procesos de certificación.
  \item Elevar al Comité de Seguridad la aprobación de cambios y otros requisitos del sistema.
  \item Aprobar los procedimientos de seguridad que forman parte del Mapa Normativo (y no son competencia del Comité) y poner en conocimiento al Comité de las modificaciones que se hayan realizado a lo largo del periodo en curso.
  \item Participará en la elaboración, en el marco del Comité de Seguridad de la Información, de la Política de Seguridad de la Información, para su aprobación por Dirección.
  \item Actuará como secretario del Comité de Seguridad de la Información, realizando las siguientes funciones:
  \begin{itemize}
    \item Convocar las reuniones del Comité de Seguridad de la Información.
    \item Preparar los temas a tratar en las reuniones del Comité, aportando información puntual para la toma de decisiones.
    \item Elaborar el acta de las reuniones.
    \item Es responsable de la ejecución directa o delegada de las decisiones del Comité.
  \end{itemize}
\end{itemize}

\textbf{Responsable del Sistema}

Serán funciones del Responsable del Sistema:

\begin{itemize}
  \item Desarrollar, operar y mantener el sistema de información durante todo su ciclo de vida, elaborando los procedimientos operativos necesarios.
  \item Definir la topología y la gestión del Sistema de Información estableciendo los criterios de uso y los servicios disponibles en el mismo.
  \item Detener el acceso a información o prestación de servicio si tiene el conocimiento de que estos presentan deficiencias graves de seguridad.
  \item Cerciorarse de que las medidas específicas de seguridad se integren adecuadamente dentro del marco general de seguridad.
  \item Proporcionar asesoramiento para la determinación de la Categoría del Sistema, en colaboración con el Responsable de Seguridad y/o Comité de Seguridad de la Información.
  \item Participar en la elaboración e implantación de los planes de mejora de la seguridad y llegado el caso en los planes de continuidad.
  \item Coordinar las funciones del administrador de la seguridad del sistema:
  \begin{itemize}
    \item La gestión, configuración y actualización, en su caso, del hardware y software.
    \item La gestión de las autorizaciones concedidas a los usuarios del sistema, en particular los privilegios concedidos.
    \item Aprobar los cambios en la configuración vigente del Sistema de Información.
    \item Asegurar que los controles de seguridad establecidos son cumplidos estrictamente.
    \item Asegurar que son aplicados los procedimientos aprobados para manejar el Sistema de Información.
    \item Supervisar las instalaciones de hardware y software, sus modificaciones y mejoras.
    \item Monitorizar el estado de seguridad.
    \item Informar al Responsable de Seguridad de cualquier anomalía, compromiso o vulnerabilidad relacionada con la seguridad.
    \item Colaborar en la investigación y resolución de incidentes de seguridad, desde su detección hasta su resolución.
  \end{itemize}
\end{itemize}

\textbf{Delegado de Protección de Datos (DPD)}

Serán funciones del Delegado de Protección de Datos:

\begin{itemize}
    \item Informar y asesorar a la organización, y a los usuarios que se ocupen del tratamiento, de las obligaciones que les incumben en virtud de la normativa vigente en materia de Protección de Datos.
    \item Supervisar el cumplimiento de lo dispuesto en normativa de seguridad y de las políticas internas de la organización, en materia de protección de datos, incluida la asignación de responsabilidades, la concienciación y formación del personal que participa en las operaciones de tratamiento, y las auditorías correspondientes.
    \item Ofrecer el asesoramiento que se le solicite acerca de la evaluación de impacto relativa a la protección de datos y supervisará su aplicación.
    \item Cooperar con la Agencia Española de Protección de Datos cuando esta lo requiera, actuando como punto de contacto con esta para cuestiones relativas al tratamiento de datos.
    \item El Delegado de Protección de Datos desempeñará sus funciones prestando atención a los riesgos asociados a las operaciones de tratamiento. Para ello, debe ser capaz de:
    \item Recabar información para determinar las actividades de tratamiento.
    \item Analizar y comprobar la conformidad de las actividades de tratamiento.
    \item Informar, asesorar y emitir recomendaciones al responsable o el encargado del tratamiento.
    \item Recabar información para supervisar el registro de las operaciones de tratamiento.
    \item Asesorar en el principio de la protección de datos por diseño y por defecto.
    \item Asesorar sobre si se lleva a cabo o no las evaluaciones de impacto, metodología, salvaguardas a aplicar, etc.
    \item Priorizar actividades en base a los riesgos.
    \item Asesorar al Responsable de Tratamiento sobre áreas a someter a auditorías, actividades de formación a realizar y operaciones de tratamiento a dedicar más tiempo y recursos.
\end{itemize}

\item \textbf{Comité de Seguridad de la Información}

\Beneficiario{} debe crear un Comité de Seguridad para coordinar la protección de datos, gestionar riesgos y garantizar el cumplimiento normativo, fortaleciendo así la cultura de seguridad en toda la organización.

Este comité deberá estar compuesto por los siguientes miembros que se clasificarán en permanentes o no permanentes atendiendo a la obligatoriedad de la participación del Comité de Seguridad de la Información:

\textbf{Miembros permanentes:}
\begin{itemize}
  \item Dirección
  \item Responsable de Sistema (RSIS)
  \item Responsable de Seguridad de la Información (RSEG)
  \item Asesores que se consideren oportunos para los temas en cuestión con voz, pero sin voto.
\end{itemize}

\textbf{Miembros no permanentes:}
\begin{itemize}
  \item Responsables del Servicio y de la Información.
  \item El Delegado de Protección de Datos.
  \item Representantes de la organización, especialistas externos del sector público o privado cuya presencia, por razón de su experiencia o vinculación con los asuntos tratados, sea necesaria o aconsejable.
  \item Los Responsables de la Información y los Servicios serán convocados por la presidencia en función de los asuntos a tratar, en representación de los distintos ámbitos o áreas de seguridad TIC. Cada área estará representada por un vocal con voto, sin perjuicio de que acudan varios representantes de la misma.
  \item El Delegado de Protección de Datos participará con voz, pero sin voto, en las reuniones del Comité de Seguridad de la Información cuando se aborden cuestiones relacionadas con el tratamiento de datos de carácter personal, o cuando se requiera su participación. En todo caso, si un asunto se sometiese a votación, se hará constar en acta la opinión del Delegado de Protección de Datos.
  \item El Secretario/a del Comité realizará las convocatorias y levantará actas de las reuniones del Comité de Seguridad. A las sesiones podrán asistir asesores invitados por el Presidente.
\end{itemize}

En una PYME donde no es posible disponer de todos los cargos necesarios para el Comité, \Beneficiario{} podrá designar miembros con múltiples roles o incorporar asesores externos. Aunque reducido, este comité sigue siendo esencial para coordinar la seguridad, gestionar riesgos y fomentar una cultura de protección.

\textbf{Funciones del Comité de Seguridad:}
\begin{itemize}
  \item Atender las inquietudes de la Alta Dirección y de los diferentes departamentos.
  \item Informar regularmente del estado de la seguridad de la información a la Alta Dirección.
  \item Promover la mejora continua del SGSI.
  \item Elaborar la estrategia de evolución de la seguridad de la información.
  \item Promover la realización de auditorías periódicas.
  \item Aprobar documentación de seguridad.
  \item Estar informado sobre la normativa y entidades relacionadas con la certificación del ENS.
  \item Proponer directrices y recomendaciones, que se reflejarán en actas.
  \item Coordinar esfuerzos de diferentes áreas en seguridad.
  \item Resolver conflictos de responsabilidad o escalarlos si fuera necesario.
  \item Asesorar en materia de seguridad cuando se le requiera.
  \item Revisar la Política de Seguridad antes de su aprobación.
  \item Aprobar el Plan de Adecuación para la implantación del ENS.
\end{itemize}

\textbf{Periodicidad de reuniones y adopción de acuerdos:}
\begin{itemize}
  \item El Comité se reunirá al menos una vez al año, pudiendo convocarse más a menudo si es necesario.
  \item Las reuniones se convocarán por su Presidencia, por iniciativa propia o a petición de la mayoría de sus miembros permanentes.
  \item Las decisiones se tomarán por consenso de los miembros permanentes.
\end{itemize}

\textbf{Designación y resolución de conflictos:}
\begin{itemize}
  \item La creación y constitución del Comité, así como la designación de sus miembros, se formalizará mediante acta inicial.
  \item Los roles nombrados se renovarán automáticamente cada año. Cualquier cambio se comunicará siguiendo el procedimiento establecido.
  \item Según el artículo 13.3 del RD del ENS, no puede existir dependencia jerárquica entre el RSEG y el RSIS, salvo excepciones justificadas. En tal caso, se establecerán medidas compensatorias.
\end{itemize}

\item \textbf{Datos personales y riesgos derivados del tratamiento}

El legislador, en el artículo 12.1.f) del RD del ENS, hace referencia al cumplimiento adecuado en materia de protección de datos personales. Para dar cumplimiento a la LOPD-GDD y al RGPD, \Beneficiario{}:

\begin{itemize}
  \item Publicará el registro de actividades de tratamiento.
  \item Realizará análisis de riesgos y evaluaciones de impacto (EIPD) cuando sea necesario.
  \item Aplicará las fases indicadas en los Anexos I y II del RD del ENS para los sistemas afectados.
\end{itemize}

\item \textbf{Obligaciones del personal}

Todo el personal, tanto interno como externo, que interactúe con el sistema de información, deberá cumplir con la presente Política de Seguridad de la Información.

\item \textbf{Formación y concienciación}

El Responsable de Seguridad del SGSI deberá garantizar que todo el personal involucrado:

\begin{itemize}
  \item Conoce la política, objetivos y procesos del SGSI.
  \item Recibe acciones formativas y de concienciación.
  \item Tiene acceso a la documentación correspondiente según su rol.
\end{itemize}

\item \textbf{Terceras partes}

Cuando \Beneficiario{} preste servicios o maneje información de terceros:

\begin{itemize}
  \item Se les hará partícipes de esta Política de Seguridad.
  \item Se establecerán canales de reporte y coordinación con comités correspondientes.
  \item Deberán aplicar esta normativa, pudiendo desarrollar procedimientos propios.
  \item Su personal recibirá formación equivalente a la de esta política.
  \item Si no pueden cumplir algún punto, el Responsable de Seguridad emitirá informe de riesgos que deberá ser aprobado por los responsables implicados.
  \item Se establecerá un Punto Operacional de Comunicación (POC) para el enlace.
\end{itemize}

\item \textbf{Auditoría}

La Dirección General de \Beneficiario{} garantizará, mediante auditorías internas y/o externas, el cumplimiento de esta política, aplicando medidas correctoras necesarias para la mejora continua.

\item \textbf{Validez y actualización}

Esta política será efectiva desde su publicación y se revisará anualmente. En la revisión se considerarán:

\begin{itemize}
  \item Cambios en el contexto interno o externo de la organización.
  \item Incidentes y No Conformidades.
  \item Resultados de los procesos de apreciación de riesgo.
  \item Actualización de normas y documentos relacionados.
\end{itemize}

Se elaborará una lista de objetivos y acciones a emprender durante el año siguiente.

\item \textbf{Sanciones}

El incumplimiento de esta política y sus procedimientos podrá dar lugar a sanciones conforme a la legislación laboral vigente, según la magnitud del incumplimiento.

\item \textbf{Ratificación}

Todos los abajo firmantes asumen y aceptan plenamente el contenido de esta Política y se comprometen a aplicarla en sus respectivas áreas para lograr el correcto funcionamiento del SGSI.

\end{enumerate}