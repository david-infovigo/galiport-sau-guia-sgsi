% !TEX root = Guia SGSI.tex

\documentclass[a4paper,12pt]{article}

\usepackage[utf8]{inputenc}
\usepackage[spanish]{babel}
\usepackage{graphicx}
\usepackage{fancyhdr}
\usepackage{geometry}
\usepackage{enumitem}
\usepackage{parskip}
\usepackage{times}
\usepackage{setspace}
\usepackage{titlesec}
\usepackage{xcolor}
\usepackage{tabularx}
\usepackage{float}
\usepackage[table]{xcolor}
\usepackage{colortbl}
\usepackage{tcolorbox}

% Datos del documento
\title{GUÍA DE IMPLEMENTACIÓN DEL SISTEMA DE GESTIÓN DE LA SEGURIDAD DE LA INFORMACIÓN}
\author{}
\date{}

% Ajuste de márgenes
\geometry{top=2.5cm,bottom=4cm,left=2.5cm,right=2.5cm}

% Ajustes de espacio para cabecera y pie
\setlength{\headheight}{1.75cm} % Ajustar altura de la cabecera
%\setlength{\footskip}{1.75cm} % Ajustar espacio para el pie de página

% Cabecera y pie con imágenes
\pagestyle{fancy}
\fancyhf{}
\fancyhead[L]{\includegraphics[width=\textwidth]{img/cabecera.png}}
\fancyfoot[C]{\thepage}
\renewcommand{\headrulewidth}{0pt}
\renewcommand{\footrulewidth}{0pt}

% Colores
\definecolor{customblue}{HTML}{002060}

% Titulo
\makeatletter
\renewcommand{\maketitle}{
  \begin{center}
    {\textbf{\normalsize \@title} \par}
  \end{center}
  \vspace{0em}
}
\makeatother

% Titulos
\titleformat{\section}
  {\normalfont\normalsize\bfseries\color{customblue}}
  {SECCIÓN \thesection:}
  {0.5em}
  {\MakeUppercase}
\titleformat{\subsection}
  {\normalfont\normalsize\bfseries\color{customblue}}
  {\thesubsection}
  {0.5em}
  {}

% Espaciado entre líneas
\renewcommand{\arraystretch}{1.5}

% Resetear el contador de secciones
\setcounter{section}{1}

% Tablas
\newcommand{\titulofila}[1]{%
  \hline \rowcolor[gray]{0.9} \multicolumn{2}{|c|}{\textbf{#1}} \\[0.5em] \hline
}
\newcommand{\celdaTitulo}[1]{\cellcolor[gray]{0.9}\makebox[\linewidth]{\textbf{#1}}}
\newcommand{\tablacentrada}[2]{%
  \noindent
  \begin{minipage}{0.90\textwidth}
    \renewcommand{\arraystretch}{1.3}
    \small
    \begin{tabularx}{\linewidth}{|>{\centering\arraybackslash}m{#1}|X|}
      #2
    \end{tabularx}
    \normalsize
  \end{minipage}
}

% Plantilla documento
\newcommand{\documentoMarco}[1]{%
  \begin{tcolorbox}[
    colframe=black,
    colback=white,
    boxrule=0.8pt,
    width=\textwidth,
    sharp corners,
    arc=0pt,
  ]
    \centering
    \includegraphics[width=0.25\textwidth]{img/logotipo.png}

    \vspace{1em}
    \raggedright
    #1
  \end{tcolorbox}
}
