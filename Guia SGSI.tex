% !TEX root = Guia SGSI.tex

% Variables
\newcommand{\Lugar}{SANTIAGO DE COMPOSTELA}
\newcommand{\Fecha}{10 de abril de 2025}
\newcommand{\Empresa}{GALIPORT, S.A.U.}
\newcommand{\CIF}{A15067663}
\newcommand{\Direccion}{VIA PTOLOMEO 14 POL IND DE TAMBRE, 15890 – SANTIAGO DE COMPOSTELA (A CORUÑA)}
\newcommand{\NombreAsesorDigital}{AACCENTIA MULTIMEDIA, S.L.}
\newcommand{\NifAsesorDigital}{B36870822}
\newcommand{\NombrePyme}{GALIPORT, S.A.U.}
\newcommand{\NifPyme}{A15067663}
\newcommand{\ActividadPyme}{(Indicar la actividad económica que realiza la pyme beneficiaria)}
\newcommand{\ServicioAsesoramiento}{Servicio de Asesoramiento en Ciberseguridad Básico}
\newcommand{\CosteServicio}{6000 €}
\newcommand{\FocoServicio}{Plan pensado para PYMEs que no disponen de una protección básica o que, aun teniéndola, no cuentan con un plan de ciberseguridad adaptado a la actividad económica desarrollada, para implantarlo, y para definir y aplicar la documentación básica de su Sistema de Gestión de Seguridad de la Información (SGSI) según ISO 27001 y ENS (categorías media-alta), y realización de un caso de uso adaptado al negocio en el área de la ciberseguridad.}
\newcommand{\Beneficiario}{GALIPORT, S.A.U.}

% !TEX root = Guia SGSI.tex

\documentclass[a4paper,12pt]{article}

\usepackage[utf8]{inputenc}
\usepackage[spanish]{babel}
\usepackage{graphicx}
\usepackage{fancyhdr}
\usepackage{geometry}
\usepackage{enumitem}
\usepackage{parskip}
\usepackage{times}
\usepackage{setspace}
\usepackage{titlesec}
\usepackage{xcolor}
\usepackage{tabularx}
\usepackage{float}
\usepackage[table]{xcolor}
\usepackage{colortbl}
\usepackage{tcolorbox}

% Datos del documento
\title{GUÍA DE IMPLEMENTACIÓN DEL SISTEMA DE GESTIÓN DE LA SEGURIDAD DE LA INFORMACIÓN}
\author{}
\date{}

% Ajuste de márgenes
\geometry{top=2.5cm,bottom=4cm,left=2.5cm,right=2.5cm}

% Ajustes de espacio para cabecera y pie
\setlength{\headheight}{1.75cm} % Ajustar altura de la cabecera
%\setlength{\footskip}{1.75cm} % Ajustar espacio para el pie de página

% Cabecera y pie con imágenes
\pagestyle{fancy}
\fancyhf{}
\fancyhead[L]{\includegraphics[width=\textwidth]{img/cabecera.png}}
\fancyfoot[C]{\thepage}
\renewcommand{\headrulewidth}{0pt}
\renewcommand{\footrulewidth}{0pt}

% Colores
\definecolor{customblue}{HTML}{002060}

% Titulo
\makeatletter
\renewcommand{\maketitle}{
  \begin{center}
    {\textbf{\normalsize \@title} \par}
  \end{center}
  \vspace{0em}
}
\makeatother

% Titulos
\titleformat{\section}
  {\normalfont\normalsize\bfseries\color{customblue}}
  {SECCIÓN \thesection:}
  {0.5em}
  {\MakeUppercase}
\titleformat{\subsection}
  {\normalfont\normalsize\bfseries\color{customblue}}
  {\thesubsection}
  {0.5em}
  {}

% Espaciado entre líneas
\renewcommand{\arraystretch}{1.5}

% Resetear el contador de secciones
\setcounter{section}{1}

% Tablas
\newcommand{\titulofila}[1]{%
  \hline \rowcolor[gray]{0.9} \multicolumn{2}{|c|}{\textbf{#1}} \\[0.5em] \hline
}
\newcommand{\celdaTitulo}[1]{\cellcolor[gray]{0.9}\makebox[\linewidth]{\textbf{#1}}}
\newcommand{\tablacentrada}[2]{%
  \noindent
  \begin{minipage}{0.90\textwidth}
    \renewcommand{\arraystretch}{1.3}
    \small
    \begin{tabularx}{\linewidth}{|>{\centering\arraybackslash}m{#1}|X|}
      #2
    \end{tabularx}
    \normalsize
  \end{minipage}
}

% Plantilla documento
\newcommand{\documentoMarco}[1]{%
  \begin{tcolorbox}[
    colframe=black,
    colback=white,
    boxrule=0.8pt,
    width=\textwidth,
    sharp corners,
    arc=0pt,
  ]
    \centering
    \includegraphics[width=0.25\textwidth]{img/logotipo.png}

    \vspace{1em}
    \raggedright
    #1
  \end{tcolorbox}
}


\begin{document}

\maketitle
\addcontentsline{toc}{section}{Guía de implementación del Sistema de Gestión de la Seguridad de la Información}

\section{Resultados}

\setcounter{subsection}{1}

\subsection{Plan de protección de negocio}

% !TEX root = ../Guia SGSI.tex

\subsubsection{Política de seguridad de la información}

\begin{enumerate}[label=\alph*)]
\item Objetivos de la Política de Seguridad de la Información

El objetivo principal de la creación de esta Política de Seguridad de la Información, por parte del Responsable de Seguridad del Sistema de Gestión de la Seguridad de la Información (SGSI) y de la Dirección de \textbf{\Beneficiario}, será el de garantizar, a los clientes y usuarios de los servicios, el acceso a la información con la calidad y el nivel de servicio que se requieren para el desempeño acordado, así como el de evitar serias pérdidas o alteración de la información y accesos no autorizados a la misma.

Se establecerá un marco para la consecución de los objetivos de seguridad de la información para la Organización. Dichos objetivos se alcanzarán a través de una serie de medidas organizativas y de normas concretas y claramente definidas.

Esta Política de Seguridad será mantenida, actualizada y adecuada a los fines de la Organización. Además, se compartirá con todos los trabajadores de \textbf{\Beneficiario}.

Los principios que tendrán que respetarse, en base a las dimensiones básicas de la seguridad, serán los siguientes:

\begin{itemize}
  \item \textbf{Confidencialidad:} propiedad por la cual únicamente puede acceder a la información gestionada por la Organización quién esté autorizado para ello, previa identificación, en el momento y por los medios habilitados.

  \item \textbf{Integridad:} propiedad que garantiza la validez, exactitud y completitud de la información gestionada por la Organización, siendo su contenido el facilitado por los afectados sin ningún tipo de manipulación y permitiendo que sea modificada únicamente por quién esté autorizado para ello.

  \item \textbf{Disponibilidad:} propiedad de ser accesible y utilizable en los intervalos acordados. La información gestionada por la Organización es accesible y utilizable por los clientes y usuarios autorizados e identificados en todo momento, quedando garantizada su propia persistencia ante cualquier eventualidad prevista.
\end{itemize}

Adicionalmente, dado que cualquier Sistema de Gestión de la Seguridad de la Información debe cumplir con la legislación vigente, se atenderá al siguiente principio:

\begin{itemize}
  \item \textbf{Legalidad:} referido al cumplimiento de las leyes, normas, reglamentaciones o disposiciones a las que está sujeta la Organización, especialmente en materia de protección de datos personales.
\end{itemize}

\item \textbf{Alcance}

Esta Política se aplicará a los sistemas de información de \Beneficiario, relacionados con el ejercicio de sus competencias y a todos los usuarios con acceso autorizado a los mismos. Todos ellos tienen la obligación de conocer y cumplir esta Política de Seguridad de la Información y su Normativa de Seguridad derivada, siendo responsabilidad del Comité de Seguridad disponer los medios necesarios para que la información llegue al personal afectado.

\item \textbf{Gestión del riesgo}

La gestión de la Seguridad de la Información en \Beneficiario será basada en el riesgo, de conformidad con la Norma internacional ISO/IEC 27001.

Se articulará mediante un proceso general de apreciación y tratamiento del riesgo, que potencialmente puede afectar a la seguridad de la información de los servicios prestados, consistente en:

\begin{itemize}
    \item Identificar las amenazas, que aprovecharán vulnerabilidades de los Sistemas de Información que soportan, o de los que depende, la seguridad de la información.
    \item Analizar el riesgo, en base a la consecuencia de materializarse la amenaza y de la probabilidad de ocurrencia.
    \item Evaluar el riesgo, según un nivel previamente establecido y aprobado de riesgo ampliamente aceptable, tolerable e inaceptable.
    \item Tratar el riesgo inaceptable, mediante los controles o salvaguardas adecuadas.
\end{itemize}

Dicho proceso será cíclico y se llevará a cabo de forma periódica, como mínimo una vez al año. Para cada riesgo identificado se asignará un propietario, pudiendo recaer múltiples responsabilidades en una misma persona o comité.

\item \textbf{Marco normativo}

El marco legal en materia de seguridad de la información viene establecido por la siguiente legislación:

\begin{itemize}
    \item El Real Decreto 311/2022, de 3 de mayo, por el que se regula el Esquema Nacional de Seguridad.
    \item La Ley Orgánica 3/2018, de 5 de diciembre, de Protección de Datos Personales y garantía de los derechos digitales, que tiene por objeto adaptar el ordenamiento jurídico español al Reglamento (UE) 2016/679 del Parlamento Europeo y el Consejo, de 27 de abril de 2016, relativo a la protección de las personas físicas en lo que respecta al tratamiento de sus datos personales y a la libre circulación de estos datos, y completar sus disposiciones.
    \item Reglamento (UE) 2016/679 del Parlamento Europeo y del Consejo, de 27 de abril de 2016, relativo a la protección de las personas físicas en lo que respecta al tratamiento de datos personales y a la libre circulación de estos datos y por el que se deroga la Directiva 95/46/CE (RGPD).
    \item Ley 6/2020, de 11 de noviembre, reguladora de determinados aspectos de los servicios electrónicos de confianza, cuyo objeto es la regulación de determinados aspectos de los servicios electrónicos de confianza, como complemento del Reglamento (UE) n.º 910/2014 del Parlamento Europeo y del Consejo, de 23 de julio de 2014, relativo a la identificación electrónica y los servicios de confianza para las transacciones electrónicas en el mercado interior (Reglamento eIDAS).
\end{itemize}

\item \textbf{Objetivos del SGSI}

El SGSI de \Beneficiario deberá garantizar:

\begin{itemize}
    \item Que se desarrollen políticas, normativas, procedimientos y guías operativas para apoyar la política de seguridad de la información.
    \item Que se identifique la información que deba ser protegida.
    \item Que se establezca y mantenga la gestión del riesgo alineada con los requerimientos de la política del SGSI y la estrategia de la Organización.
    \item Que se establezca una metodología para la apreciación y tratamiento del riesgo.
    \item Que se establezcan criterios con los que medir el nivel de cumplimiento del SGSI.
    \item Que se revise el nivel de cumplimiento del SGSI.
    \item Que se corrijan las no conformidades mediante la implementación de acciones correctivas.
    \item Que el personal reciba formación y concienciación sobre la seguridad de la información.
    \item Que todo el personal sea informado sobre la obligación de cumplimiento de la política de seguridad de la información.
    \item La asignación de los recursos necesarios para gestionar el SGSI.
    \item La identificación y cumplimiento de todos los requerimientos legales, regulatorios y contractuales.
    \item Que se identifiquen y analicen las implicaciones de seguridad de la información respecto a los requerimientos de negocio.
    \item Que se mida el grado de madurez del propio sistema de gestión de la seguridad de la información.
    \item Que se realice la mejora continua sobre el SGSI.
\end{itemize}

\item \textbf{Criterios de la seguridad de la informaci\'on}

\Beneficiario{} establece las siguientes acciones para organizar la Seguridad de la Informaci\'on:

\begin{itemize}
  \item Designar\'a roles de seguridad: responsables unificados de Servicios y de la Informaci\'on, Responsable de la Seguridad, Responsable del Sistema y Delegado de Protecci\'on de Datos.
  \item Constituir\'a un \'organo consultivo y estrat\'egico para la toma de decisiones en materia de Seguridad de la Informaci\'on. Este \'{o}rgano se denominar\'a Comit\'e de Seguridad de la Informaci\'on.
\end{itemize}

\item \textbf{Definici\'on de Roles y Responsabilidades}

La estructura organizativa de seguridad, y jerarqu\'ia en el proceso de decisiones la componen:

\begin{tabular}{|p{4cm}|p{11cm}|}
\hline
\textbf{Rol} & \textbf{Funciones} \\
\hline
Direcci\'on & \'{O}rganos colegiados o unipersonales que deciden la misi\'on y los objetivos de la Organizaci\'on. \\
\hline
Comit\'e de Seguridad y Privacidad & \'{O}rganos colegiados o unipersonales que toman decisiones que concretan c\'omo alcanzar los objetivos de seguridad y protecci\'on de la privacidad marcados por los \'{o}rganos de gobierno. \\
\hline
Responsable de la Informaci\'on & A nivel de gobierno. Tiene la responsabilidad \'{u}ltima sobre qu\'e seguridad requiere una cierta informaci\'on manejada por la Organizaci\'on. \\
\hline
Responsable de Servicio & A nivel de gobierno o, en ocasiones baja a nivel ejecutivo. Tiene la responsabilidad \'{u}ltima de determinar los niveles de servicio aceptables por la Organizaci\'on. \\
\hline
Responsable de Seguridad & A nivel ejecutivo. Funciona como supervisor de la operaci\'on del sistema y veh\'iculo de reporte al Comit\'e de Seguridad de la Informaci\'on. \\
\hline
Responsable del Sistema & A nivel operacional. Toma decisiones operativas: arquitectura del sistema, adquisiciones, instalaciones y operaci\'on del d\'ia a d\'ia. \\
\hline
Administradores de seguridad & Son las personas encargadas de ejecutar las acciones diarias de operaci\'on del sistema seg\'un las indicaciones recibidas de sus superiores jer\'arquicos. \\
\hline
Delegado de Protecci\'on de Datos & Figura obligatoria para administraciones p\'ublicas, es el encargado de asesorar y supervisar todos los aspectos relacionados con el tratamiento de datos de car\'acter personal, incluidos los aspectos de seguridad (integridad, confidencialidad y disponibilidad) y violaci\'on de datos personales. Su nombramiento se produce por otra v\'ia ya que sus cometidos no se ci\~nen \'{u}nicamente a aspectos de seguridad. \\
\hline
\end{tabular}

\vspace{1em}

En el caso de PYMES, donde los equipos especializados como los Responsables de Informaci\'on, Seguridad de la Informaci\'on, Sistema y el Delegado de Protecci\'on de Datos (DPD) pueden no estar definidos de manera individual, estas responsabilidades se agrupan de forma m\'as compacta. La direcci\'on de la empresa, el responsable de TI y el DPD (en caso de que exista), se encargan de asumir estas funciones de manera coordinada.

\textbf{Responsables de Informaci\'on y de los Servicios}

Ser\'an funciones de los Responsables de la Informaci\'on y de los Servicios:

\begin{itemize}
  \item Establecer los requisitos de seguridad aplicables a la Informaci\'on (niveles de seguridad de la Informaci\'on) y a los Servicios (niveles de seguridad de los servicios) pudiendo recabar una propuesta al Responsable de Seguridad y teniendo en cuenta la opini\'on del Responsable del Sistema.
  \item Dictaminar respecto a los derechos de acceso a la informaci\'on y los servicios.
  \item Aceptar los niveles de riesgo residual que afectan a la informaci\'on y los servicios.
  \item Poner en comunicaci\'on del Responsable de Seguridad cualquier variaci\'on respecto a la Informaci\'on y los Servicios de los que es responsable, especialmente la incorporaci\'on de nuevos Servicios o Informaci\'on a su cargo. El cual dar\'a traslado de dichos cambios, al Comit\'e de Seguridad de la Informaci\'on, en su pr\'oxima reuni\'on.
  \item Tiene la responsabilidad \'{u}ltima del uso que se haga de determinados servicios e informaci\'on y, por tanto, de su protecci\'on.
\end{itemize}

\textbf{Responsable de la Seguridad de la informaci\'on}

Ser\'an funciones del Responsable de Seguridad:

\begin{itemize}
  \item Mantener y verificar el nivel adecuado de seguridad de la Informaci\'on manejada y de los Servicios electr\'onicos prestados por los sistemas de informaci\'on.
  \item Promover la formaci\'on y concienciaci\'on en materia de seguridad de la informaci\'on.
  \item Designar responsables de la ejecuci\'on del an\'alisis de riesgos, de la Declaraci\'on de Aplicabilidad, identificar medidas de seguridad, determinar configuraciones necesarias y elaborar documentaci\'on del sistema.
  \item Aprobar la Declaraci\'on de Aplicabilidad a partir de las medidas de seguridad requeridas conforme al Anexo II del ENS y la ISO 27001.
  \item Proporcionar asesoramiento para la determinaci\'on de la Categor\'ia del Sistema, en colaboraci\'on con el Responsable del Sistema y/o Comit\'e de Seguridad.
  \item Participar en la elaboraci\'on e implantaci\'on de los planes de mejora de la seguridad y, llegado el caso, en los planes de continuidad, procediendo a su validaci\'on.
  \item Gestionar las revisiones externas o internas del sistema, as\'i como los procesos de certificaci\'on.
  \item Elevar al Comit\'e de Seguridad la aprobaci\'on de cambios y otros requisitos del sistema.
  \item Aprobar los procedimientos de seguridad que forman parte del Mapa Normativo (y no son competencia del Comit\'e) y poner en conocimiento al Comit\'e de las modificaciones que se hayan realizado a lo largo del periodo en curso.
  \item Participar\'a en la elaboraci\'on, en el marco del Comit\'e de Seguridad de la Informaci\'on, de la Pol\'itica de Seguridad de la Informaci\'on, para su aprobaci\'on por Direcci\'on.
  \item Actuar\'a como secretario del Comit\'e de Seguridad de la Informaci\'on, realizando las siguientes funciones:
  \begin{itemize}
    \item Convocar las reuniones del Comit\'e de Seguridad de la Informaci\'on.
    \item Preparar los temas a tratar en las reuniones del Comit\'e, aportando informaci\'on puntual para la toma de decisiones.
    \item Elaborar el acta de las reuniones.
    \item Es responsable de la ejecuci\'on directa o delegada de las decisiones del Comit\'e.
  \end{itemize}
\end{itemize}

\textbf{Responsable del Sistema}

Ser\'an funciones del Responsable del Sistema:

\begin{itemize}
  \item Desarrollar, operar y mantener el sistema de informaci\'on durante todo su ciclo de vida, elaborando los procedimientos operativos necesarios.
  \item Definir la topolog\'ia y la gesti\'on del Sistema de Informaci\'on estableciendo los criterios de uso y los servicios disponibles en el mismo.
  \item Detener el acceso a informaci\'on o prestaci\'on de servicio si tiene el conocimiento de que estos presentan deficiencias graves de seguridad.
  \item Cerciorarse de que las medidas espec\'ificas de seguridad se integren adecuadamente dentro del marco general de seguridad.
  \item Proporcionar asesoramiento para la determinaci\'on de la Categor\'ia del Sistema, en colaboraci\'on con el Responsable de Seguridad y/o Comit\'e de Seguridad de la Informaci\'on.
  \item Participar en la elaboraci\'on e implantaci\'on de los planes de mejora de la seguridad y llegado el caso en los planes de continuidad.
  \item Coordinar las funciones del administrador de la seguridad del sistema:
  \begin{itemize}
    \item La gesti\'on, configuraci\'on y actualizaci\'on, en su caso, del hardware y software.
    \item La gesti\'on de las autorizaciones concedidas a los usuarios del sistema, en particular los privilegios concedidos.
    \item Aprobar los cambios en la configuraci\'on vigente del Sistema de Informaci\'on.
    \item Asegurar que los controles de seguridad establecidos son cumplidos estrictamente.
    \item Asegurar que son aplicados los procedimientos aprobados para manejar el Sistema de Informaci\'on.
    \item Supervisar las instalaciones de hardware y software, sus modificaciones y mejoras.
    \item Monitorizar el estado de seguridad.
    \item Informar al Responsable de Seguridad de cualquier anomal\'ia, compromiso o vulnerabilidad relacionada con la seguridad.
    \item Colaborar en la investigaci\'on y resoluci\'on de incidentes de seguridad, desde su detecci\'on hasta su resoluci\'on.
  \end{itemize}
\end{itemize}

\textbf{Delegado de Protecci\'on de Datos (DPD)}

Ser\'an funciones del Delegado de Protecci\'on de Datos:

\begin{itemize}
    \item Informar y asesorar a la organizaci\'on, y a los usuarios que se ocupen del tratamiento, de las obligaciones que les incumben en virtud de la normativa vigente en materia de Protecci\'on de Datos.
    \item Supervisar el cumplimiento de lo dispuesto en normativa de seguridad y de las pol\'iticas internas de la organizaci\'on, en materia de protecci\'on de datos, incluida la asignaci\'on de responsabilidades, la concienciaci\'on y formaci\'on del personal que participa en las operaciones de tratamiento, y las auditor\'ias correspondientes.
    \item Ofrecer el asesoramiento que se le solicite acerca de la evaluaci\'on de impacto relativa a la protecci\'on de datos y supervisar\'a su aplicaci\'on.
    \item Cooperar con la Agencia Espa\~nola de Protecci\'on de Datos cuando \'esta lo requiera, actuando como punto de contacto con \'esta para cuestiones relativas al tratamiento de datos.
    \item El Delegado de Protecci\'on de Datos desempe\~nar\'a sus funciones prestando atenci\'on a los riesgos asociados a las operaciones de tratamiento. Para ello, debe ser capaz de:
    \item Recabar informaci\'on para determinar las actividades de tratamiento.
    \item Analizar y comprobar la conformidad de las actividades de tratamiento.
    \item Informar, asesorar y emitir recomendaciones al responsable o el encargado del tratamiento.
    \item Recabar informaci\'on para supervisar el registro de las operaciones de tratamiento.
    \item Asesorar en el principio de la protecci\'on de datos por dise\~no y por defecto.
    \item Asesorar sobre si se lleva a cabo o no las evaluaciones de impacto, metodolog\'ia, salvaguardas a aplicar, etc.
    \item Priorizar actividades en base a los riesgos.
    \item Asesorar al Responsable de Tratamiento sobre \'{a}reas a cometer a auditor\'ias, actividades de formaci\'on a realizar y operaciones de tratamiento a dedicar m\'as tiempo y recursos.
\end{itemize}

\item \textbf{Comit\'e de Seguridad de la Informaci\'on}

\Beneficiario{} debe crear un Comit\'e de Seguridad para coordinar la protecci\'on de datos, gestionar riesgos y garantizar el cumplimiento normativo, fortaleciendo as\'i la cultura de seguridad en toda la organizaci\'on.

Este comit\'e deber\'a estar compuesto por los siguientes miembros que se clasificar\'an en permanentes o no permanentes atendiendo a la obligatoriedad de la participaci\'on del Comit\'e de Seguridad de la Informaci\'on:

\textbf{Miembros permanentes:}
\begin{itemize}
  \item Direcci\'on
  \item Responsable de Sistema (RSIS)
  \item Responsable de Seguridad de la Informaci\'on (RSEG)
  \item Asesores que se consideren oportunos para los temas en cuesti\'on con voz, pero sin voto.
\end{itemize}

\textbf{Miembros no permanentes:}
\begin{itemize}
  \item Responsables del Servicio y de la Informaci\'on.
  \item El Delegado de Protecci\'on de Datos.
  \item Representantes de la organizaci\'on, especialistas externos del sector p\'ublico o privado cuya presencia, por raz\'on de su experiencia o vinculaci\'on con los asuntos tratados, sea necesaria o aconsejable.
  \item Los Responsables de la Informaci\'on y los Servicios ser\'an convocados por la presidencia en funci\'on de los asuntos a tratar, en representaci\'on de los distintos \'{a}mbitos o \'{a}reas de seguridad TIC. Cada \'{a}rea estar\'a representada por un vocal con voto, sin perjuicio de que acudan varios representantes de la misma.
  \item El Delegado de Protecci\'on de Datos participar\'a con voz, pero sin voto, en las reuniones del Comit\'e de Seguridad de la Informaci\'on cuando se aborden cuestiones relacionadas con el tratamiento de datos de car\'acter personal, o cuando se requiera su participaci\'on. En todo caso, si un asunto se sometiese a votaci\'on, se har\'a constar en acta la opini\'on del Delegado de Protecci\'on de Datos.
  \item El Secretario/a del Comit\'e realizar\'a las convocatorias y levantar\'a actas de las reuniones del Comit\'e de Seguridad. A las sesiones podr\'an asistir asesores invitados por el Presidente.
\end{itemize}

En una PYME donde no es posible disponer de todos los cargos necesarios para el Comit\'e, \Beneficiario{} podr\'a designar miembros con m\'ultiples roles o incorporar asesores externos. Aunque reducido, este comit\'e sigue siendo esencial para coordinar la seguridad, gestionar riesgos y fomentar una cultura de protecci\'on.

\textbf{Funciones del Comit\'e de Seguridad:}
\begin{itemize}
  \item Atender las inquietudes de la Alta Direcci\'on y de los diferentes departamentos.
  \item Informar regularmente del estado de la seguridad de la informaci\'on a la Alta Direcci\'on.
  \item Promover la mejora continua del SGSI.
  \item Elaborar la estrategia de evoluci\'on de la seguridad de la informaci\'on.
  \item Promover la realizaci\'on de auditor\'ias peri\'odicas.
  \item Aprobar documentaci\'on de seguridad.
  \item Estar informado sobre la normativa y entidades relacionadas con la certificaci\'on del ENS.
  \item Proponer directrices y recomendaciones, que se reflejar\'an en actas.
  \item Coordinar esfuerzos de diferentes \'{a}reas en seguridad.
  \item Resolver conflictos de responsabilidad o escalarlos si fuera necesario.
  \item Asesorar en materia de seguridad cuando se le requiera.
  \item Revisar la Pol\'itica de Seguridad antes de su aprobaci\'on.
  \item Aprobar el Plan de Adecuaci\'on para la implantaci\'on del ENS.
\end{itemize}

\textbf{Periodicidad de reuniones y adopci\'on de acuerdos:}
\begin{itemize}
  \item El Comit\'e se reunir\'a al menos una vez al a\~no, pudiendo convocarse m\'as a menudo si es necesario.
  \item Las reuniones se convocar\'an por su Presidencia, por iniciativa propia o a petici\'on de la mayor\'ia de sus miembros permanentes.
  \item Las decisiones se tomar\'an por consenso de los miembros permanentes.
\end{itemize}

\textbf{Designaci\'on y resoluci\'on de conflictos:}
\begin{itemize}
  \item La creaci\'on y constituci\'on del Comit\'e, as\'i como la designaci\'on de sus miembros, se formalizar\'a mediante acta inicial.
  \item Los roles nombrados se renovar\'an autom\'aticamente cada a\~no. Cualquier cambio se comunicar\'a siguiendo el procedimiento establecido.
  \item Seg\'un el art\'iculo 13.3 del RD del ENS, no puede existir dependencia jer\'arquica entre el RSEG y el RSIS, salvo excepciones justificadas. En tal caso, se establecer\'an medidas compensatorias.
\end{itemize}

\item \textbf{Datos personales y riesgos derivados del tratamiento}

El legislador, en el art\'iculo 12.1.f) del RD del ENS, hace referencia al cumplimiento adecuado en materia de protecci\'on de datos personales. Para dar cumplimiento a la LOPD-GDD y al RGPD, \Beneficiario{}:

\begin{itemize}
  \item Publicar\'a el registro de actividades de tratamiento.
  \item Realizar\'a an\'alisis de riesgos y evaluaciones de impacto (EIPD) cuando sea necesario.
  \item Aplicar\'a las fases indicadas en los Anexos I y II del RD del ENS para los sistemas afectados.
\end{itemize}

\item \textbf{Obligaciones del personal}

Todo el personal, tanto interno como externo, que interact\'ue con el sistema de informaci\'on, deber\'a cumplir con la presente Pol\'itica de Seguridad de la Informaci\'on.

\item \textbf{Formaci\'on y concienciaci\'on}

El Responsable de Seguridad del SGSI deber\'a garantizar que todo el personal involucrado:

\begin{itemize}
  \item Conoce la pol\'itica, objetivos y procesos del SGSI.
  \item Recibe acciones formativas y de concienciaci\'on.
  \item Tiene acceso a la documentaci\'on correspondiente seg\'un su rol.
\end{itemize}

\item \textbf{Terceras partes}

Cuando \Beneficiario{} preste servicios o maneje informaci\'on de terceros:

\begin{itemize}
  \item Se les har\'a part\'icipes de esta Pol\'itica de Seguridad.
  \item Se establecer\'an canales de reporte y coordinaci\'on con comit\'es correspondientes.
  \item Deber\'an aplicar esta normativa, pudiendo desarrollar procedimientos propios.
  \item Su personal recibir\'a formaci\'on equivalente a la de esta pol\'itica.
  \item Si no pueden cumplir alg\'un punto, el Responsable de Seguridad emitir\'a informe de riesgos que deber\'a ser aprobado por los responsables implicados.
  \item Se establecer\'a un Punto Operacional de Comunicaci\'on (POC) para el enlace.
\end{itemize}

\item \textbf{Auditor\'ia}

La Direcci\'on General de \Beneficiario{} garantizar\'a, mediante auditor\'ias internas y/o externas, el cumplimiento de esta pol\'itica, aplicando medidas correctoras necesarias para la mejora continua.

\item \textbf{Validez y actualizaci\'on}

Esta pol\'itica ser\'a efectiva desde su publicaci\'on y se revisar\'a anualmente. En la revisi\'on se considerar\'an:

\begin{itemize}
  \item Cambios en el contexto interno o externo de la organizaci\'on.
  \item Incidentes y No Conformidades.
  \item Resultados de los procesos de apreciaci\'on de riesgo.
  \item Actualizaci\'on de normas y documentos relacionados.
\end{itemize}

Se elaborar\'a una lista de objetivos y acciones a emprender durante el a\~no siguiente.

\item \textbf{Sanciones}

El incumplimiento de esta pol\'itica y sus procedimientos podr\'a dar lugar a sanciones conforme a la legislaci\'on laboral vigente, seg\'un la magnitud del incumplimiento.

\item \textbf{Ratificaci\'on}

Todos los abajo firmantes asumen y aceptan plenamente el contenido de esta Pol\'itica y se comprometen a aplicarla en sus respectivas \'{a}reas para lograr el correcto funcionamiento del SGSI.

\end{enumerate}

% !TEX root = ../Guia SGSI.tex

\subsubsection{Normativa de control de acceso}

\begin{enumerate}[label=\alph*)]

    \item \textbf{Política de control de acceso lógico}

    \Beneficiario{} seguirá las siguientes pautas para controlar el acceso lógico (a través de medios telemáticos):

    \begin{itemize}
      \item Se aplicarán controles de acceso en todos los niveles de la arquitectura y tipología de los Sistemas de Información de la Organización. Esto incluirá: redes, plataformas o sistemas operativos, bases de datos y aplicaciones. Los atributos de cada uno de ellos deberán reflejar alguna forma de identificación y autenticación, autorización de acceso, verificación de recursos de información y registro y monitorización de las actividades.
      \item Los usuarios podrán acceder a los recursos necesarios para realizar las labores propias de su puesto. Los derechos de acceso a los mismos también serán los mínimos posibles en función de dichas necesidades.
      \item El conocimiento y formación de los usuarios en el uso correcto de los medios de control de acceso será fundamental para garantizar la efectividad de la presente política y su desarrollo. Se deberán desarrollar actividades de formación y se establecerán medios para comunicar a los usuarios y diferentes responsables sobre el uso correcto de los medios de acceso a sistemas y servicios.
      \item El uso de la informática móvil y teletrabajo deberá tener un nivel de seguridad equiparable al existente en el uso de equipos locales. Las medidas de seguridad a adoptar deberán tener en cuenta, en todo caso, los riesgos que este tipo de forma de trabajar puedan llevar implícitos como, por ejemplo, el entorno de trabajo en que estas actividades se desarrollen, que deberá ser adecuadamente protegido y los procesos de autenticación de usuarios y máquinas.
      \item La implementación de los controles de acceso deberá tener en cuenta los tipos de accesos posibles y sus riesgos, la criticidad de la información que se aloje en ellos y los requisitos legales aplicables.
      \item El acceso a los Sistemas de Información requerirá siempre de autenticación.
      \item Los usuarios deberán siempre autenticarse como usuarios no privilegiados del sistema, excepcionalmente y sólo con fines de administración podrán autenticarse como administradores de este.
      \item Todas las contraseñas asignadas a las cuentas de usuario deberán respetar la política de contraseñas detallada en el presente documento.
      \item Los usuarios deberán usar la información y los sistemas de información, garantizando el nivel de seguridad adecuado según las directrices marcadas en las normas de uso de los sistemas de información.
      \item Periódicamente se revisarán los derechos de acceso asignados a los usuarios para cada sistema y aplicaciones detallados en el alcance de este documento. Los derechos de acceso privilegiados deberán revisarse con una periodicidad menor. Además de lo anterior, deberá realizarse una revisión de los permisos de acceso correspondientes a un usuario siempre que esta sufra una modificación significativa de sus responsabilidades, posición o rol en la organización.
    \end{itemize}

    \item \textbf{Identificadores}

    \Beneficiario{} debe establecer un procedimiento para definir los identificadores de sus usuarios según estas características:

    \begin{itemize}
      \item Su superior jerárquico autorizará la creación de un identificador de usuario, según el procedimiento de altas, bajas y modificaciones de permisos de usuarios.
      \item No se permitirá el uso de identificadores de grupo o genéricos, salvo cuando sea estrictamente necesario y por razones operacionales. Esta circunstancia deberá estar debidamente justificada y aprobada formalmente, y se aplicarán los controles de seguridad precisos.
      \item Los identificadores de usuarios anónimos y los identificadores por defecto estarán siempre deshabilitados.
      \item Los identificadores no deberán dar indicios de nivel de privilegio asociado.
      \item Cuando sea posible, se establecerán listas de control de acceso a los recursos de información.
      \item Los identificadores, siempre que sea posible, deberán contar con una asignación y una fecha de validez, tras la cual se deshabilitarán.
      \item Los usuarios serán responsables de todas las actividades realizadas con sus identificadores, contraseñas y dispositivos de acceso. Por lo tanto, no deberán permitir que otras personas los utilicen y conozcan.
    \end{itemize}

    \item \textbf{Política de contraseñas}

    Las contraseñas (junto con el código de usuario o user-id) son el medio de acceso al sistema de información de \Beneficiario{}. Es necesario que las contraseñas que se utilicen como mecanismo de autenticación sean robustas para dificultar su vulneración.

    \Beneficiario{} debe implementar una política de contraseñas siguiendo las siguientes normas de seguridad:

    \begin{table}[H]
        \centering
        \small
        \tablacentrada{3.5cm}{
            \titulofila{Generación de contraseñas}
            \textbf{Longitud} & Deberán tener una longitud igual o superior a 10 caracteres. \\ \hline
            \textbf{Complejidad} &
            \begin{itemize}[leftmargin=1em, topsep=0pt, itemsep=0pt]
            \item No deberán contener la información del usuario, como el DNI/NIE, nombre, apellidos, etc.
            \item Deberán estar compuestas por al menos 3 de los siguientes 4 conjuntos de caracteres:
            \begin{enumerate}[label=\arabic*., leftmargin=1em, topsep=0pt, itemsep=0pt]
                \item Caracteres alfanuméricos en mayúsculas.
                \item Caracteres alfanuméricos en minúsculas.
                \item Caracteres numéricos.
                \item Símbolos/caracteres especiales.
            \end{enumerate}
            \end{itemize} \\ \hline
            \textbf{Repetición} & No deberán ser igual a ninguna de las 3 últimas contraseñas usadas. Se evitará usar contraseñas similares. \\ \hline
            \textbf{Semántica} &
            \begin{itemize}[leftmargin=1em, topsep=0pt, itemsep=0pt]
            \item Repetición de caracteres.
            \item Palabras del diccionario.
            \item Secuencias simples de letras, números o secuencias de teclado.
            \item Información que pueda asociarse fácilmente al usuario como nombres de familiares o mascotas, números de teléfono, matrículas, fechas o en general información biográfica del usuario.
            \end{itemize} \\ \hline
            \textbf{Precauciones} &
            \begin{itemize}[leftmargin=1em, topsep=0pt, itemsep=0pt]
            \item Se evitará apuntar las contraseñas en papel, enviarlas por medios electrónicos o almacenarlas sin cifrar.
            \item Se mantendrá el carácter secreto.
            \item No se reutilizarán contraseñas entre servicios.
            \item Se reforzará la seguridad en función de la sensibilidad de la información.
            \end{itemize} \\ \hline
        }
        \caption{Generación de contraseñas}
        \label{tab:guia-sgsi-generación-contraseñas}
        \normalsize
    \end{table}

    \vspace{1em}

    \begin{table}[H]
        \centering
        \small
        \tablacentrada{3.5cm}{
            \titulofila{Distribución de contraseñas}
            \textbf{Medio de entrega} & Las contraseñas iniciales deberán ser entregadas en mano o por medios que no permitan su acceso por personas no autorizadas. Si se usan medios electrónicos, se enviarán separadas del identificador. \\ \hline
            \textbf{Contraseñas iniciales} & Se generarán automáticamente y deberán cambiarse en el primer acceso. \\ \hline
        }
        \caption{Distribución de contraseñas}
        \label{tab:guia-sgsi-distribución-contraseñas}
        \normalsize
    \end{table}

    \vspace{1em}

    \begin{table}[H]
        \centering
        \small
        \tablacentrada{3.5cm}{
            \titulofila{Uso de contraseñas}
            \textbf{Renovación} &
            \begin{itemize}[leftmargin=1em, topsep=0pt, itemsep=0pt]
            \item Deben renovarse al menos una vez al año, o más frecuentemente según la criticidad.
            \item El sistema debe forzar el cambio o el usuario debe hacerlo manualmente.
            \end{itemize} \\ \hline
            \textbf{Cambio} &
            \begin{itemize}[leftmargin=1em, topsep=0pt, itemsep=0pt]
            \item Permitido tras olvido o bloqueo.
            \item Cambio obligatorio tras cesión o compromiso.
            \item Siempre comunicado directamente al usuario.
            \end{itemize} \\ \hline
            \textbf{Custodia} &
            \begin{itemize}[leftmargin=1em, topsep=0pt, itemsep=0pt]
                \item No se comunicarán por medios inseguros ni se escribirán en claro.
            \end{itemize} \\ \hline
            \textbf{Gestión} & Se usará una herramienta específica como KeePass o Lastpass, manteniéndola cifrada o protegida. \\ \hline
        }
        \caption{Uso de contraseñas}
        \label{tab:guia-sgsi-uso-contraseñas}
        \normalsize
    \end{table}

    \vspace{1em}

    \begin{table}[H]
        \centering
        \small
        \tablacentrada{3.5cm}{
            \titulofila{Contraseñas en los sistemas}
            \textbf{Pantalla} & No se mostrarán en claro. \\ \hline
            \textbf{Salvapantalla} & Deberá estar protegido por contraseña tras inactividad. \\ \hline
            \textbf{Por defecto} & Serán cambiadas o desactivadas. \\ \hline
            \textbf{Recordar contraseña} & Se evitará usar esta función. \\ \hline
            \textbf{Expiración automática} & Existirán mecanismos que obliguen al cambio. \\ \hline
        }
        \caption{Contraseñas en los sistemas}
        \label{tab:guia-sgsi-contraseñas-sistemas}
        \normalsize
    \end{table}

    \item \textbf{Inicio seguro de sesión}

    El acceso a los sistemas operativos de \Beneficiario{} está controlado por un proceso de inicio de sesión seguro diseñado para minimizar los intentos de accesos no autorizados, y contará con las siguientes características:

    \begin{table}[H]
        \centering
        \small
        \tablacentrada{3.5cm}{
            \titulofila{Inicio seguro de sesión}
            \textbf{Información del sistema} & Hasta que no se complete con éxito el proceso de autenticación, no se deberá mostrar ningún tipo de información del sistema (tal como identificadores del sistema o versiones de software instalado) que puedan ayudar a identificarlo, así como cualquier otro tipo de información que pueda facilitar su acceso no autorizado. \\ \hline
            \textbf{Nº de intentos de acceso} & El número de intentos de acceso en los sistemas estará limitado a un máximo de 5 intentos. Además, la cuenta permanecerá bloqueada al menos entre 15 y 30 minutos desde el último intento fallido (en función de la criticidad del sistema). \\ \hline
            \textbf{Tiempo de acceso} & Deberá limitarse el tiempo mínimo y máximo permitido para el proceso de acceso a los sistemas, como por ejemplo que se acceda durante el horario de oficina. Si se excede, el sistema deberá finalizar el proceso de acceso o “log in”. \\ \hline
        }
        \caption{Inicio seguro de sesión}
        \label{tab:guia-sgsi-inicio-sesión}
        \normalsize
    \end{table}

    \vspace{1em}

    \item \textbf{Desconexión automática de terminales}

    \Beneficiario{} dispone de controles que cierran las sesiones abiertas e inactivas durante un tiempo superior a 10 minutos, al menos, en los sistemas más sensibles y en los accesos privilegiados.

    \item \textbf{Monitorización de accesos}

    \Beneficiario{} no dispone de un procedimiento de monitorización de los sistemas para detectar accesos no autorizados y desviaciones, registrando eventos que suministren evidencias en caso de que se produzcan incidentes relativos a la seguridad. \Beneficiario{} debe implementar un sistema de monitorización de accesos para detectar intrusiones no autorizadas. Así, se tendrán en cuenta:

    \begin{table}[H]
        \centering
        \small
        \tablacentrada{3.5cm}{
            \titulofila{Monitorización de accesos}
            \textbf{Registro de eventos} &
            \begin{itemize}[leftmargin=1em, topsep=0pt, itemsep=0pt]
                \item Intentos de acceso fallidos.
                \item Bloqueos de cuenta.
                \item Debilidad de contraseñas.
                \item Normalización de identificadores.
                \item Cuentas inactivas y deshabilitadas.
                \item Últimos accesos a cuentas.
            \end{itemize} \\ \hline
            \textbf{Registro de uso de los sistemas} &
            \begin{itemize}[leftmargin=1em, topsep=0pt, itemsep=0pt]
                \item Accesos no autorizados.
                \item Uso de privilegios.
                \item Alertas de sistema.
            \end{itemize} \\ \hline
        }
        \caption{Monitorización de accesos}
        \label{tab:guia-sgsi-monitorización-accesos}
        \normalsize
    \end{table}

    \Beneficiario{} debe establecer las normas y mecanismos de protección para controlar los accesos a las redes que están incluidas en el alcance y asegurará que no se hace un uso indebido de sus recursos de información. Para ello se establecerán los siguientes controles:

    \begin{itemize}
        \item Interfaces apropiados entre la red corporativa de la Organización y las redes públicas.
        \item Mecanismos de autenticación apropiados en los equipos de los usuarios.
        \item Sistemas de control de acceso para restringir el acceso de los usuarios a la información.
    \end{itemize}

    \item \textbf{Utilización de las prestaciones del sistema}

    La utilización de programas o utilidades que puedan eludir los controles de seguridad de los sistemas y aplicaciones estará restringido y estrechamente controlado. Se establecerán los siguientes controles que limitarán el uso de estas prestaciones en el sistema:

    \begin{itemize}
        \item Procesos de identificación, autenticación y autorización formales para el acceso a estas prestaciones.
        \item Limitar el uso de prestaciones del sistema al mínimo número de usuarios posible.
        \item Autorizar el uso de prestaciones con un propósito concreto.
        \item Registrar siempre el uso de las prestaciones del sistema mediante el uso de eventos del sistema adecuadamente protegidos.
        \item Definir y documentar los niveles de autorización para las prestaciones del sistema.
    \end{itemize}

    \item \textbf{Restricción de acceso a las aplicaciones}

    Los usuarios recibirán el mínimo nivel de acceso a las aplicaciones necesario según sus funciones dentro de \Beneficiario{}, ya que un nivel de acceso por encima de dichas necesidades podría ocasionar un riesgo para la confidencialidad e integridad de la información. Para ello, se establecerán restricciones de los derechos de acceso de los usuarios (por ejemplo, lectura, escritura, borrado, ejecución) a través de cada aplicación.

    Además, se garantizará que:
    \begin{itemize}
        \item No se comprometa la seguridad de otros sistemas con los que se compartan recursos.
        \item Los equipos de los usuarios de \Beneficiario{} solo tendrán instaladas las aplicaciones estrictamente necesarias para la realización de sus tareas profesionales. \Beneficiario{} debe establecer una política de instalación mínima de aplicaciones para evitar software no deseado con potenciales vulnerabilidades susceptibles de ser explotadas por atacantes.
        \item El acceso lógico a las aplicaciones y a la información tratada en ellas estará restringido únicamente a los usuarios autorizados. \Beneficiario{} debe establecer una política de acceso restringido a usuarios autorizados para proteger sus datos y aplicaciones de accesos no autorizados y exfiltraciones.
    \end{itemize}

    \item \textbf{Política de uso de los servicios de red}

    Los usuarios de \Beneficiario{} únicamente tendrán acceso a aquellos servicios de red cuyo uso les haya sido específicamente autorizado. Para ello:
    \begin{itemize}
        \item Los servicios de red solo podrán utilizarse para la función para la que se han dispuesto, quedando prohibido su uso para otros cometidos o para llevar a cabo funciones fuera de las asociadas al puesto desempeñado.
        \item Los privilegios asignados a los usuarios para acceder a las redes de \Beneficiario{} serán registrados y revisados por el responsable correspondiente, quien podrá solicitar información sobre los requisitos de acceso del usuario a los propietarios de las redes.
        \item Se emplearán elementos de seguridad de red para garantizar las conexiones de los usuarios que se encuentran en redes internas, así como aquellas conexiones realizadas desde redes externas, en función del riesgo existente en cada una de estas conexiones. En este sentido, la Organización segregará sus redes de manera que se pueda restringir el acceso a los servicios proporcionados en los siete niveles de red definidos por el estándar OSI.
        \item \Beneficiario{} debe implementar un sistema de VPN para el uso de conexiones seguras a través de redes públicas con la finalidad de evitar ataques Man in the Middle, intercepción de tráfico o robo de credenciales.
        \item Para los accesos remotos a la red corporativa de \Beneficiario{}, se establecerán controles y mecanismos para tratar convenientemente la información transmitida, los sistemas y recursos accedidos, la identidad de las personas que realicen dichos accesos y las posibles implicaciones que el acceso en global conlleve.
        \item \Beneficiario{} debe habilitar una DMZ para aislar sus servicios WEB más expuestos y proteger su red interna de ataques directos. Los accesos al resto de servicios de la Organización también deberán ser bloqueados mediante la configuración de las reglas de acceso en los cortafuegos establecidos al efecto.
        \item Se registrarán y monitorizarán los accesos a las redes y servicios proporcionados por la Organización a fin de controlar y prevenir accesos no autorizados.
    \end{itemize}

    \item \textbf{Protección de los puertos de diagnóstico y configuración remota}

    Gran parte de los equipos informáticos, sistemas de red y comunicaciones disponen de funcionalidades para el diagnóstico y configuración remota. Si estos sistemas no están bien protegidos se convertirán en puntos de acceso incontrolado.

    Por tanto, los puertos de diagnóstico de los sistemas de \Beneficiario{} estarán controlados y protegidos frente a accesos no autorizados tanto a nivel físico como lógico.

    \begin{itemize}
        \item Los armarios y racks en el que se encuentren estos equipos estarán cerrados con llave para asegurar que no se produzcan accesos físicos no permitidos.
        \item Los servidores y sistemas de comunicación de \Beneficiario{} tendrán abiertos los puertos estrictamente necesarios para su uso y explotación.
        \item Los puertos, servicios y herramientas de configuración y diagnóstico instalados en equipos o dispositivos de red cuyo uso no sea necesario para los propósitos de \Beneficiario{} estarán deshabilitados.
    \end{itemize}

    \item \textbf{Segregación de redes}

    Los sistemas de información de \Beneficiario{} no están segmentados en diferentes redes que permitan únicamente el tráfico necesario y autorizado.

    \Beneficiario{} debe implementar una segmentación de redes mediante VLAN para separar el tráfico entre distintos grupos de usuarios y dispositivos y limitar la propagación de ataques potenciales.

    A modo de ejemplo:
    \begin{itemize}
        \item Redes Internas: Redes propias de la Organización.
        \item Redes de Acceso Internas: Conexiones desde las diferentes instalaciones propias y/o autorizadas de la Organización.
    \end{itemize}

    Los perímetros de seguridad entre cada VLAN se controlarán mediante el uso de dispositivos de red que gestionen el acceso y tráfico de información entre redes, bloqueando los accesos no autorizados.

    \Beneficiario{} dispondrá de medidas de seguridad en las redes inalámbricas (Wifi) que garanticen la autenticidad, confidencialidad e integridad de la información que viaje a través de dicha red. La autenticación en la red WIFI usará contraseñas robustas para evitar ataques de fuerza bruta o de diccionario y evitar accesos no autorizados. Asimismo, el usuario deberá autenticarse en cada servicio proporcionado a través de la red WIFI.

    \item \textbf{Control de conexiones a la red}

    El control de las conexiones a la red se deberá realizar utilizando dispositivos de filtrado, ruteado y control a interconexión entre redes.

    En el caso de redes compartidas, sobre todo las que se extienden a través de los límites de \Beneficiario{}, el control deberá llevarse a cabo mediante distintos dispositivos que regulen la capacidad de los usuarios de conectarse a y desde las redes de la Organización. Estos controles podrán comprender:

    \begin{itemize}
        \item \Beneficiario{} no dispone de sistemas proxy de control, registro y/o prohibición del tráfico entrante y saliente a internet. \Beneficiario{} debe implementar el uso de proxies para evitar accesos no autorizados y proteger su red interna.
        \item \Beneficiario{} puede restringir el acceso a servicios fuera de horario laboral para limitar ventanas de entrada por parte de los atacantes.
    \end{itemize}

    \item \textbf{Control de enrutamiento de red}

    Se realizarán controles de enrutamiento para garantizar que el tráfico transmitido a través de las redes de \Beneficiario{} sea el necesario y esté debidamente autorizado.

    Estos controles de rutas se basarán en sistemas que permitan realizar un filtrado de información en base a la dirección de origen y dirección de destino (Ej. firewalls, routers, etc.).

    \item \textbf{Control de la seguridad de red}

    Las redes de \Beneficiario{} deberán estar permanentemente protegidas frente a amenazas. La Organización deberá realizar controles como:

    \begin{itemize}
        \item \Beneficiario{} debe implantar una política de escaneos de vulnerabilidades de sus redes internas y externas para detectar vectores de ataque usados por potenciales atacantes.
        \item \Beneficiario{} debe realizar pruebas de pentesting en su infraestructura de red con la finalidad de asegurar una defensa continua contra nuevas amenazas.
        \item \Beneficiario{} debe implantar herramientas de monitorización para su infraestructura con la finalidad de detectar comportamientos y actividades anómalas en su red corporativa.
    \end{itemize}

\end{enumerate}

% !TEX root = ../Guia SGSI.tex

\subsubsection{Gestión de acceso de usuarios}

\begin{enumerate}[label=\alph*)]

    \item \textbf{Registro de usuarios:} \Beneficiario{} deberá implantar las pautas dictadas por esta normativa mediante procedimientos formales en materia de registro de usuarios y gestión de permisos que garanticen el acceso de los usuarios a la información y sistemas para los cuales estén autorizados. Como regla general, estos procedimientos deberán hacer referencia a los siguientes aspectos:

    \begin{itemize}
        \item El uso de identificadores de usuarios genéricos siempre debe ser excepcional y estar justificado y documentado en todos los casos. \Beneficiario{} debe aplicar una política de usuarios con identificador único para localizarlo con claridad y evitar suplantaciones de identidad.
        \item Se verifica que el usuario tiene la autorización del propietario o responsable del sistema para acceder a la información y/o al sistema para su tratamiento, antes de facilitarle el acceso. Siempre deberá ser el responsable del sistema quien apruebe los cambios en derechos y permisos de acceso.
        \item Se garantiza que el acceso no será efectivo hasta que se hayan completado los procedimientos de autorización y registro.
        \item Se mantiene un registro actualizado de las personas autorizadas para usar un servicio concreto.
        \item Se verifica que el nivel de acceso concedido es el adecuado para la actividad a realizar, debiendo verificar adicionalmente que cumple con lo establecido por la normativa de seguridad de la organización.
        \item Se informa al usuario de la normativa aplicable sobre control de acceso a la información y/o sistema para su tratamiento.
        \item \Beneficiario{} no retira de forma inmediata los derechos de acceso de aquellos usuarios que dejen la organización. Del mismo modo, si existe un cambio de puesto, no se retiran los permisos anteriores para asignar los nuevos.
        \item \Beneficiario{} revisa y elimina de forma periódica los identificadores de usuario y cuentas obsoletas por inactividad. La Organización debe implantar un procedimiento para eliminar cuentas de usuarios que dejan la organización o con largos periodos de inactividad confirmada para evitar el uso malicioso de credenciales robadas a miembros que ya no forman parte de la Organización.
        \item \Beneficiario{} no garantiza la inexistencia de identificadores de usuario duplicados y debe definir un procedimiento para evitar incoherencias y potenciales suplantaciones de identidad.
    \end{itemize}

    \item \textbf{Alta de usuarios:} Deberán desarrollarse procedimientos formales de alta de usuarios para cada sistema, entorno o aplicación incluidos en el alcance de este documento. Estos procedimientos deberán respetar la normativa vigente en materia de control de accesos e incluirán al menos estos aspectos:

    \begin{itemize}
        \item Mecanismos de control en la asignación de permisos de acceso que permitan evitar los controles del sistema.
        \item Uso de identificadores de usuario únicos que permitan vincular a los usuarios con sus acciones y responsabilizarles de las mismas.
        \item Mecanismos para verificar la adecuación del nivel de acceso y su consistencia con la política de seguridad de la información vigente.
        \item Deberán garantizar que no se accederá al servicio hasta que se hayan completado los procedimientos de autorización.
        \item Deberán contar con el mantenimiento de un registro formalizado de todos los usuarios registrados para usar el servicio.
    \end{itemize}

    \item \textbf{Baja de usuarios:} Deberán desarrollarse procedimientos formales de baja de usuarios para cada sistema, entorno o aplicación incluidos en el alcance de \Beneficiario{}. Estos procedimientos deberán respetar la normativa vigente en materia de control de accesos e incluirán al menos los siguientes aspectos:

    \begin{itemize}
        \item La eliminación inmediata de las autorizaciones de acceso a los usuarios que dejen la Organización o cambien de puesto de trabajo.
        \item El volcado de la información del usuario a un sistema de almacenado seguro con acceso restringido al responsable de usuario.
        \item Deberán incluir la revisión periódica y eliminación de identificadores y cuentas de usuario redundantes y/o inactivas.
        \item Se deberá contar con un registro de bajas de usuarios por sistema.
    \end{itemize}

    \item \textbf{Modificación de permisos de usuarios:} Algunos de los supuestos de modificación de permisos serán: cambio de departamento, de puesto de trabajo, modificación del software o permisos de acceso asignados. Todas las demás modificaciones se tratarán como nuevas altas o bajas. También en este caso, se desarrollarán procedimientos formales de modificación de los permisos de acceso de los usuarios para cada sistema, entorno o aplicación en alcance. Estos procedimientos respetarán la normativa vigente en materia de control de accesos e incluirán al menos:

    \begin{itemize}
        \item La modificación inmediata de los permisos de acceso de los usuarios que cambien de departamento o puesto de trabajo dentro de \Beneficiario{}.
        \item La revisión periódica de los permisos y privilegios de los usuarios por sistema.
        \item El registro de las modificaciones de permisos de usuarios por sistema.
    \end{itemize}

    \item \textbf{Gestión de privilegios}

    \Beneficiario{} deberá implantar las pautas dictadas por esta normativa mediante procedimientos formales en materia de gestión de privilegios que garanticen el acceso de los usuarios a la información y sistemas para los cuales estén autorizados. Estos procedimientos deben respetar la normativa vigente en materia de control de accesos e incluirán al menos:

    \begin{itemize}
        \item La identificación de los privilegios asociados a cada elemento del sistema, por ejemplo, el sistema operativo, el sistema gestor de base de datos y cada aplicación, así como las categorías de empleados que necesiten de ellos.
        \item La asignación de privilegios a los individuos según los principios de “necesidad de su uso” y “caso por caso”. Por ejemplo, el requisito mínimo necesario para cumplir su función y solo cuando sea necesario realizarla.
        \item El mantenimiento de un proceso de autorización y un registro de todos los privilegios que se asignen, teniendo en cuenta que no se otorgarán privilegios hasta que el proceso de autorización haya concluido.
        \item Se promoverá el desarrollo y uso de rutinas del sistema para evitar la asignación de privilegios a los usuarios.
    \end{itemize}

    \item \textbf{Procedimiento de gestión de usuarios}

    \Beneficiario{} no dispone de un procedimiento de gestión de usuarios. A continuación de definen los procesos que \Beneficiario{} debe seguir para su implantación.

    \begin{itemize}
        \item \textbf{Alta usuario}

        En el momento en que el usuario precise acceso al sistema de información o se incorpore a su puesto en el área correspondiente, su responsable definirá sus necesidades tecnológicas, pudiendo incluir el alta en ciertos sistemas, así como la asignación de las herramientas de trabajo necesarias (ordenador, teléfono, etc.):

        \begin{enumerate}[label=\arabic*.]
            \item Solicitud. El Responsable del Departamento del usuario del que se solicite el alta, enviará una solicitud de alta vía correo electrónico al Responsable de Sistemas de la Organización indicándole las necesidades del nuevo Usuario, e incluirá, entre otros, los siguientes datos:
            \begin{itemize}
                \item Nombre y apellidos.
                \item NIF.
                \item Puesto de trabajo (con indicación de la denominación de éste y las unidades administrativas a las que pertenece).
                \item Ubicación (señalando edificio, planta, en su caso, sala o despacho).
                \item Tiempo que va a ocupar su puesto (indefinido o temporal).
                \item Sistemas a los que va a necesitar acceso y perfil necesario (los correspondientes permisos de acceso a recursos del sistema).
                \item Activos que va a necesitar (equipo informático, teléfono, etc.).
            \end{itemize}
            \item Firma de la normativa: El Responsable de Sistemas remitirá el Responsable del Departamento el documento en el que se relacionarán los deberes y responsabilidades de los Usuarios en relación con la información accedida (deber de confidencialidad), y al régimen disciplinario en caso de incumplimiento. Esta normativa será firmada por el Usuario y devuelta al Responsable de Sistemas para su custodia.
            \item Alta en sistemas y/o asignación de equipamiento: El Responsable de Recursos Humanos indicará al Responsable de Seguridad que proceda a su ejecución.
            \begin{itemize}
                \item En caso de que la credencial tenga una fecha de expiración, se activará la cuenta durante el tiempo que haya establecido el Responsable del Departamento (si el sistema lo permite).
                \item El Responsable de Seguridad realizará un seguimiento continuo de las fechas de expiración de las cuentas que tengan un periodo establecido.
            \end{itemize}
            \item Confirmación del alta y/o asignación de equipamiento: Una vez dado de alta el Usuario y/o proporcionado el equipamiento, el Responsable de Sistemas o el personal del equipo, enviará un correo electrónico al Responsable de Departamento que solicitó el alta, y le informará de que se ha ejecutado correctamente.
        \end{enumerate}

        \item \textbf{Baja usuario}

        El proceso a seguir se detalla a continuación:

        \begin{enumerate}[label=\arabic*.]
            \item Comunicación. El Responsable de Recursos Humanos informará al Responsable de Seguridad de la fecha de baja del usuario, al objeto de que, a partir de dicha fecha, se revoque el permiso de acceso físico a las instalaciones y lógico a los sistemas.
            \item Localización de los activos. El Responsable de Departamento indicará al Responsable de Recursos Humanos los activos de la Organización que el Usuario tiene a su disposición.
            \item Ejecución de la baja y/o retirada de activos. El Responsable de Seguridad ejecutará directamente la baja o, en su caso, indicará Administrador de cada sistema que proceda a su ejecución.
            \item Comunicación de la baja. Una vez confirmado que se ha dado de baja al Usuario y/o retirado los activos de la Organización, el Responsable de Recursos Humanos enviará la confirmación de la baja completa al Responsable del Departamento que la solicitó.
        \end{enumerate}

        \item \textbf{Modificación de permisos}

        La modificación de los derechos o permisos de acceso de un Usuario seguirá el procedimiento definido para el alta de Usuario.

        \item \textbf{Gestión y revisión del procedimiento}

        La gestión de este procedimiento corresponderá al Responsable de Seguridad de la Organización, que será competente para:

        \begin{itemize}
            \item Interpretar las dudas que puedan surgir en su aplicación.
            \item Proceder a su revisión, cuando sea necesario para actualizar su contenido.
            \item Verificar su efectividad.
        \end{itemize}

        Anualmente (o con menor periodicidad, si existen circunstancias que así lo aconsejen), el Responsable de Seguridad revisará el presente procedimiento, que se someterá, de haber modificaciones, a una nueva aprobación.

        La revisión se orientará tanto a la identificación de oportunidades de mejora en la gestión de la seguridad de la información, como a la adaptación a los cambios habidos en el marco legal, infraestructura tecnológica, organización general, etc.

        Será el Responsable de Seguridad, la persona encargada de la custodia y divulgación de la versión aprobada de este documento.

    \end{itemize}

\end{enumerate}

% !TEX root = ../Guia SGSI.tex

\subsubsection{Normativa de uso de los sistemas de información}

\begin{enumerate}[label=\alph*)]

    \item \textbf{General}

    \textbf{Regulación} Las siguientes instrucciones serán de aplicación en la observancia del cumplimiento del Reglamento (UE) 2016/679 del Parlamento Europeo y del Consejo, de 27 de abril de 2016, relativo a la protección de las personas físicas en lo que respecta al tratamiento de datos personales y a la libre circulación de estos datos, y de la Ley Orgánica de Protección de Datos y Garantía de los Derechos Digitales.

    \textbf{Obligaciones} Dado que esta normativa trata de salvaguardar un derecho fundamental mediante la adopción de diferentes medidas de seguridad, técnicas y organizativas, el Usuario deberá atender a las siguientes obligaciones:

    \tablacentrada{3.5cm}{
        \hline
        \textbf{Deber de secreto} & Guardar el necesario secreto respecto a cualquier tipo de información de carácter personal conocida en función del trabajo desarrollado, incluso una vez concluida la relación con \Beneficiario{}. \\ \hline
        \textbf{Contraseñas} &
        \begin{itemize}
            \item Las contraseñas de acceso al sistema informático son personales e intransferibles. El Usuario será el único responsable de las consecuencias derivadas de su mal uso, divulgación o pérdida.
            \item Está prohibido emplear identificadores y contraseñas de otros Usuarios para acceder al sistema informático.
            \item Los usuarios deberán utilizar contraseñas seguras.
        \end{itemize} \\ \hline
        \textbf{Bloqueo del puesto} & El Usuario deberá bloquear la sesión en caso de ausentarse temporalmente de su puesto de trabajo para evitar accesos no autorizados. \\ \hline
        \textbf{Almacenamiento de archivos} & Los ficheros de carácter personal deberán ser guardados en el espacio habilitado por la Organización para facilitar la realización de copias de seguridad y proteger el acceso frente a personas no autorizadas. \\ \hline
        \textbf{Manipulación de los archivos} & Solo las personas autorizadas podrán introducir, modificar o anular los datos personales contenidos en los ficheros. \\ \hline
        \textbf{Ficheros temporales} &
        \begin{itemize}
            \item Los ficheros temporales se borrarán una vez hayan dejado de ser necesarios para los fines que motivaron su creación.
            \item Mientras estén vigentes, deberán ser almacenados en la carpeta habilitada en la red informática o de forma que puedan ser fácilmente localizados.
        \end{itemize} \\ \hline
        \textbf{Correo electrónico} & No se utilizará el correo electrónico para el envío de información de carácter personal especialmente sensible, salvo que se adopten mecanismos necesarios para evitar que la información sea inteligible o manipulada por terceros. \\ \hline
        \textbf{Violaciones de datos personales} & Se deberán comunicar las violaciones de seguridad de datos personales de las que se tenga conocimiento, de acuerdo con el procedimiento establecido. \\ \hline
    }

    \item \textbf{Ficheros en papel}

    En relación con los ficheros en soporte o documento papel, el Usuario deberá observar las diligencias indicadas anteriormente con respecto a la confidencialidad de la información, acceso autorizado a la información en atención a las necesidades de su trabajo, gestión de soportes y documentos, trabajo fuera de las instalaciones de \Beneficiario{}.

    Asimismo, con carácter especial y únicamente de aplicación a los ficheros en papel, el Usuario deberá cumplir además con las siguientes diligencias:

    \tablacentrada{3.5cm}{
        \hline
        \textbf{Archivadores o dependencias} &
        \begin{itemize}
            \item Mantener debidamente custodiadas las llaves de acceso a locales, despachos, armarios o archivadores que contengan documentos con datos personales.
            \item Cerrar con llave la puerta del despacho al término de la jornada o en caso de ausentarse temporalmente.
        \end{itemize} \\ \hline
        \textbf{Almacenamiento de documentos} & Los documentos deberán ser archivados siguiendo los criterios establecidos por la Organización para garantizar su correcta conservación, localización y consulta. \\ \hline
        \textbf{Custodia de documentos} & Los documentos en soporte papel deberán ser custodiados para impedir accesos no autorizados, especialmente fuera de la jornada laboral. \\ \hline
        \textbf{Traslado} & Durante el traslado de documentos, se deberán adoptar medidas para impedir el acceso o manipulación por terceros. \\ \hline
        \textbf{Destrucción} & Los documentos en papel que contengan datos personales deberán ser destruidos mediante destructora de papel u otro medio que garantice la eliminación segura. \\ \hline
        \textbf{Violaciones de seguridad} & Se deberán comunicar las violaciones de seguridad relacionadas con los ficheros en papel al Responsable de Seguridad. \\ \hline
    }

    \item \textbf{Plazos de retención}

    La información será mantenida durante los plazos que legal o contractualmente sean aplicables, así como los plazos necesarios para responder ante auditorías.

    Cuando por cualquier motivo el tratamiento por \Beneficiario{} de datos de carácter personal debiera ser cancelado, los datos en cuestión serán bloqueados durante los plazos de retención que legal o contractualmente les fueran aplicables. Transcurridos estos plazos, no siendo posible la destrucción de los datos que hayan de ser cancelados, bien sea por imposibilidad técnica o por razón del procedimiento o soporte empleados, el acceso a los datos de carácter personal deberá ser bloqueado permanentemente, hasta que su destrucción fuera posible.

    Ejemplos de plazos de retención:

    \tablacentrada{3.5cm}{
        \hline
        \celdaTitulo{Datos} & \celdaTitulo{Plazo de retención} \\ \hline
        Información de empleados & 4 años \\ \hline
        Cámaras de seguridad & 30 días \\ \hline
        Control de accesos de personas físicas & 30 días \\ \hline
        Datos de clientes & Según lo definido en el contrato con cada cliente \\ \hline
    }

    Se puede ver con más detalle en el Registro de Actividades de Tratamiento (RAT).

    \begin{itemize}
        \item Las copias de seguridad se realizarán por el Departamento de Sistemas. Se deberá evitar la realización de copias de la información por parte de personas no autorizadas.
        \item Si se almacena localmente información en los dispositivos móviles, se comunicará al Departamento de Sistemas, y se seguirán sus instrucciones para la realización de copias de seguridad.
        \item Si se requiere realizar copias por el usuario se adoptarán las medidas adecuadas para la protección de dichas copias.
        \item No se realizarán copias en dispositivos o sistemas de información para uso privado.
    \end{itemize}

\end{enumerate}


% !TEX root = ../Guia SGSI.tex

\subsubsection{Normativa de acceso a internet}

Con carácter general, los usuarios de \Beneficiario{} disponen de acceso a Internet como herramienta de productividad y conocimiento, así como de mejora de los sistemas de trabajo y búsqueda de información. Esta herramienta es propiedad de la Organización, la cual se reserva el derecho de conceder o anular dichos accesos conforme a los criterios que crea convenientes.

Será necesario garantizar un uso adecuado de los recursos informáticos de acceso a Internet, por los siguientes motivos:

\begin{itemize}
    \item Seguridad: debido al riesgo de infección por software maligno.
    \item Volumen del tráfico externo de datos: garantizando que el acceso a contenidos necesarios para la actividad profesional no se vea perjudicado por el tráfico generado por contenidos no vinculados con las competencias de \Beneficiario{}.
    \item Volumen del tráfico interno de datos: como consecuencia de contenidos descargados de la Web y su posterior almacenamiento. Esta situación aconseja también regular el tipo de ficheros cuya descarga y almacenamiento está permitido.
    \item Ética: es ineludible el compromiso que la Organización debe mantener con la sociedad, a la hora de vetar el acceso a contenidos que pudieran ser poco éticos, ofensivos o delictivos.
\end{itemize}

\begin{enumerate}[label=\alph*)]

\item \textbf{Limitaciones}

\begin{table}[H]
    \centering
    \small
    \tablacentrada{3.5cm}{
        \hline
        \textbf{Responsabilidad} &
        \begin{itemize}
        \item Internet es un servicio que \Beneficiario{} pone a disposición de su personal para uso estrictamente profesional.
        \item Los usuarios serán los únicos responsables de las sesiones iniciadas en Internet desde sus terminales de trabajo, y se comprometen a acatar las reglas y normas de funcionamiento establecidas en la presente Normativa.
        \item El acceso a Internet por personal externo requerirá la previa autorización por escrito de la Dirección.
        \end{itemize} \\ \hline
        \textbf{Monitorización} &
        \begin{itemize}
        \item La Organización se reservará el derecho a filtrar el contenido al que el usuario puede acceder a través de Internet desde los recursos y servicios propiedad de la Organización, así como a monitorizar y registrar los accesos realizados desde los mismos.
        \item En caso de que un usuario considere necesario acceder a alguna dirección incluida en una de las categorías filtradas, se pondrá en contacto con su responsable directo para que éste gestione el acceso correspondiente.
        \end{itemize} \\ \hline
    }
    \caption{Limitaciones del acceso a Internet}
    \label{tab:guia-sgsi-limitaciones-acceso-internet}
    \normalsize
\end{table}

\begin{table}[H]
    \centering
    \small
    \tablacentrada{3.5cm}{
        \hline
        \textbf{Usos no permitidos que implican un riesgo para la seguridad} &
        \begin{itemize}
        \item En ningún caso se modificarán las configuraciones de los navegadores (opciones de Internet) de los equipos ni la activación de servidores o puertos sin la autorización expresa. Todos los equipos que así lo estima la empresa, ya estarán configurados para su acceso a Internet.
        \item Se prohibirá expresamente el acceso, la descarga y/o el almacenamiento en cualquier soporte, de páginas con contenidos ilegales, dañinos, inadecuados o que atenten contra la moral y las buenas costumbres y, en general, de todo tipo de contenidos que incumplan las normas éticas y de cortesía de la Organización.
        \item No se permitirá el almacenamiento en los equipos de archivos y contenidos personales descargados vía Internet, especialmente aquellos que violen la legislación vigente relativa a Propiedad Intelectual. Los usuarios deberán respetar y dar cumplimiento a las disposiciones legales de derechos de autor, marcas registradas y derechos de propiedad intelectual de cualquier información visualizada u obtenida mediante Internet haciendo uso de los recursos informáticos o de red de la Organización.
        \item Estará prohibido el uso de Internet mediante los recursos informáticos o de red de la empresa con fines recreativos, así como para obtener o distribuir material violento o pornográfico, o para obtener o distribuir material incompatible con los valores de la Organización.
        \item El uso de chats o programas de conversación en tiempo real estarán restringidos a aquellos que la Organización haya autorizado expresamente.
        \item No se permitirá la descarga de software ejecutable desde internet.
        \end{itemize}
        \\ \hline
        \textbf{Incidencias} &
        \begin{itemize}
        \item Cualquier incidente de seguridad relacionado con la navegación por Internet, deberá ser comunicado sin demora al Departamento de Sistemas.
        \end{itemize} \\ \hline
    }
    \caption{Usos no permitidos que implican un riesgo para la seguridad}
    \label{tab:guia-sgsi-usos-no-permitidos-acceso-internet}
    \normalsize
\end{table}

\item \textbf{Normativa para trabajar fuera de las instalaciones}

El trabajo fuera de las instalaciones de \Beneficiario{} comprende tanto el teletrabajo habitual y permanente de los usuarios desplazados, como el trabajo ocasional, usando en ambos casos, dispositivos de computación y comunicación (usualmente: ordenador portátil, Tablet, teléfono móvil, etc.). Este modo de trabajo comprenderá también las conexiones remotas realizadas desde congresos o sesiones de formación, alojamientos o situaciones similares.

Esta modalidad conlleva el riesgo de trabajar en lugares desprotegidos, esto es, sin las barreras de seguridad físicas y lógicas implementadas en las instalaciones de la entidad. Fuera de este perímetro de seguridad aumentan las vulnerabilidades y la probabilidad de materialización de las amenazas, lo que hace necesario adoptar medidas de seguridad adicionales.

Se incluyen, seguidamente, un conjunto de normas de obligado cumplimiento, que tienen como objetivo el reducir el riesgo, que complementan lo ya recogido en la Normativa de uso de los sistemas de información.

\begin{enumerate}[label=\arabic*)]
    \item Uso personal y profesional: Los dispositivos sólo pueden utilizarse para fines profesionales.
    \item Copias de seguridad: Sigue las instrucciones del departamento de informática.
    \item Cierre de sesión: También en casa debes cerrar las sesiones abiertas con tu organización cuando finalices el trabajo.
    \item Doble Factor de Autenticación: Habilítalo para no depender únicamente de la seguridad de tu contraseña.
    \item Normativa interna: Sigue la normativa de seguridad existente.
    \item Prevención Anti-malware: No instales software desde direcciones no seguras, activa el firewall y mantén el antivirus actualizado.
    \item Uso de canales de comunicación establecidos: Utiliza sólo los canales establecidos y autorizados.
    \item Correo electrónico: Extrema tu cautela cuando recibas correos electrónicos no solicitados.
    \item Mantenimiento de los equipos: Cuida de tu equipo y mantenlo actualizado.
    \item Gestión de incidencias: Comunica incidencias según el procedimiento habitual.
\end{enumerate}

\end{enumerate}

% !TEX root = ../Guia SGSI.tex

\subsubsection{Uso personal y profesional de los dispositivos}

\begin{itemize}
    \item Se requerirá autorización para la salida del dispositivo fuera de las instalaciones de \Beneficiario{}.
    \item No podrán prestarse a terceros salvo autorización expresa, que incluirá en todo caso la definición de las condiciones de uso.
    \item Su uso estará restringido a las actividades que por su función estén soportadas por el uso del dispositivo asignado.
    \item Estará prohibido el uso para otras finalidades de carácter personal.
    \item La utilización quedará restringida al empleado al que se le ha asignado el dispositivo.
\end{itemize}


% !TEX root = ../Guia SGSI.tex

\subsubsection{Uso de los canales de comunicación establecidos}

\begin{itemize}
    \item En las conexiones remotas a los sistemas de \Beneficiario{}, se utilizará el servicio VPN y seguirán las directrices establecidas por Dirección.
    \item Se usarán preferentemente redes 3G o 4G antes que redes WIFI inseguras.
    \item Se desconfiará de aquellas redes wifi que no tengan contraseña y, si la tienen, de aquellas cuya contraseña no sea lo suficientemente segura (1234, el nombre de la red, número de habitación…).
    \item Se recomienda desconectar wifi y bluetooth cuando no sean necesarios.
\end{itemize}


% !TEX root = ../Guia SGSI.tex

\subsubsection{Normativa de protección de correo electrónico, servidores y endpoints}

\begin{enumerate}[label=\alph*)]

\item \textbf{Normativa de uso del correo electrónico}

\textbf{Concepto}. El correo electrónico (e-mail) es un servicio de red para permitir a los usuarios de la Organización enviar y recibir mensajes. Junto con los mensajes también pueden ser enviados ficheros adjuntos.

\textbf{Caracteres}. Las características peculiares de este medio de comunicación (universalidad, bajo coste, anonimato, etc.) han propiciado la aparición de amenazas que utilizan el correo electrónico para propagarse o que aprovechan sus vulnerabilidades.

\textbf{Especificaciones}. \Beneficiario{}, consciente de los problemas de seguridad y responsabilidad legal que ocasiona el uso del correo electrónico, dispondrá de las siguientes especificaciones:

\begin{itemize}
    \item Responsabilidad
    \begin{itemize}
        \item Los usuarios serán responsables de todas las actividades realizadas con las cuentas de acceso y su respectivo buzón de correos provistos por la Organización.
        \item Los usuarios deberán ser conscientes de los riesgos que acarrea el uso indebido de las direcciones de correo electrónico suministradas por la Organización.
        \item Las cuentas de correo son personales e intransferibles. Salvo en casos puntuales para los que deberá solicitarse y obtenerse la correspondiente autorización, no se debe ceder el uso de la cuenta de correo a terceras personas, lo que podría provocar una suplantación de identidad y el acceso a información confidencial.
        \item Los mensajes de correo transmiten información en sus cabeceras (en principio ocultas) que indican datos adicionales del emisor, por lo que deben tenerse en cuenta posibles repercusiones (como daños a la imagen institucional) que podría acarrear una mala utilización de este recurso.
    \end{itemize}
    \item Uso aceptable
    \begin{itemize}
        \item Como norma general no se utilizará la herramienta de correo electrónico con fines ajenos al propio desarrollo de las actividades que cada usuario tiene encomendadas en la Organización.
        \item La utilización del correo electrónico por personal externo requerirá la previa autorización por escrito de la Dirección.
        \item La forma y contenidos de los correos enviados por el usuario estarán alineados con las normas éticas y de cortesía marcadas por la Organización, y en ningún caso se enviarán correos ofensivos, amenazantes o de mal gusto.
        \item El usuario deberá mantener ordenados y clasificados todos sus buzones y carpetas. Los correos inservibles deberán ser eliminados, y todos los archivos adjuntos almacenados en el equipo o unidad de disco habilitada.
    \end{itemize}
    \item Usos que no serán permitidos y que implican un riesgo para la seguridad
    \begin{itemize}
        \item La instalación y uso de cualquier otra aplicación de correo electrónico, así como de una versión diferente de la aplicación homologada que no haya sido autorizada e instalada por el personal técnico autorizado.
        \item La difusión de contenido ilegal; como por ejemplo amenazas, código malicioso, apología del terrorismo, pornografía infantil, software pirata, o de cualquier otra naturaleza delictiva.
        \item El uso no autorizado de servidores propiedad de la Organización para el envío de correo personal.
        \item El envío masivo de correos publicitarios o de cualquier otro tipo que no guarde relación alguna con las necesidades de negocio de la Organización. Este hecho, además, puede llegar a ser interpretado como “spamming”.
        \item La divulgación, independientemente del formato en que se encuentren, de correos que revelen datos del directorio o de usuarios pertenecientes a la Organización, fuera de los límites laborales establecidos por la misma.
        \item En el caso de se requiera enviar un mensaje de correo electrónico a varios destinatarios, se utilizará preferentemente el campo CCO (copia oculta) para introducir las direcciones de correo de los destinatarios, con excepción de aquellos mensajes en los que necesariamente se requiera la identificación de todos los destinatarios para confirmar que han sido informados.
    \end{itemize}
    \item Diligencia
    \begin{itemize}
        \item Los archivos adjuntos recibidos serán analizados por las herramientas antivirus antes de ser abiertos o ejecutados. Los correos sospechosos o de dudosa procedencia no serán abiertos, y menos aún los archivos adjuntos que contengan. Su eliminación debe ser inmediata. Gran parte del código malicioso suele insertarse en ficheros adjuntos, ya sea en forma de ejecutables (.exe, por ejemplo) o en forma de macros de aplicaciones (Word, Excel, etc.).
        \item No se empleará el correo electrónico como medio de comunicación para enviar o recibir información confidencial o que contenga datos de carácter personal. Únicamente, y en aquellos casos en los que sea estrictamente necesario, se utilizará este medio, en cuyo caso, se enviará con las medidas de seguridad apropiadas para cada tipo concreto de información mediante la utilización de un software de cifrado, previa autorización expresa del Responsable de Seguridad.
        \item En la medida de lo posible, se desactivará la vista previa. Utilizar la vista previa para los correos de la bandeja de entrada comporta los mismos riesgos que abrirlos. Del mismo modo, limitar el uso de HTML. El código malicioso puede encontrarse fusionado con el código HTML del mensaje. Desactivar la visualización HTML de los mensajes ayuda a evitar que el código malicioso se ejecute.
        \item Los navegadores utilizados para acceder al correo vía web deberán estar permanentemente actualizados a su última versión, al menos en cuanto a parches de seguridad, así como correctamente configurados.
        \item Una vez finalizada la sesión web, será obligatoria la desconexión con el servidor mediante un proceso que elimine la posibilidad de reutilización de la sesión cerrada.
        \item Se desactivarán las características de recordar contraseñas para el navegador.
        \item Se activará la opción de borrado automático al cierre del navegador, de la información sensible registrada por el mismo: histórico de navegación, descargas, formularios, caché, cookies, contraseñas, sesiones autenticadas, etc.
    \end{itemize}
    \item Incidencias
    Los usuarios deberán comunicar a sus responsables directos sobre cualquier anomalía que detecten en su correo, así como de la apertura de un correo sospechoso o cualquier alerta generada por el antivirus.
    \item Monitorización
    La Organización se reserva el derecho a revisar los ficheros LOG de los servidores, con el fin de comprobar el cumplimiento de estas normas y prevenir actividades que puedan afectar a la Organización como responsable civil subsidiario.
    \item Phishing
    \begin{itemize}
        \item Si se recibe un correo electrónico sospechoso se deberá comprobar la veracidad del remitente por otras vías (teléfono, mensajería instantánea), por ejemplo, comprobar el dominio del correo del remitente y que su nombre coincide con su cuenta de correo electrónico (nombre y dominio).
        \item Se deberá revisar el contenido del mensaje y desconfiar de él si está mal redactado o contiene faltas de ortografía.
        \item Si el correo incluye un enlace sospechoso, no se deberá abrir. Habrá que fijarse en la URL. En los casos de phishing, la URL no coincide con la de la organización que están intentando suplantar, aunque la apariencia de la web sea similar.
        \item Si el correo incluye ficheros adjuntos, se deben verificar antes de abrir.
        \item No se responderá nunca a un email sospechoso. Además, no se reenviará el correo a personas de nuestro entorno, viniendo de nosotros podrían confiar en él y caer en la trampa.
    \end{itemize}
\end{itemize}

\item \textbf{Seguridad en los dispositivos}

\Beneficiario{} facilita a sus usuarios el equipamiento informático necesario para la realización de las tareas relacionadas con su puesto de trabajo.

\textbf{Propiedad de los recursos}. Este equipamiento será propiedad de la Organización y por tanto no estará destinado a un uso personal. Como consecuencia de esto, \Beneficiario{} se reservará el derecho de revisar, sin previo aviso, los equipos, el uso de Internet y el teléfono corporativo que esté haciendo cada usuario y, en caso de que existieran indicios de que se está llevando a cabo una utilización indebida. De esta forma el usuario quedará informado de que el resultado de los controles efectuados puede ser utilizado para llevar a cabo, en su caso, las actuaciones disciplinarias previstas por la normativa vigente.

\textbf{Obligaciones de los usuarios}. Los Usuarios deberán cumplir las siguientes medidas de seguridad para el uso de los ordenadores personales:

\begin{itemize}
    \item Conexión de otros dispositivos
    \begin{itemize}
        \item No estará permitido conectar dispositivos que no estén autorizados a la red de la Organización.
        \item Tampoco se podrán conectar a los dispositivos autorizados, otros dispositivos que no estén autorizados expresamente.
    \end{itemize}
    \item Ubicación del dispositivo
    No estará permitido variar la ubicación física de los dispositivos asignados a una ubicación.
    \item Configuración del dispositivo
    No estará permitido alterar la configuración física, configuración de seguridad ni el software de los dispositivos.
    \item Uso de dispositivos y de la red
    Los dispositivos, así como la red de información que \Beneficiario{} ponga a disposición de los usuarios estarán destinados a permitir el desempeño de las funciones y tareas profesionales que estos tienen encomendadas, estando prohibido el uso para finalidades de carácter personal, o bien para realizar actos desleales o que pudieran ser considerados ilícitos.
    \item Antivirus
    El Usuario deberá comprobar que su antivirus se actualiza con regularidad. En caso contrario deberá notificarlo como una incidencia de seguridad.
    \item Uso de la información
    \begin{itemize}
        \item Estará prohibido utilizar, copiar o transmitir información contenida en los sistemas informáticos para uso privado.
        \item El Usuario se abstendrá de copiar la información contenida en los ficheros en los que se almacenen datos de carácter personal u otro tipo de información de este Organismo en ordenador propio, pendrives o a cualquier otro soporte informático, salvo que solicite autorización al Responsable de Seguridad, y se adopten las medidas de seguridad correspondiente. Asimismo, los datos contenidos en este tipo de soportes deberán ser suprimidos una vez hayan dejado de ser útiles y pertinentes para la satisfacción de los fines que motivaron su creación. Durante el periodo de tiempo que los ficheros o archivos permanezcan en el equipo o soporte informático externo, deberá restringir el acceso y uso de la información que obra en los mismos.
        \item El Usuario deberá restringir a terceros (familiares, amistades o cualesquiera otros) el acceso a los archivos o ficheros titularidad de la Organización y dispuesto a razón única de las funciones o tareas desempeñadas en la misma. Se establecerán medidas de protección adicionales que aseguren la confidencialidad de la información almacenada en el equipo se almacenen datos de carácter personal.
        \item Se prohibirá todo uso de programas de compartición de contenidos, habitualmente utilizados para la descarga de archivos de música, vídeo, etc. que no estén permitidos por la empresa.
    \end{itemize}
    \item Identificación y autenticación
    Las contraseñas de acceso al equipo, sistema y/o a la red, concedidos por \Beneficiario{} serán personales e intransferibles, y será el Usuario el único responsable de las consecuencias que pudieran derivarse de su mal uso, divulgación o pérdida.
    Por cuestiones de seguridad no estarán permitidas prácticas como:
    \begin{itemize}
        \item Emplear identificadores y contraseñas de otros Usuarios para acceder al sistema y a la red de la Organización.
        \item Intentar modificar o acceder al registro de accesos.
        \item Burlar las medidas de seguridad establecidas en el sistema informático, intentando acceder a ficheros.
    \end{itemize}
    \item Incidencias con los dispositivos o accesos
    Cuando se considere que el acceso se ha visto comprometido se deberá comunicar al responsable correspondiente.
    \item Bloqueo del puesto de trabajo
    Al abandonar el puesto de trabajo deberán cerrarse las sesiones con las aplicaciones establecidas, se habilitará el protector de pantalla con bloqueo con contraseña, y se apagarán los equipos al finalizar la jornada laboral, excepto en los casos en que el equipo deba permanecer encendido.
\end{itemize}

\item \textbf{Mantenimiento de los endpoints}

\begin{itemize}
\item Se utilizarán los equipos siempre de acuerdo con las especificaciones indicadas por el área de informática o, en su caso, del fabricante.
\item Se deberán mantener los equipos en buen estado de conservación.
\item Se evitará el uso en condiciones de temperatura o humedad inadecuadas, o en entornos que lo desaconsejen (mesas con alimentos y líquidos, entornos sucios, etc.).
\item Se transportarán de manera segura los equipos, evitando proporcionar información sobre el contenido de estos.
\item Se deberán realizar las actualizaciones en los equipos siguiendo las instrucciones que se reciban.
\end{itemize}

\item \textbf{Prevención antimalware}

\begin{itemize}
\item En caso de disponer de autorización del Responsable de Sistemas para instalar software, se realizará desde direcciones seguras.
\item Se deberá mantener el antivirus actualizado y disponer de una correcta configuración del firewall.
\item No se descargarán ficheros adjuntos o imágenes cuya fuente no haya sido comprobada o no sea fiable, ya que pueden contener malware.
\item Si se sospecha que un dispositivo ha sido infectado por un virus u otro software malicioso, se dejará de procesar la información, y se comunicará con urgencia a Departamento de Sistemas.
\item No se activarán opciones de trabajo de sesión abierta en un sitio web.
\item Se eliminará la posibilidad de reutilización de la sesión cerrada.
\item No se eliminará la configuración definida de bloqueo automático de sesión, ni de bloqueo de pantalla.
\end{itemize}

\end{enumerate}

% !TEX root = ../Guia SGSI.tex

\subsubsection{Seguridad de los servicios de red}

Las conexiones no seguras a los servicios de red pueden afectar a la seguridad de \Beneficiario{}; por lo tanto, se controla el acceso a los servicios de red tanto internos como externos. Esto es necesario para garantizar que los usuarios que tengan acceso a las redes y a sus servicios no comprometan la seguridad de estos.

Para llevar a cabo tales asignaciones de acceso a los servicios y recursos, el Responsable de Seguridad:

\begin{itemize}
    \item Identificará las redes y servicios de red a los cuales se permite el acceso para cada usuario del sistema.
    \item Solicitará autorización para la asignación de accesos.
    \item Monitorizará los accesos y el uso de los servicios de red.
    \item Revisará, al menos trimestralmente, los derechos de acceso a redes y servicios de red de los usuarios.
\end{itemize}

\begin{enumerate}[label=\alph*)]

\item \textbf{Controles de red}

La capacidad de conexión (niveles de acceso) de los usuarios a las redes de la Organización se encontrará limitada a través de las correspondientes configuraciones de los sistemas de información:

\begin{itemize}
    \item Normativa interna.
    \item Política de contraseñas.
    \item Control de acceso basado en roles (RBAC): \Beneficiario{} debe implantar un control de acceso basado en roles (RBAC) para asegurar que los usuarios solo tengan acceso a la información y los recursos necesarios para realizar sus tareas.
    \item Encriptación de las comunicaciones: La encriptación puede utilizarse para proteger las comunicaciones de la interceptación no autorizada.
\end{itemize}

Los técnicos de sistemas de \Beneficiario{} no establecen medidas técnicas para restringir la capacidad en las conexiones de los usuarios a las redes:

\begin{itemize}
    \item Segmentación de la red: \Beneficiario{} no divide la red en segmentos más pequeños a través de distintas VLAN para limitar el acceso de un usuario solo a la parte de la red que necesita para su trabajo.
    \begin{itemize}
        \item En las redes compartidas, especialmente aquellas que se extienden fuera de los límites de la Organización, se incorporan controles de enrutamiento (enrutamiento estático, Autenticación de protocolos de enrutamiento, NAT (Traducción de Direcciones de Red), Monitorización de redes, Protección inalámbrica) para asegurar que las conexiones informáticas y los flujos de información no violen la normativa de control de acceso.
        \item El Departamento de Sistemas de \Beneficiario{} no revisa periódicamente la correcta configuración de los controles de enrutamiento, requiriendo para ello la información que los técnicos de sistemas obtengan directamente de los ficheros de configuración de los sistemas de comunicaciones.
    \end{itemize}
\end{itemize}

\Beneficiario{} debe implementar una segmentación de su red corporativa mediante VLAN para limitar el acceso de los usuarios a los recursos estrictamente necesarios. Del mismo modo, debe realizarse una revisión periódica de los controles de enrutamiento para reforzar la privacidad de sus datos y limitar la propagación de potenciales ataques.

Los equipos integrados en las redes de \Beneficiario{} están identificados de la siguiente manera:

\begin{itemize}
    \item Hostname y número de serie.
    \item Dirección IP que lo identifica de forma única en el sistema.
\end{itemize}

\item \textbf{Diagnóstico remoto y protección de los puertos de configuración}

Los sistemas de comunicación que requieran de monitorización por parte de los técnicos de sistemas se configurarán habilitando los puertos de diagnóstico y restringiendo su acceso.

\item \textbf{Segregación de redes}

\Beneficiario{} debe implementar una política de uso de VLAN para proteger su red corporativa restringiendo los accesos necesarios por área de trabajo para mitigar la propagación de ataques potenciales y accesos no autorizados.

El Responsable de Seguridad se encargará de que existan mapas de red adecuados y detallados para cada una de las redes existentes, así como los sistemas de comunicaciones que las conforman. Estos mapas deberán ser revisados y actualizados convenientemente.

En ellos, se incluirán las segmentaciones efectuadas respecto de los dominios existentes, los perímetros de seguridad lógicos controlados por cortafuegos o salvaguardas análogas.

Los usuarios, administradores y el resto del personal de \Beneficiario{} deberán conocer los accesos estrictos para el desempeño de sus funciones, evitando accesos no autorizados o la posibilidad de estos, con especial importancia en redes de comunicación vulnerables o comprometidas como las redes inalámbricas.

Cualquier cambio en el diseño de la actual arquitectura de segregación de redes dentro de \Beneficiario{} será autorizado por el Responsable de Seguridad, el cual, llegado el caso, evaluará la información técnica relacionada y procederá a la apertura del correspondiente Plan de Mejora.

\item \textbf{Medidas básicas de seguridad para la protección de los servidores de correo electrónico}

De forma general, se deberán aplicar las siguientes medidas de seguridad:

\begin{itemize}
    \item Políticas de contraseñas seguras, siguiendo las directrices marcadas por la política de contraseñas implantada en la organización.
    \item \Beneficiario{} emplea el protocolo TLS, preferiblemente la versión más actual (TLS 1.3), para cifrar los datos durante el proceso de transmisión de información entre servidor y cliente, de forma que se proteja la información sensible de la interceptación malintencionada y se garantice la confidencialidad e integridad de las comunicaciones por correo electrónico de \Beneficiario{}.
    \item \Beneficiario{} no aplica el cifrado de extremo a extremo como medida de seguridad avanzada para garantizar que sólo el destinatario previsto pueda descifrar el contenido. \Beneficiario{} debe implantar esta capa adicional de protección en especial cuando se transmiten datos sensibles o confidenciales por correo electrónico.
    \item Configuración avanzada para evitar los ataques phishing:
    \begin{itemize}
        \item \Beneficiario{} debe implantar un registro SPF que contenga las direcciones IP de la Organización para protegerse contra suplantaciones de identidad y campañas de phishing.
        \item \Beneficiario{} no aplica DKIM para asegurar que los correos provienen de un dominio autorizado y no han sido alterados para protegerse contra potenciales ataques de phishing.
        \item \Beneficiario{} no dispone de políticas para redirigir los correos que no superen esta verificación a través de DMARC (Domain-based Message Authentication, Reporting, and Conformance). \Beneficiario{} debe implementar un sistema de redirección de correos no verificados para proteger la reputación de su dominio previniendo campañas de phishing y suplantaciones de identidad.
    \end{itemize}
\end{itemize}

En una situación ideal, el servidor de envío adjunta una firma DKIM en el correo electrónico. El servidor de la Organización recupera el registro SPF para verificar la autorización del servidor de envío y comprueba la firma DKIM para asegurar la integridad del email. Si ambas comprobaciones son superadas, el email se considera legítimo. Si alguna comprobación falla, el servidor a través de la política DMARC decide si entregar, poner en cuarentena o rechazar el email.

\item \textbf{Medidas preventivas de protección de la infraestructura de servidores y endpoints}

Se deberán poner en práctica una serie de medidas preventivas que garanticen al menos una mínima defensa de la infraestructura, los servidores y los endpoints.

Para ello, se deberán tener en cuenta las siguientes medidas de seguridad:

\begin{itemize}
    \item Actualizaciones periódicas de software y parches de seguridad de los servidores de forma periódica para reducir la exposición de vulnerabilidades conocidas. \Beneficiario{} debe implantar un procedimiento de actualizaciones que garantice que la infraestructura que alberga el sistema esté protegida de las amenazas conocidas.
    \item Configuración y mantenimiento periódico del tráfico de red entrante y saliente. Un filtrado bien configurado y mantenido impide accesos no autorizados entre la red interna y las amenazas externas. Se deben realizar revisiones periódicas de las reglas del cortafuegos manteniendo al día los cambios existentes en el entorno de trabajo y teniendo en cuenta nuevas tendencias de seguridad que puedan surgir.
    \item Supervisión y gestión: Se desplegarán soluciones de supervisión y gestión que funcionen con otras herramientas para la actualización y que proporcionen monitorización, acceso y control de forma directa y remota para gestionar código, actualizaciones, contraseñas, etc.
    \item Zero trust (Política de confianza cero): Concepto relacionado con la configuración e instalación de dispositivos bajo la premisa y el principio del mínimo privilegio, es decir, no confiar en nada e ir añadiendo solo aplicaciones/equipos/direcciones fiables y autorizadas bajo autorización expresa del Responsable de Seguridad, siendo esta una política proactiva encaminada a la protección de los dispositivos que reduzca al máximo los posibles errores humanos producidos.
\end{itemize}

\item \textbf{Soportes físicos en tránsito}

El transporte de activos de información deberá efectuarse conforme a unos mínimos exigibles de seguridad que aplicarán a todos los soportes físicos en tránsito. Así, \Beneficiario{} asegurará que dichos soportes y por ello, dicha información, llega a su destino cumpliendo con los estándares de seguridad, es decir, manteniendo intacta su confidencialidad, disponibilidad e integridad.

\begin{itemize}
    \item El Responsable de Seguridad debe ser el encargado de aplicar las medidas de seguridad oportunas a cada soporte físico en tránsito, dependiendo del nivel de seguridad aplicable a los datos o a la información que se transporte.
    \item \Beneficiario{} debe establecer una política de cifrado en sus dispositivos físicos para garantizar la confidencialidad de la información en caso de robo o pérdida.
\end{itemize}

\item \textbf{Mensajería electrónica}

Con el fin de mantener la protección sobre los mensajes de tipo electrónico, \Beneficiario{} asegurará la confidencialidad de su contenido en el tránsito de la información, estableciendo controles y sistemas que eviten el acceso o consulta no autorizados a dichos activos o la realización de actividades fraudulentas con ellos.

Con el fin de mantener la protección sobre los activos participantes en las operaciones comerciales de tipo electrónico, la organización deberá asegurar la confidencialidad de su contenido en el tránsito de la información a través de redes públicas, estableciendo controles y sistemas que eviten el acceso o consulta no autorizados a dichos activos o la realización de actividades fraudulentas con ellos.

A consecuencia de esto, \Beneficiario{} salvaguardará dicha confidencialidad mediante las siguientes medidas:

\begin{itemize}
    \item Determinar el nivel de seguridad aplicable a cada activo que interviene en las operaciones de comercio electrónico a fin de elevar o disminuir el nivel de protección conforme a dicha clasificación.
    \item \Beneficiario{} debe implantar una política de encriptación de su documentación sensible para protegerla en caso de pérdida o interceptación por parte de potenciales atacantes.
    \item \Beneficiario{} debe implantar una política de acuerdos de confidencialidad para garantizar los estándares de seguridad de su información corporativa por parte de terceros.
    \item Cerciorarse de la correcta identificación del destinatario objetivo de la información mediante sistemas que identifiquen unívocamente a emisor y receptor (certificados digitales, firma electrónica, etc.)
    \item Obtención de la autorización previa antes de utilizar sistemas externos como de mensajería electrónica o aplicaciones similares que puedan utilizarse en la organización, estableciendo niveles de acceso y de autenticación.
\end{itemize}

\item \textbf{Acuerdos de intercambio}

\Beneficiario{} deberá realizar contratos o acuerdos de intercambio para sellar el proceso de transferencia de información de una entidad a otra. En los mismos se deberá reflejar el compromiso de ambas partes con la seguridad, asegurando la protección de la información en todo momento, así como las responsabilidades adquiridas por cada una de las partes firmantes de dicho acuerdo.

A su vez, dicho acuerdo respetará la legislación vigente que aplica a dicho intercambio de información, como pueden ser la protección de datos, el copyright o las licencias de software, por poner un ejemplo. Otras medidas complementarias que se podrán implementar son:

\begin{itemize}
    \item Los responsables del envío, recepción y custodia de la información.
    \item Sistemas para asegurar la trazabilidad y no repudio (confirmación de lectura, acuse de recibo).
    \item Políticas de envío de paquetes / documentación según criticidad, acuerdo de envíos y depósitos, así como transportistas autorizados.
    \item Responsabilidades en caso de pérdida de datos.
    \item Sistema de etiquetado de información para datos críticos.
    \item Sistemas de grabación de datos y software utilizados, junto con los controles adicionales y criptográficos a utilizar.
    \item Control de niveles de acceso, así como mantenimiento de la cadena de custodia en la información en tránsito.
\end{itemize}

\item \textbf{Acuerdos de confidencialidad o no revelación}

A fin de preservar la seguridad de la información manejada en la organización, \Beneficiario{} se asegura de que sus empleados se comprometan con dicha seguridad antes incluso del inicio de la actividad laboral. Por ello, la organización a la hora de incorporar personal a la misma se encargará de que la redacción y firma del contrato laboral incluya el establecimiento de firma de un acuerdo de compromiso con la seguridad de la información con el trabajador como medida preventiva de seguridad antes del desempeño de sus funciones.

Al finalizar la relación laboral con un empleado, \Beneficiario{} revisa los casos en los que se hubieran definido cláusulas de confidencialidad o acuerdos contractuales, que contemplaran relaciones sujetas a confidencialidad posterior a la finalización de la contratación.

En estos casos, de detectarse incumplimiento por parte del extrabajador o empleado de los acuerdos establecidos que pudieran comprometer la seguridad de la información de la organización, el Responsable de Seguridad lo pondrá en conocimiento de la Dirección con la finalizar de incoar el proceso disciplinario aplicable.

Estos acuerdos también deberán ser de aplicación a terceros en caso de que puedan acceder a información de \Beneficiario{} debido a relaciones contractuales o ejecuciones de actividades y/o proyectos, para lo cual deberá:

\begin{itemize}
    \item Identificarse la información objeto de protección.
    \item Definición del tiempo del acceso o del acuerdo y los casos que este compromiso deba ser indefinido.
    \item Firma del acuerdo por responsables autorizados.
    \item Definición de la propiedad de la información para establecer controles y limitaciones (propiedad intelectual o derechos de la información).
    \item Derechos para auditar y monitorizar dichos procesos de tránsito de información o el uso de información confidencial por parte de las partes.
    \item Sistemas de notificación de incidencias o debilidades en el uso de la información, así como las acciones a llevar a cabo en caso de producirse un problema.
    \item Medidas a implementar de devolución o destrucción de la información en caso de finalización de la colaboración o motivos que permitan la suspensión del acuerdo o contrato.
\end{itemize}

\end{enumerate}

% !TEX root = ../Guia SGSI.tex

\subsubsection{Normativa de uso de redes sociales}

\begin{enumerate}[label=\alph*)]

\item \textbf{Principios}

\Beneficiario{} promueve la participación en redes sociales y comunidades digitales, teniendo en cuenta los siguientes principios:
\begin{itemize}
    \item Legalidad: en las publicaciones que se realicen se tendrá en cuenta las normas nacionales, internacionales, las normas internas aprobadas por la entidad y en especial la normativa de propiedad intelectual, protección de datos personales y la Declaración Universal de Derechos Humanos.
    \item Respeto: se respetará al interlocutor, promoveremos la libertad de expresión, cuidando apreciaciones relacionadas con la política, religión, ética o ámbitos sensibles.
    \item No discriminación: la Organización no tolerará la discriminación (incluida la edad, el sexo, la raza, el color, el credo, la religión, el origen étnico, la nacionalidad, la ciudadanía, la discapacidad o el estado civil, el ciberacoso, la violencia de género, la LGTBIfobia o cualquier otra protegida o legalmente reconocida por el estado, leyes locales o internacionales).
    \item Valor: se aportará valor en la difusión de contenidos, no posicionándose como expertos en aquellos ámbitos en los que no se tiene experiencia.
    \item Protección de menores: la Organización estará comprometida con la protección de los menores y en las comunicaciones cuidará que no se atente contra los derechos de los menores y jóvenes, en especial el derecho a la protección de sus datos personales.
    \item Seguridad: en el acceso a las cuentas corporativas de redes sociales, blog y web, se tendrá en cuenta las normas aprobadas en el ámbito del Sistema de Gestión de Seguridad de la Información.
    \item Responsabilidad: en ningún caso las opiniones emitidas por los empleados a través de las redes sociales o comunidades digitales representarán la opinión de la entidad, y serán en exclusiva responsabilidad de quien las emite.
    \item Igualdad: se promoverá una referencia igualitaria a hombres y mujeres fomentando la igualdad de géneros.
\end{itemize}

\item \textbf{Criterios para tener en cuenta en las publicaciones}
\begin{itemize}
    \item Se evitará mezclar mensajes difundidos a nombre de la Organización con aquéllos difundidos a título particular.
    \item Si se asiste a eventos relacionados con el trabajo en \Beneficiario{}, y si se tratan temas relacionados con la operativa del trabajo, se divulgará la pertenencia a la entidad.
    \item No se publicará ni divulgará información que se considere de uso interno o confidencial en cualquier grado.
    \item Se tendrá cuidado con la información personal que se comparte.
    \item No se publicará información personal de clientes ni sobre proyectos que se estén desarrollando, salvo que expresamente se haya autorizado (para lo cual se deberá de disponer de alguna evidencia, siendo suficiente un email) y sea aprobada internamente su publicación.
    \item No se compartirán imágenes personales sin obtener el consentimiento o se asegure tener la legitimación necesaria.
    \item Se publicarán creaciones con imágenes adquiridas o descargadas exclusivamente de un banco de imágenes libres o que hayan sido obtenidas directamente por la entidad, garantizando el respeto a la propiedad intelectual.
    \item En ningún caso se difundirán mensajes ofensivos o difamatorios hacia clientes, empleados u otras personas, ni se utilizará un lenguaje insultante, provocador o que incite al odio.
\end{itemize}

Recuerde: si duda sobre si puede hablar sobre un tema laboral, como la casuística de un cliente, trasládalo a los responsables de Comunicación, RRHH o a Dirección.

\item \textbf{Condiciones de uso y normas de participación establecidas}
\begin{itemize}
    \item Utilice las redes sociales de acuerdo con su finalidad y en ningún caso para fines ilegales, comerciales o publicitarios.
    \item No se haga pasar por otra persona o produciendo engaño sobre la relación con otra persona o entidad.
    \item Respete las opiniones y las manifestaciones del resto de participantes, y mantenga una actitud correcta y un lenguaje respetuoso y no ofensivo.
    \item Respete el derecho al honor, a la intimidad personal y familiar, y a la propia imagen del resto de participantes.
    \item Respete la propiedad intelectual ajena, de acuerdo con la legislación vigente en materia de propiedad intelectual e industrial.
    \item Respete las normas de participación propias de cada red social en la que se participa.
    \item \Beneficiario{} se reserva el derecho de solicitar la eliminación, bloqueo, cancelación, suspender o no validar contenidos que puedan ser contrarios a esta política.
\end{itemize}

\item \textbf{Sistema disciplinario interno}

En caso de que no se apliquen los principios y normas de actuación expuestos en este documento se podrá:

\begin{itemize}
    \item Causar problemas legales a la Organización con clientes o colaboradores.
    \item Causar un daño reputacional que impacte en la posibilidad de conseguir y mantener clientes.
    \item En todo caso, \Beneficiario{} podría adoptar medidas legales contra el empleado, las cuales podría llegar a contemplar el despido ante una negligencia grave.
\end{itemize}

\end{enumerate}

% !TEX root = ../Guia SGSI.tex

\subsubsection{Procedimiento de protección de seguridad con mecanismos específicos anti-ransomware}

El ransomware es una extorsión que se realiza a través de un malware que se introduce en los dispositivos: ordenadores, portátiles y dispositivos móviles. Este software malicioso secuestra la información, impidiendo el acceso a la misma generalmente cifrándola, y solicitando un rescate a cambio de su liberación.

Para evitar un incidente de ransomware se deben seguir las normativas establecidas previamente y hacer copias de seguridad periódicas y comprobar que es posible restaurarlas.

\begin{enumerate}[label=\alph*)]

\item \textbf{Copias de seguridad}

\Beneficiario{} debe implementar un procedimiento para ordenar y regular la realización periódica y el almacenamiento de copias de seguridad de la información para garantizar la continuidad de sus actividades y la integridad de su información en caso de desastre.

El procedimiento debe asegurar que:

\begin{itemize}
    \item Sea posible su recuperación ante su pérdida, robo, destrucción, corrupción, falsificación o modificación surgida como consecuencia de desastres, tratamientos no autorizados o fallos de los sistemas informáticos o errores humanos durante el tratamiento de la información.
    \item Se preserve la confidencialidad e integridad de la información durante el tratamiento del que es objeto en la realización y gestión de las copias de seguridad y, en su caso, durante el tránsito de las copias de seguridad. Para lo cual debe ser cifrado por la herramienta de copias.
    \item Se mantenga la información, para asegurar la conformidad con la legislación aplicable y los compromisos adquiridos con terceros, así como permitir la defensa ante eventuales reclamaciones de terceros y contar con elementos de prueba en caso de ejercitar acciones en sede jurisdiccional o voluntaria.
\end{itemize}

\item \textbf{Pruebas de recuperación}

El Responsable de Sistemas de \Beneficiario{} verifica diariamente que las diversas operaciones de copia de seguridad realizadas desde la jornada laboral anterior se han producido de la forma, con la precisión y en el tiempo esperado. Para facilitar esta tarea, si una copia de seguridad ha fallado, se notifica al responsable para que pueda remediar esta situación inmediatamente.

Además, se asegura la realización de recuperaciones de información de forma aleatoria, de cara a comprobar el correcto funcionamiento de los sistemas de copia, así como el estado de los soportes en los que se almacenan las copias.

El Responsable de Sistemas de Información suspende temporalmente esta medida de precaución cuando la frecuencia y dispersión de los procesos de recuperación sea tal que demuestre fehacientemente un correcto funcionamiento de los sistemas de copia y de los soportes utilizados.

En cualquier caso, no transcurren más de 6 meses sin realizar procesos de recuperación real o de verificación en los sistemas de copia de seguridad. A la vista de las incidencias que puedan detectarse, podrá ser necesario reconfigurar el hardware o el software de los sistemas de copias de seguridad o incluso revisar la estrategia de salvaguarda.

\item \textbf{Recuperación puntual de la información}

Cuando, como consecuencia de fallos de hardware o software, errores humanos o cualquier otra causa, se haya visto afectada la integridad o la disponibilidad de una información y sea necesaria su recuperación a un estado anterior lo más próximo posible al momento de la incidencia, \Beneficiario{} dispone de un proceso basado en los siguientes puntos:

\begin{itemize}
    \item El titular de la información a recuperar, o una persona autorizada, notifica la incidencia de acuerdo con el Procedimiento de Gestión ante Incidentes de Seguridad, a las personas en las que el Responsable de Sistemas de Información delegue las tareas de recuperación de información, indicando:
    \begin{itemize}
        \item Los activos de información a recuperar.
        \item La ubicación exacta de los activos de información en cuestión.
        \item La fecha y hora, así como la causa de la pérdida de integridad o disponibilidad de la información.
        \item Las personas responsables de las tareas de recuperación:
        \begin{itemize}
            \item Comprobarán que el solicitante está autorizado a solicitar la recuperación de la información.
            \item Localizarán la copia de seguridad que contiene la versión de la información a recuperar más cercana a la incidencia, requiriendo la colaboración del solicitante en caso de duda.
            \item Restaurarán la información, ubicándola en su lugar de origen.
            \item Comunicarán al solicitante el estado final de la operación de recuperación, indicando, en caso positivo, la fecha y hora de generación de la copia utilizada.
        \end{itemize}
    \end{itemize}
\end{itemize}

La notificación y gestión de las incidencias de recuperación de la información sigue los cauces establecidos en el Procedimiento de gestión ante incidentes de seguridad.

\item \textbf{Recuperación en caso de desastre}

El procedimiento de recuperación de información en caso de pérdida masiva de integridad o disponibilidad provocada por una incidencia o contingencia de carácter grave se contempla en el Plan de continuidad de negocio.

\end{enumerate}

% !TEX root = ../Guia SGSI.tex

\subsubsection{Procedimiento de actualización y parcheo periódico de software}

\begin{enumerate}[label=\alph*)]

\item \textbf{Análisis de vulnerabilidades}

La aplicación de parches y actualizaciones en los sistemas de información es una medida de protección frente a debilidades conocidas en los sistemas. Estas debilidades podrían ser explotadas de forma maliciosa o provocar fallos en el normal funcionamiento ante la materialización de determinadas situaciones.

Se puede requerir la actualización de sistemas o aplicación de parches en los siguientes casos:

Avisos de fabricantes y proveedores, a nivel de:
\begin{itemize}
    \item Hardware
    \item Software
    \item Sistema Operativo
    \item Bases de Datos
    \item Identificación de vulnerabilidades no solventadas en los sistemas implantados.
    \item Como parte de una operación de mantenimiento preventivo periódico.
    \item Como resultado de acciones correctivas tras la detección de incidentes.
\end{itemize}

En servidores con cualquier Sistema Operativo, \Beneficiario{} utiliza una herramienta para gestionar actualizaciones automáticas manteniendo de esta forma un calendario de parcheos (SCCM) donde se establece el momento en el que deben instalarse actualizaciones en los servidores.

\Beneficiario{} fija un plazo mensual de 30 días para realizar la operativa, salvo en el caso de vulnerabilidades críticas que requieran parcheado urgente, en cuyo caso se procede de inmediato o de la manera más rápida posible, garantizando eso si el proceso de vuelta atrás por si fuera necesario aplicarlo.

Los técnicos asignados a esta operativa no están informados sobre las últimas vulnerabilidades, actualizaciones y parches publicados ni están suscritos a listas de información o boletines donde se publica información diariamente sobre las ultimas vulnerabilidades incorporadas al repositorio. \Beneficiario{} debe un implantar un sistema de comunicación con diversas fuentes para estar al tanto de las últimas vulnerabilidades y parches y evitar la explotación de las mismas con fines maliciosos.

Una vez detectada la necesidad de actualización o parcheo se procede a la localización del paquete software recomendado por la fuente de información origen.

Siempre que sea posible se utilizan parches o actualizaciones suministrados por el vendedor o fabricante del activo involucrado. En ningún caso se utilizan paquetes software de dudosa procedencia o que no estén recomendados por el vendedor o fabricante de forma oficial.

\item \textbf{Aplicación de parches}

\Beneficiario{} debe implementar una norma de actualizaciones y parcheos siguiendo las indicaciones que se mostrarán a continuación.

Puestos de trabajo:

\begin{itemize}
    \item Identificar los parches necesarios para cada puesto de trabajo.
    \item Descargar los parches de una fuente confiable.
    \item Distribuir los parches a través de una solución de gestión de parches.
    \item Instalar los parches en cada puesto de trabajo, esto puede hacerse automáticamente o requerir la intervención del usuario.
    \item Verificar que los parches se hayan instalado correctamente en cada puesto de trabajo.
\end{itemize}

Electrónica de red:

\begin{itemize}
    \item Identificar los parches necesarios para cada dispositivo de red.
    \item Descargar los parches del fabricante del dispositivo.
    \item Probar los parches en un entorno de prueba si es posible.
    \item Aplicar los parches a los dispositivos de red durante una ventana de mantenimiento para minimizar la interrupción del servicio.
    \item Verificar que los parches se hayan instalado correctamente y que los dispositivos de red funcionen correctamente después de la instalación.
\end{itemize}

Todas las actualizaciones siguen una planificación de antemano, teniendo en cuenta si estas actualizaciones serán de carácter urgente o no.

En caso de requerir de actualizaciones de emergencia, y que estas no puedan disponer de una planificación, se tiene en cuenta las siguientes consideraciones:

\begin{itemize}
    \item \textbf{Comunicación:} se informa a todas las partes interesadas sobre la necesidad del parche de emergencia, incluyendo el motivo, el tiempo estimado y el impacto potencial.
    \item \textbf{Evaluación de riesgos:} se identifican los riesgos asociados con la aplicación del parche, así como los riesgos de no aplicarlo.
    \item \textbf{Pruebas:} si es posible, se prueba el parche en un entorno de prueba antes de implementarlo. En una situación de emergencia, esto puede no ser posible, pero siempre es preferible cuando se puede hacer.
    \item \textbf{Plan de respaldo:} se prepara un plan de respaldo en caso de que algo salga mal. Esto podría incluir una copia de seguridad del sistema o datos, o tener un plan para revertir el parche si causa problemas.
    \item \textbf{Tiempo de inactividad:} algunos parches pueden requerir tiempo de inactividad, por lo que se planifica este tiempo para minimizar las posibles interrupciones.
    \item \textbf{Documentación:} se documenta todo el proceso, incluyendo por qué se necesita el parche, qué acciones se realizarán, y cualquier problema encontrado y cómo se ha resuelto.
    \item \textbf{Verificación:} después de aplicar el parche se verifica que todo funcione correctamente y que el problema que se suponía que el parche debía resolver se ha solucionado.
\end{itemize}

En caso de ser necesaria una parada del servicio que implique la correcta ejecución de la actividad corporativa, ésta se hace preferiblemente fuera del horario laboral y, en cualquier caso, cuando el impacto de dicha parada sea el menor posible. Si la parada del servicio no pudiese realizarse sin repercusión en la actividad normal, se comunica a todo el personal afectado. Esta comunicación incluye la fecha prevista para la actuación y una estimación de la duración de la parada.

Se establece un orden y prioridad en los activos a los cuales se aplique una actualización, suponiendo en la mayoría de los casos que se dará prioridad a aquellos con más riesgo o que contengan información sensible o estratégica.

Una vez realizado el cambio se comprueba que los requisitos de seguridad del sistema implicado no se han visto afectados por el mismo, realizando las correcciones oportunas en caso contrario.

\item \textbf{Control del cambio, proceso de vuelta atrás}

Antes de la instalación de un parche o actualización, \Beneficiario{} garantiza la restauración de los sistemas implicados a la situación anterior a la instalación de este, en lo que se conoce como Proceso de Vuelta Atrás o “Rollback”.

El mecanismo de recuperación consiste en una copia de respaldo del servidor realizada con anterioridad, o bien en una secuencia de acciones a llevar a cabo que permita devolver el sistema a su estado anterior.

En cualquier caso, esta secuencia deberá estar debidamente documentada.

\end{enumerate}

\end{document}