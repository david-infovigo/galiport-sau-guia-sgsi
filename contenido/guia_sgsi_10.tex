% !TEX root = ../Guia SGSI.tex

\subsubsection{Normativa de uso de redes sociales}

\begin{enumerate}[label=\alph*)]

\item \textbf{Principios}

\Beneficiario{} promueve la participación en redes sociales y comunidades digitales, teniendo en cuenta los siguientes principios:
\begin{itemize}
    \item Legalidad: en las publicaciones que se realicen se tendrá en cuenta las normas nacionales, internacionales, las normas internas aprobadas por la entidad y en especial la normativa de propiedad intelectual, protección de datos personales y la Declaración Universal de Derechos Humanos.
    \item Respeto: se respetará al interlocutor, promoveremos la libertad de expresión, cuidando apreciaciones relacionadas con la política, religión, ética o ámbitos sensibles.
    \item No discriminación: la Organización no tolerará la discriminación (incluida la edad, el sexo, la raza, el color, el credo, la religión, el origen étnico, la nacionalidad, la ciudadanía, la discapacidad o el estado civil, el ciberacoso, la violencia de género, la LGTBIfobia o cualquier otra protegida o legalmente reconocida por el estado, leyes locales o internacionales).
    \item Valor: se aportará valor en la difusión de contenidos, no posicionándose como expertos en aquellos ámbitos en los que no se tiene experiencia.
    \item Protección de menores: la Organización estará comprometida con la protección de los menores y en las comunicaciones cuidará que no se atente contra los derechos de los menores y jóvenes, en especial el derecho a la protección de sus datos personales.
    \item Seguridad: en el acceso a las cuentas corporativas de redes sociales, blog y web, se tendrá en cuenta las normas aprobadas en el ámbito del Sistema de Gestión de Seguridad de la Información.
    \item Responsabilidad: en ningún caso las opiniones emitidas por los empleados a través de las redes sociales o comunidades digitales representarán la opinión de la entidad, y serán en exclusiva responsabilidad de quien las emite.
    \item Igualdad: se promoverá una referencia igualitaria a hombres y mujeres fomentando la igualdad de géneros.
\end{itemize}

\item \textbf{Criterios para tener en cuenta en las publicaciones}
\begin{itemize}
    \item Se evitará mezclar mensajes difundidos a nombre de la Organización con aquéllos difundidos a título particular.
    \item Si se asiste a eventos relacionados con el trabajo en \Beneficiario{}, y si se tratan temas relacionados con la operativa del trabajo, se divulgará la pertenencia a la entidad.
    \item No se publicará ni divulgará información que se considere de uso interno o confidencial en cualquier grado.
    \item Se tendrá cuidado con la información personal que se comparte.
    \item No se publicará información personal de clientes ni sobre proyectos que se estén desarrollando, salvo que expresamente se haya autorizado (para lo cual se deberá de disponer de alguna evidencia, siendo suficiente un email) y sea aprobada internamente su publicación.
    \item No se compartirán imágenes personales sin obtener el consentimiento o se asegure tener la legitimación necesaria.
    \item Se publicarán creaciones con imágenes adquiridas o descargadas exclusivamente de un banco de imágenes libres o que hayan sido obtenidas directamente por la entidad, garantizando el respeto a la propiedad intelectual.
    \item En ningún caso se difundirán mensajes ofensivos o difamatorios hacia clientes, empleados u otras personas, ni se utilizará un lenguaje insultante, provocador o que incite al odio.
\end{itemize}

Recuerde: si duda sobre si puede hablar sobre un tema laboral, como la casuística de un cliente, trasládalo a los responsables de Comunicación, RRHH o a Dirección.

\item \textbf{Condiciones de uso y normas de participación establecidas}
\begin{itemize}
    \item Utilice las redes sociales de acuerdo con su finalidad y en ningún caso para fines ilegales, comerciales o publicitarios.
    \item No se haga pasar por otra persona o produciendo engaño sobre la relación con otra persona o entidad.
    \item Respete las opiniones y las manifestaciones del resto de participantes, y mantenga una actitud correcta y un lenguaje respetuoso y no ofensivo.
    \item Respete el derecho al honor, a la intimidad personal y familiar, y a la propia imagen del resto de participantes.
    \item Respete la propiedad intelectual ajena, de acuerdo con la legislación vigente en materia de propiedad intelectual e industrial.
    \item Respete las normas de participación propias de cada red social en la que se participa.
    \item \Beneficiario{} se reserva el derecho de solicitar la eliminación, bloqueo, cancelación, suspender o no validar contenidos que puedan ser contrarios a esta política.
\end{itemize}

\item \textbf{Sistema disciplinario interno}

En caso de que no se apliquen los principios y normas de actuación expuestos en este documento se podrá:

\begin{itemize}
    \item Causar problemas legales a la Organización con clientes o colaboradores.
    \item Causar un daño reputacional que impacte en la posibilidad de conseguir y mantener clientes.
    \item En todo caso, \Beneficiario{} podría adoptar medidas legales contra el empleado, las cuales podría llegar a contemplar el despido ante una negligencia grave.
\end{itemize}

\end{enumerate}