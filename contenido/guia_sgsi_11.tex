% !TEX root = ../Guia SGSI.tex

\subsubsection{Procedimiento de protección de seguridad con mecanismos específicos anti-ransomware}

El ransomware es una extorsión que se realiza a través de un malware que se introduce en los dispositivos: ordenadores, portátiles y dispositivos móviles. Este software malicioso secuestra la información, impidiendo el acceso a la misma generalmente cifrándola, y solicitando un rescate a cambio de su liberación.

Para evitar un incidente de ransomware se deben seguir las normativas establecidas previamente y hacer copias de seguridad periódicas y comprobar que es posible restaurarlas.

\begin{enumerate}[label=\alph*)]

\item \textbf{Copias de seguridad}

\Beneficiario{} debe implementar un procedimiento para ordenar y regular la realización periódica y el almacenamiento de copias de seguridad de la información para garantizar la continuidad de sus actividades y la integridad de su información en caso de desastre.

El procedimiento debe asegurar que:

\begin{itemize}
    \item Sea posible su recuperación ante su pérdida, robo, destrucción, corrupción, falsificación o modificación surgida como consecuencia de desastres, tratamientos no autorizados o fallos de los sistemas informáticos o errores humanos durante el tratamiento de la información.
    \item Se preserve la confidencialidad e integridad de la información durante el tratamiento del que es objeto en la realización y gestión de las copias de seguridad y, en su caso, durante el tránsito de las copias de seguridad. Para lo cual debe ser cifrado por la herramienta de copias.
    \item Se mantenga la información, para asegurar la conformidad con la legislación aplicable y los compromisos adquiridos con terceros, así como permitir la defensa ante eventuales reclamaciones de terceros y contar con elementos de prueba en caso de ejercitar acciones en sede jurisdiccional o voluntaria.
\end{itemize}

\item \textbf{Pruebas de recuperación}

El Responsable de Sistemas de \Beneficiario{} verifica diariamente que las diversas operaciones de copia de seguridad realizadas desde la jornada laboral anterior se han producido de la forma, con la precisión y en el tiempo esperado. Para facilitar esta tarea, si una copia de seguridad ha fallado, se notifica al responsable para que pueda remediar esta situación inmediatamente.

Además, se asegura la realización de recuperaciones de información de forma aleatoria, de cara a comprobar el correcto funcionamiento de los sistemas de copia, así como el estado de los soportes en los que se almacenan las copias.

El Responsable de Sistemas de Información suspende temporalmente esta medida de precaución cuando la frecuencia y dispersión de los procesos de recuperación sea tal que demuestre fehacientemente un correcto funcionamiento de los sistemas de copia y de los soportes utilizados.

En cualquier caso, no transcurren más de 6 meses sin realizar procesos de recuperación real o de verificación en los sistemas de copia de seguridad. A la vista de las incidencias que puedan detectarse, podrá ser necesario reconfigurar el hardware o el software de los sistemas de copias de seguridad o incluso revisar la estrategia de salvaguarda.

\item \textbf{Recuperación puntual de la información}

Cuando, como consecuencia de fallos de hardware o software, errores humanos o cualquier otra causa, se haya visto afectada la integridad o la disponibilidad de una información y sea necesaria su recuperación a un estado anterior lo más próximo posible al momento de la incidencia, \Beneficiario{} dispone de un proceso basado en los siguientes puntos:

\begin{itemize}
    \item El titular de la información a recuperar, o una persona autorizada, notifica la incidencia de acuerdo con el Procedimiento de Gestión ante Incidentes de Seguridad, a las personas en las que el Responsable de Sistemas de Información delegue las tareas de recuperación de información, indicando:
    \begin{itemize}
        \item Los activos de información a recuperar.
        \item La ubicación exacta de los activos de información en cuestión.
        \item La fecha y hora, así como la causa de la pérdida de integridad o disponibilidad de la información.
        \item Las personas responsables de las tareas de recuperación:
        \begin{itemize}
            \item Comprobarán que el solicitante está autorizado a solicitar la recuperación de la información.
            \item Localizarán la copia de seguridad que contiene la versión de la información a recuperar más cercana a la incidencia, requiriendo la colaboración del solicitante en caso de duda.
            \item Restaurarán la información, ubicándola en su lugar de origen.
            \item Comunicarán al solicitante el estado final de la operación de recuperación, indicando, en caso positivo, la fecha y hora de generación de la copia utilizada.
        \end{itemize}
    \end{itemize}
\end{itemize}

La notificación y gestión de las incidencias de recuperación de la información sigue los cauces establecidos en el Procedimiento de gestión ante incidentes de seguridad.

\item \textbf{Recuperación en caso de desastre}

El procedimiento de recuperación de información en caso de pérdida masiva de integridad o disponibilidad provocada por una incidencia o contingencia de carácter grave se contempla en el Plan de continuidad de negocio.

\end{enumerate}