% !TEX root = ../Guia SGSI.tex

\subsubsection{Gestión de acceso de usuarios}

\begin{enumerate}[label=\alph*)]

    \item \textbf{Registro de usuarios:} \Beneficiario{} deberá implantar las pautas dictadas por esta normativa mediante procedimientos formales en materia de registro de usuarios y gestión de permisos que garanticen el acceso de los usuarios a la información y sistemas para los cuales estén autorizados. Como regla general, estos procedimientos deberán hacer referencia a los siguientes aspectos:

    \begin{itemize}
        \item El uso de identificadores de usuarios genéricos siempre debe ser excepcional y estar justificado y documentado en todos los casos. \Beneficiario{} debe aplicar una política de usuarios con identificador único para localizarlo con claridad y evitar suplantaciones de identidad.
        \item Se verifica que el usuario tiene la autorización del propietario o responsable del sistema para acceder a la información y/o al sistema para su tratamiento, antes de facilitarle el acceso. Siempre deberá ser el responsable del sistema quien apruebe los cambios en derechos y permisos de acceso.
        \item Se garantiza que el acceso no será efectivo hasta que se hayan completado los procedimientos de autorización y registro.
        \item Se mantiene un registro actualizado de las personas autorizadas para usar un servicio concreto.
        \item Se verifica que el nivel de acceso concedido es el adecuado para la actividad a realizar, debiendo verificar adicionalmente que cumple con lo establecido por la normativa de seguridad de la organización.
        \item Se informa al usuario de la normativa aplicable sobre control de acceso a la información y/o sistema para su tratamiento.
        \item \Beneficiario{} no retira de forma inmediata los derechos de acceso de aquellos usuarios que dejen la organización. Del mismo modo, si existe un cambio de puesto, no se retiran los permisos anteriores para asignar los nuevos.
        \item \Beneficiario{} revisa y elimina de forma periódica los identificadores de usuario y cuentas obsoletas por inactividad. La Organización debe implantar un procedimiento para eliminar cuentas de usuarios que dejan la organización o con largos periodos de inactividad confirmada para evitar el uso malicioso de credenciales robadas a miembros que ya no forman parte de la Organización.
        \item \Beneficiario{} no garantiza la inexistencia de identificadores de usuario duplicados y debe definir un procedimiento para evitar incoherencias y potenciales suplantaciones de identidad.
    \end{itemize}

    \item \textbf{Alta de usuarios:} Deberán desarrollarse procedimientos formales de alta de usuarios para cada sistema, entorno o aplicación incluidos en el alcance de este documento. Estos procedimientos deberán respetar la normativa vigente en materia de control de accesos e incluirán al menos estos aspectos:

    \begin{itemize}
        \item Mecanismos de control en la asignación de permisos de acceso que permitan evitar los controles del sistema.
        \item Uso de identificadores de usuario únicos que permitan vincular a los usuarios con sus acciones y responsabilizarles de las mismas.
        \item Mecanismos para verificar la adecuación del nivel de acceso y su consistencia con la política de seguridad de la información vigente.
        \item Deberán garantizar que no se accederá al servicio hasta que se hayan completado los procedimientos de autorización.
        \item Deberán contar con el mantenimiento de un registro formalizado de todos los usuarios registrados para usar el servicio.
    \end{itemize}

    \item \textbf{Baja de usuarios:} Deberán desarrollarse procedimientos formales de baja de usuarios para cada sistema, entorno o aplicación incluidos en el alcance de \Beneficiario{}. Estos procedimientos deberán respetar la normativa vigente en materia de control de accesos e incluirán al menos los siguientes aspectos:

    \begin{itemize}
        \item La eliminación inmediata de las autorizaciones de acceso a los usuarios que dejen la Organización o cambien de puesto de trabajo.
        \item El volcado de la información del usuario a un sistema de almacenado seguro con acceso restringido al responsable de usuario.
        \item Deberán incluir la revisión periódica y eliminación de identificadores y cuentas de usuario redundantes y/o inactivas.
        \item Se deberá contar con un registro de bajas de usuarios por sistema.
    \end{itemize}

    \item \textbf{Modificación de permisos de usuarios:} Algunos de los supuestos de modificación de permisos serán: cambio de departamento, de puesto de trabajo, modificación del software o permisos de acceso asignados. Todas las demás modificaciones se tratarán como nuevas altas o bajas. También en este caso, se desarrollarán procedimientos formales de modificación de los permisos de acceso de los usuarios para cada sistema, entorno o aplicación en alcance. Estos procedimientos respetarán la normativa vigente en materia de control de accesos e incluirán al menos:

    \begin{itemize}
        \item La modificación inmediata de los permisos de acceso de los usuarios que cambien de departamento o puesto de trabajo dentro de \Beneficiario{}.
        \item La revisión periódica de los permisos y privilegios de los usuarios por sistema.
        \item El registro de las modificaciones de permisos de usuarios por sistema.
    \end{itemize}

    \item \textbf{Gestión de privilegios}

    \Beneficiario{} deberá implantar las pautas dictadas por esta normativa mediante procedimientos formales en materia de gestión de privilegios que garanticen el acceso de los usuarios a la información y sistemas para los cuales estén autorizados. Estos procedimientos deben respetar la normativa vigente en materia de control de accesos e incluirán al menos:

    \begin{itemize}
        \item La identificación de los privilegios asociados a cada elemento del sistema, por ejemplo, el sistema operativo, el sistema gestor de base de datos y cada aplicación, así como las categorías de empleados que necesiten de ellos.
        \item La asignación de privilegios a los individuos según los principios de “necesidad de su uso” y “caso por caso”. Por ejemplo, el requisito mínimo necesario para cumplir su función y solo cuando sea necesario realizarla.
        \item El mantenimiento de un proceso de autorización y un registro de todos los privilegios que se asignen, teniendo en cuenta que no se otorgarán privilegios hasta que el proceso de autorización haya concluido.
        \item Se promoverá el desarrollo y uso de rutinas del sistema para evitar la asignación de privilegios a los usuarios.
    \end{itemize}

    \item \textbf{Procedimiento de gestión de usuarios}

    \Beneficiario{} no dispone de un procedimiento de gestión de usuarios. A continuación de definen los procesos que \Beneficiario{} debe seguir para su implantación.

    \begin{itemize}
        \item \textbf{Alta usuario}

        En el momento en que el usuario precise acceso al sistema de información o se incorpore a su puesto en el área correspondiente, su responsable definirá sus necesidades tecnológicas, pudiendo incluir el alta en ciertos sistemas, así como la asignación de las herramientas de trabajo necesarias (ordenador, teléfono, etc.):

        \begin{enumerate}[label=\arabic*.]
            \item Solicitud. El Responsable del Departamento del usuario del que se solicite el alta, enviará una solicitud de alta vía correo electrónico al Responsable de Sistemas de la Organización indicándole las necesidades del nuevo Usuario, e incluirá, entre otros, los siguientes datos:
            \begin{itemize}
                \item Nombre y apellidos.
                \item NIF.
                \item Puesto de trabajo (con indicación de la denominación de éste y las unidades administrativas a las que pertenece).
                \item Ubicación (señalando edificio, planta, en su caso, sala o despacho).
                \item Tiempo que va a ocupar su puesto (indefinido o temporal).
                \item Sistemas a los que va a necesitar acceso y perfil necesario (los correspondientes permisos de acceso a recursos del sistema).
                \item Activos que va a necesitar (equipo informático, teléfono, etc.).
            \end{itemize}
            \item Firma de la normativa: El Responsable de Sistemas remitirá el Responsable del Departamento el documento en el que se relacionarán los deberes y responsabilidades de los Usuarios en relación con la información accedida (deber de confidencialidad), y al régimen disciplinario en caso de incumplimiento. Esta normativa será firmada por el Usuario y devuelta al Responsable de Sistemas para su custodia.
            \item Alta en sistemas y/o asignación de equipamiento: El Responsable de Recursos Humanos indicará al Responsable de Seguridad que proceda a su ejecución.
            \begin{itemize}
                \item En caso de que la credencial tenga una fecha de expiración, se activará la cuenta durante el tiempo que haya establecido el Responsable del Departamento (si el sistema lo permite).
                \item El Responsable de Seguridad realizará un seguimiento continuo de las fechas de expiración de las cuentas que tengan un periodo establecido.
            \end{itemize}
            \item Confirmación del alta y/o asignación de equipamiento: Una vez dado de alta el Usuario y/o proporcionado el equipamiento, el Responsable de Sistemas o el personal del equipo, enviará un correo electrónico al Responsable de Departamento que solicitó el alta, y le informará de que se ha ejecutado correctamente.
        \end{enumerate}

        \item \textbf{Baja usuario}

        El proceso a seguir se detalla a continuación:

        \begin{enumerate}[label=\arabic*.]
            \item Comunicación. El Responsable de Recursos Humanos informará al Responsable de Seguridad de la fecha de baja del usuario, al objeto de que, a partir de dicha fecha, se revoque el permiso de acceso físico a las instalaciones y lógico a los sistemas.
            \item Localización de los activos. El Responsable de Departamento indicará al Responsable de Recursos Humanos los activos de la Organización que el Usuario tiene a su disposición.
            \item Ejecución de la baja y/o retirada de activos. El Responsable de Seguridad ejecutará directamente la baja o, en su caso, indicará Administrador de cada sistema que proceda a su ejecución.
            \item Comunicación de la baja. Una vez confirmado que se ha dado de baja al Usuario y/o retirado los activos de la Organización, el Responsable de Recursos Humanos enviará la confirmación de la baja completa al Responsable del Departamento que la solicitó.
        \end{enumerate}

        \item \textbf{Modificación de permisos}

        La modificación de los derechos o permisos de acceso de un Usuario seguirá el procedimiento definido para el alta de Usuario.

        \item \textbf{Gestión y revisión del procedimiento}

        La gestión de este procedimiento corresponderá al Responsable de Seguridad de la Organización, que será competente para:

        \begin{itemize}
            \item Interpretar las dudas que puedan surgir en su aplicación.
            \item Proceder a su revisión, cuando sea necesario para actualizar su contenido.
            \item Verificar su efectividad.
        \end{itemize}

        Anualmente (o con menor periodicidad, si existen circunstancias que así lo aconsejen), el Responsable de Seguridad revisará el presente procedimiento, que se someterá, de haber modificaciones, a una nueva aprobación.

        La revisión se orientará tanto a la identificación de oportunidades de mejora en la gestión de la seguridad de la información, como a la adaptación a los cambios habidos en el marco legal, infraestructura tecnológica, organización general, etc.

        Será el Responsable de Seguridad, la persona encargada de la custodia y divulgación de la versión aprobada de este documento.

    \end{itemize}

\end{enumerate}